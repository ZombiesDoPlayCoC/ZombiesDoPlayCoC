\section{“静止”,达夫·格林伍德}

达夫·格林伍德诞生于被想象出来的马里亚诺·格林伍德的子宫之中,是吉姆·格林伍德精神上的重生、吉姆真挚的愿望和感情,也是受到黄衣之王腐蚀的一面概念的实体。

但另一方面,达夫·格林伍德但又完全超越了吉姆·格林伍德的想象,是纯粹的黄衣之王概念的一角,他虚构的身体里流淌着万千的黄印,他是这个空间真正的主人。这种概念如同黄印一般,与之接触就会受到黄衣之王的腐蚀。他同时存在于与他接触过的所有意识之中,只要达夫的存在尚未被忘却,他就会不断地对他寄宿的受害者产生影响,并在受害者无意识的传播下分裂、扩散。

在被腐蚀空间之中,达夫拥有实体以及改变这个空间的绝对权限;而在这些空间之外,达夫则依附于吉姆·格林伍德的实体中。实体化的达夫·格林伍德的体型和吉姆类似,但是一大团球形的无规则振动和变化的黑色线条占据了原本应该是头部的位置,长在身体的脖颈之上。

直视达夫·格林伍德的头部是特工能做的最坏的一件事。这个黑团是可能性世界的集合,它们互相蕴含又无穷无尽:注视者会发觉自己的视线正在不由自主地聚集于其上,这些黑线在自己的视线中不断地放大:特工会发现每条黑线都是一条火车轨道,一些列车沿着这些轨道上按部就班地前进。继续放大:列车之中是形形色色的行人,这些行人在忙碌着做着各种各样的事情、天下所有可能的事情,而这些行人之中有一个自己正在盯着一团黑色的跳动曲线。继续放大,这团黑线仍然是轨道,上面运行着火车。然后一种失重的感觉朝着特工袭来——他看见自己正在一个巨大无比的、毫无规律的铁轨上,并随着无数平行且抖动着的其他巨型铁轨及铁轨上的火车,绕着一个城堡运动,不断地运动。而这些黑线,组成了无数抖动且交错的黄印。

达夫同样象征着“静止”这个概念。如果特工试图攻击达夫,特工会在产生这个想法的当下,立刻感到一种已经成功攻击过达夫的错觉,并紧接着发现自己实际上什么也没有做(例如弹夹还是满的,自己还站在原地)。但是当特工再次想要攻击时,特工会再一次产生这种错觉。特工无法真正抵达攻击达夫的事实,因为他总是以为自己已经完成了攻击——达夫是一种思维上的概念,只是特工在构想中攻击达夫时,达夫就已经遭受了特工的攻击,从而反过来影响特工的思想,使得特工产生了自己已经攻击了达夫的感受。但这如同沙砾一般思想造成的损伤不会在达夫的外表上有任何体现。发现这种悖论的特工遭受 1/1d6 的理智损失。