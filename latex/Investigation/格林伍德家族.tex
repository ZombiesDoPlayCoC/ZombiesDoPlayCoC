\section{格林伍德家族}
\subsection{吉姆失踪案}
在警署中可以找到吉姆失踪案的相关信息。吉姆的母亲约瑟芬是报案人。约瑟芬表示吉姆回到自己的房间以后再没有出来,第二日约瑟芬发现吉姆的房间中空无一人。警方没能在奥克兰市内发现吉姆任何住行的迹象,在连续数日搜查无果后,判定为失踪。

关于吉姆·格林伍德的信息则很少。吉姆·格林伍德于1985年作为早产儿出生于奥克兰市立医院,被医院标记为可能存在遗传病。在奥克兰地方志,关于格林伍德家族的结尾处有一两句提到吉姆:沉默寡言,在社交方面可能存在一些问题。如果能够找到他曾经的老师,老师只会描述他为:非常不合群,但是很喜欢画画。 

\subsection{马里亚诺自杀案}
在警署中可以找到马里亚诺自杀案的相关信息。其妻子约瑟芬和邻居是报案人和第一发现人。马里亚诺·格林伍德被认定为自杀,子弹从右脑进入,立即死亡。死者手臂上有巨大的淤青,是由某类较宽的绑带用力勒住致使。现场并未发现此类绑带,法医认定捆绑痕迹产生于死亡之前。

约瑟芬表示自己发现死者时,死者在一个完全封闭的房间中,门窗皆对外反锁,和约瑟芬一同撞开房门的邻居证明了这个事实;手枪上只找到了马里亚诺的指纹;另一方面,由于格林伍德油漆公司常年经营不善,马里亚诺罹患抑郁症数年,处于精神不正常状态。警方综合分析认为马里亚诺为自杀。

在互联网以及一些报纸刊物上很容易找到对于这个家族企业的描述:1975年的石油危机让这家油漆企业遭受重创,并且加州湾对岸冉冉升起的电子产业给奥克兰本地的传统产业带来了巨大的压力。同年,马里亚诺接手油漆厂,但他对这种现状无能为力,只能苟延残喘,并于1992年彻底破产。此后的马里亚诺不再在公众场合中抛头露面。

关于这个落魄的企业家的自杀,一些花边报纸有着“相当可靠”的猜测。《奥克兰花边》对其妻子约瑟芬横加指责,认为是妻子的外遇成为压死骆驼的最后一根稻草,导致了马里亚诺的自杀。《花边》认为约瑟芬对这个问题的回避,为这个假设成立提供了证明。