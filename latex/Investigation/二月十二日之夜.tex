\section{叙诡:二月十二日之夜}

幻象阈值:30。这个场景请尽量布置在整局游戏接近结束的时候,至少是第二次特工进入夜间的闹钟酒吧时,我们可以使用这个把戏。

在这个事件中,我们将叙述的时间拨回2月12日的夜晚,叙述的主人公也变成那时闯入演出的特工,而非玩家以为自己正在扮演的这个时刻的这位特工。主持人可以通过叙诡的形式掩藏叙述对象的变化,让玩家以为自己仍然在扮演自己以为的“那位特工”。

这种叙诡是对失真症导致的时间认知紊乱的一种模拟:特工对于时间的认识不再沿着“客观时间”流逝的方向前进,而是变得颠三倒四,过去、现在和未来的认知杂糅在一起。这种混沌的现象是深刻地理解了卡尔克萨的性质的一种表现。

\subsection{演出之前}

在演出开始之前,特工可以进行一些简单的探索。特工会发现坏掉的椅子和舞台中央的污渍不知所踪,酒吧中的客人变得正常——这些反常的现象都会让这个叙诡轻易地漏出马脚,很快,特工就会意识到自己正身处那个疯狂的夜晚。

8点左右,约瑟芬和服务生开始清空出酒吧中央的空间。就在毕宿五攀登到天空中最高处之时,演出开始了。

\textbf{补充。}至于这个场景是真的让特工重新进入了那个时空,还是只是另一种幻象,则交给掌局者决定。对于前者,即便特工拥有游戏开始之后的全部记忆,也并不会产生叙事上的悖论——他的思想受到了来自无穷可能性的扰动变得混乱不堪,他所察觉到的来自未来的记忆也是黄衣之王启示的一部分。

\subsection{预演}
演出开始,两个上身赤裸、戴着灰色盒子头套的男性从那扇铁门进入舞池。两位演员全身的皮肤上都是夜光颜料绘制的图案,舞动的肢体在昏暗的场景中拉出一道道光带。

起初他们的动作看起来原始且不协调,毫无规律可言,仿佛只是在一片空地绕着圈随意地行走。但是随着舞蹈的进行,舞者的动作浮现出难以言喻的规律性。从音箱中传出的富鲁格号声此时加入了演出。尽管两位舞者的动作越发难以琢磨,但那种诡异的规律感和一致性则更加明晰。两位舞者身体散发的光带在移动中滞留,逐渐形成了一个由符号围成的清晰的圈。

铜管的声音在这种疯狂的节奏中迷失,发出梦幻的嘶哑。其中一个舞者(吉姆)开始发出尖锐的喊叫声,另一个舞者(达夫)则发出低频的嗡嗡声,随后高频的声音逐渐降低和低频的声响一致。

此时,特工可以选择立刻逃离这里,或者通过成功的\textbf{意志*5}检定来暂时避免成为观众的一员。检定失败的特工将会接收到来自卡尔克萨的启示,进入\textbf{尤里卡时刻}。

\subsection{成功的意志检定}
通过意志检定的特工可以看到幻象之外的疯狂场景。观众跟随着这两位舞者进行相仿的喊叫,以夸张和暴力的节奏踩跺着地板,抖动着自己的身体,绕着中心的圈舞动——整个空间被协调到恐怖的声音和振动统治,一种令胸腔都疼痛的巨大低频声响在空间中游走。

这种低频声音将空间震裂,数只有翼的生物从空间的裂缝飞出。混乱无比的场景降临在这片空间:社会规范和生理学约束在这里成为了泡影,无论对象是男是女,一个还是多个,是人、墙缝还是带着翅膀的怪异生物都无所谓,他们渴望地呼唤着彼此,与彼此交合。所有交合着的生物都以精确相仿的节奏抖动着,进行着这场亵渎的活动(检定失败的特工就会成为这些人的一员)。

特工如果摆脱了仪式的影响,他可以进行一些自由活动。如果特工留在这个空间中,那么每行动两轮,都需要通过一次成功的\textbf{意志*5-20}检定以摆脱黄印的影响,如若失败就会成为仪式的一员,进入\textbf{尤里卡时刻}。这个空间中的所有生物都沉浸在神启之中,不会对特工的行为做出任何反馈。

特工对任何一个乘客发起攻击都不会终止他们的狂欢。乘客的伤口会以诡异的方式开始自愈,然后继续自己的行为。攻击神秘的有翼生物会发现深色的液体从它的身体中流出,但这也不能阻止它。

如果特工攻击吉姆,由于这个仪式产生了与卡尔克萨其他可能性的共鸣,来自其他可能性的吉姆会替代这个时空的吉姆,继续完成的他的动作:被攻击的吉姆变成黑线消散,达夫的身体涌出大量的黑线,顺着两人形成的光圈移动到吉姆的位置,逐渐编织成一个新的吉姆。目击这一现状的特工遭受 1/1d4 的理智损失。如果特工攻击达夫,那么他会亲身感受到达夫蕴含着的“静止”带来的奇景(见\textbf{达夫·格林伍德})。

使用工具记录这里发生的事情会使得它被黄衣之王腐化。其记录下来的信息不具有准确性,并伴随着传播黄衣之王的风险。

在此时呼叫援手意味着直接向外传递腐蚀(腐蚀+4),呼救的工具也会在第一次使用后就被黄衣之王腐蚀。如果特工联络绿色三角洲,绿色三角洲表示会尽快处理此事,但不是今晚。市警会同意前来救援,但也会很快被拉入这个疯狂的世界。

\textbf{注意}:如果特工联络了绿色三角洲,则会导致绿色三角洲与特工主动接触,参见\textbf{突发事件:特工被绿色三角洲怀疑}。

特工也可以选择在头脑清醒时离开这里,这会让他回到叙诡发生之前的现实世界那夜。当然特工也可以从此处进入“剧院”。

这场演出会一直持续到毕宿五落下地平线才会结束。一台被黄衣之王力量侵蚀已久的摄影机记下了整场活动(即便它被破坏,也可以借由“心想事成”而完全恢复)。

\subsection{尤里卡时刻}
意志检定失败的特工开始和“剧院”建立直接的连接,大量彼岸的知识开始涌入特工的脑海:特工开始意识到自己是卡尔克萨的居民,接到了前往“舞会”的邀请函而踏上了前往城堡的旅途,而自己正在一列前往伊提王朝宴会的火车上。

“剧院”内温和而沉静的氛围会和上一刻喧哗无比的闹钟酒吧形成巨大的反差,甚至带有一丝梦醒时的朦胧。特工的视界也开始发生变化:他开始脱离现实世界直接看见“剧院”,以及包裹它的无边的哈利湖,以及远方若隐若现的巨大城堡。这一段描绘应该充满了荒诞而突兀的揭示,在摧毁特工本有的世界观的同时,将其“失去的对于卡尔克萨的认知”交还给他。

经过这一事件,特工会直接进入“剧院”之中,他可以在“剧院”之中继续自己的探索。如果特工离开“剧院”,他会在进入叙诡之前那日的次日随便某处醒来,伴随着尚未完全消散的性高潮的愉悦——刚才的启示又像梦境一般消散了。

这个事件是对于彼岸的第一次揭示,在这个事件之后,特工可以进入彼岸。掌局者可以根据特工的背景设计富有特色的尤里卡时刻,我们这里给出一个简单的示例以供参考。

\subsubsection{示例:从造梦机醒来}
特工感觉到自己视线中的事物的轮廓开始向外扩散出黑线,舞者的灰色方盒也在空间中停滞留下一串盒子的图像——一串盒子组成的圈。突然噪声停止了,无穷的黑线覆盖了整个视线。

特工发现自己坐在“剧院”之中,他像梦醒一样回想起来:自己是伊提王朝的贵族,正坐在一列名为“剧院”的火车上,等待着火车抵达自己的终点。自己此前从某位工匠那里购买了一个特殊的机械(参见\textbf{剧院事件:造梦机}),而自己正在观察它展示的场景作为娱乐。自己仿佛是刚看到一个甜美的晚会时,就被坐在旁边的“伴侣”唤醒了。

这位伴侣会对特工在此前模组中的经历评价一番,“我从来没想到你会对这个东西这么痴迷?”如果特工在人物背景之中也有一个伴侣,这位“伴侣”也会对其轻佻地讥讽几句。

如果特工愿意的话,他可以和这位伴侣就在这车厢中温存一番:二人高潮时刻,造梦机的声响继续飘荡在车厢之中,以毫无感情的声音描述着“现实世界”中后续发生的故事:“**特工的名字**获得了极乐:一种极致的愉悦在他的身体和内心中的每个角落绽放开来,像万千游向湖面的鱼……”

