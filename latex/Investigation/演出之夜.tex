如果特工在一切开始之前没能挫败吉姆一行人的计划,那么“悖论”的最终演出会在湖滨公园举办,而现场的情况会传输到奥克兰中央广场的大屏幕上。

如果特工想要完成绿色三角洲的职责,他需要破坏奥克兰中央广场的转播,并阻止现场的演出。如果特工可以在短时间内完成第二步,那么忽略第一步不会带来太大的问题。

对于本模组,在最终场景杀死那几个手无缚鸡之力的普通人并非难事。相反,失真症会成为他最大的阻碍,前往终点的道路才是终幕的挑战所在。特工的心灵不断被腐蚀、消耗,特工对于现实的感触在逐渐被新的现实所替代,支离破碎的、交叉错乱的现实与幻象才是真正的威胁——以至于受损的心灵在半途倒下。

\section{运行终幕}

我们会将本章节的内容分为“现实”和“彼岸”两部分展开,这两部分存在一定的平行关系,而非特定的先后顺序——他们之间应该是互相交缠的。特工会在现实接近终点的过程中被反复拉入彼岸,像是正在观看不断来回切换底片的两部电影(例如\textbf{事件:最后一枪})。

你可以使用一些特殊的事件(例如通过\textbf{事件:再会亚当森}),在本章节较早的时刻就将特工第一次拉入彼岸,让他在这里进行短暂的探索。

在特工返回现实之后,特工在彼岸与现实中的感知会缠绕在一起。你可以在特工在现实中涉及到某个关键词时,就将他拉入彼岸世界,或者反之亦可。有的时候,你可以只是以闪回的形式,快速地交代另一个世界中瞬时的画面或者对话,然后就回到此前的叙事。

\textbf{注意}:这种叙事并不一定意味着特工真的在彼岸与现实间反复穿梭,而是如同\textbf{叙诡:二月十二日之夜}中那样,是特工时空认知紊乱的一种呈现——只不过现在更加严重了。因此,这两个世界间切换不需要遵从事件发生的先后顺序。但是为了保证玩家的主动性,我们建议只调整彼岸中事件的顺序(它们的先后顺序是\textbf{既定}的),而让现实发生之事按照客观的时间呈现。但是在彼岸行动的时间也总是会使得特工在现实的时间被消耗——因为这意味着特工在现实中“走神了”。最后,这种切换应该仅仅作为渲染终末氛围的一种调料,而不是故意切碎叙事的休止符。

\section{彼岸}

这是\textbf{尤里卡时刻}所揭露的世界,位于“剧院”之外的空间。相较“剧院”,它更加靠近卡尔克萨,因此难以与“现实”直接产生联系。它像是一个镜面世界,受到预演的干预而更具有“理想主义”色彩,并在混沌的时光中收留了众多旅者。

从地理条件和城市规划上来看,彼岸和奥克兰极为相似,但是完全浸没在哈利湖下。这个世界保留了“剧院”中的维多利亚式风格,以及带有一些近未来和超现实成分:满载维多利亚式装潢的铁制悬浮电车在街道之间游走,复杂的浮雕装饰镶嵌在摩天大楼的棱角处,众多铜质喇叭插在漂浮在天空中的电灯上。此外,这个世界没有白天——双月永恒悬挂在水面之上。

特工进入这个世界时,自己属于“现实”的记忆会开始模糊,并且大量属于这个世界的信息会涌入。特工在这个世界中引导着一个和“现实”区别不大的生活,不同的是,其中的不幸和矛盾尽数移除了,自己也不再是绿色三角洲的一员。(这些信息和\textbf{尤里卡时刻}揭露的应该一致。)

彼岸世界应该极力地渲染“正常”的氛围,仿佛这个世界最开始如此,最终如此,所有人皆以为它如此。

\subsection{离开彼岸}

彼岸之人需要先从这里进入“剧院”,才能再一次回到“现实”。特工只有两种方法离开彼岸:在彼岸世界找到“剧院”的站台,进入正停靠那里的“剧院”,并按照“剧院”的规则回到现实;或者特工可以在这个世界找到造梦机,借助造梦机回到现实(参见\textbf{剧院事件:造梦机})。

\subsection{再会亚当森}

这个事件可以作为特工进入彼岸的引子。它像本模组开头那样,使用敲门声打断特工本来的思绪,像是打断梦境将其唤醒一般,将其引入彼岸。同样是夜晚(这里没有白天),但这一次的来访者是亚当森·考克斯。他手上拿着一个小礼盒和一张卡片,穿着他标志性的格子衫,大大落落地站在门口。

这里的亚当森不是特工在绿色三角洲的同事,而只是“朋友”关系。

亚当森会提到自己收到了前往舞会的邀请函,并且也打算前往那里。他把手上的东西交给特工,表示自己会希望在舞会见到特工。这张卡片和特工此前在亚当森卧室书桌上找到的完全一样,礼盒中装着的则是《瓦普内克剧院》绘本。

如果特工问起他这两个东西是怎么来的,亚当森会回答这是吉姆·格林伍德寄给自己的,并且吉姆也委托自己把这两个东西都交给特工。

如果特工问亚当森他为什么要去那里,他会耸耸肩,“这可是那位王的舞会。”如果特工提出要和亚当森一同前往,亚当森会感谢特工的美意然后以想和家人一同前往作为托辞拒绝。如果特工跟踪亚当森,他会在亚当森登上一列电车之后失去他的踪迹。特工可以问起这位吉姆是谁,“我们的朋友,一个理想主义者”。

亚当森会和特工拥抱之后和特工告别。在告别的拥抱中,特工可以回忆起两人度过的无数愉悦而舒适的时光。这段友谊可以持续到天涯的尽头。

\subsection{家}

特工在彼岸的家和在“现实”中非常相似,但是更符合特工的“想象”。特工可以在这里找到他其乐融融的家人。如果特工有配偶的话,她会轻吻一下特工的脸,然后继续收拾行李,“我们什么时候前往那里?”她看上去非常激动。

\subsection{琐事}

特工确实在这里“生活”,他会在这里找到若干自己生活和存在的痕迹:他的电子邮箱里塞满了同事生日派对和婚礼的邀请,附近的街坊邻居也会热情地和特工主动攀谈……特工在街道上行走时,可能会被某个路人拦下,他亲切地和特工打招呼:先是用手拍拍肩膀,再拥抱一下,“听说你拿到了邀请函!我的天!祝贺你,祝贺你。祝福我们的王。”特工获得邀请函的消息仿佛已经被传开了。

\subsection{格林伍德宅邸}

这栋建筑现在是一个巨大的鲸鱼堡垒,房屋的内部湿滑粘腻,某种细胞组织构成了墙壁,原本是窗户的地方构成了鲸鱼的若干个眼睛,门廊组成了鲸鱼张开的嘴。一条铁轨穿过鲸鱼的尾部,连接着天际。当特工进入格林伍德宅邸时,会听见从外部传来的火车汽笛的声音,然后特工会发现“剧院”就停在铁轨之上。

\subsection{光辉的站台}

如果特工在彼岸中抵达了现实中闹钟酒吧的位置,他会发现这里现在是一个富丽堂皇的车站。连接着天际线的铁轨穿过这里,宽阔而明亮的大门欢迎着一切来者。形形色色的人潮在这里进进出出,售卖报纸和饮料的机械玩偶在门前来回游走。此时可以从两端看到一列火车停在站台里。站台找不到列车的时刻表。如果特工提出想要乘坐“剧院”,侍者会请求出示邀请函,在确认无误后会引领着特工通过站台进入“剧院”。

\subsection{但丁打了个勾}

这是一个瘦得有点皮包骨的老人,穿着一套袖口有些磨损的老旧西装。他在站台大门旁放了一个小木桌和凳子,此时他正用两手杵着拐杖坐在凳子上。桌子上摆着一台奇特的蒸汽装置(参见\textbf{剧院事件:造梦机})。

他自称但丁·阿利吉耶里,一名来自罗马的失恋者,造梦机的发明人。他本拿着王的侍者俾德丽采交给自己的邀请函,但他不小心把它遗失了。于是俾德丽采交给了他一张清单,上面写着众多人物的名字,但丁如果能将造梦机兜售给这些人物,他就能取回邀请函,前往天堂与俾德丽采重逢。

他会首先进行自我介绍,然后就询问特工的名字,\textbf{人源情报}不低于40\%的特工会从他的脸上抓到一丝转瞬即逝的快乐情绪。他会相当模糊地为特工介绍造梦机的功能,并且表示愿意提供一些胶囊让特工体验造梦机,前提是获取特工的一些血液。

但丁会以特工的血液和一段回忆作为交换造梦机的筹码。如果特工同意了,他就会从腰包里拿出一个脏兮兮的古典样式的针管,从特工的手臂抽取大概200毫升的血液——这个针眼很快就自动痊愈了。特工在被抽血时会感到一种难以言喻的不安感,如同一只手正攀上了自己的心脏。(特工可能在这个过程中将意识转换回“现实”。)

交易完成后,但丁就会将造梦机交给特工,然后拿出清单在上面划上一个勾,收起自己的摊位前往下一个站点。特工会在清单上看见自己的名字,以及一些别的名字,比如特工的重要之人。

\subsection{铜管声}

每隔一段“时间”,特工可以听到从某处传来几声若有若无的铜管乐器的声音,紧随其后就能听见从所有电灯装置着的喇叭上播放出的音乐声,周围的行人都驻足并高喊:“祝福我们的王!”特工很快就意识到这个声音也在从自己的嘴中喊出。循着最初的声音的方向,特工可以抵达彼岸的边缘。

\subsection{彼岸的边缘}

彼岸的边缘像是一个巨大海底沟壑的岸边,其旁边的海沟深不见底,黑暗得如同虚空。更远处有一层遮罩若隐若现,像若干面镜子,可以看见在对面有着另一个彼岸的痕迹。成功的 \textbf{警觉} 检定可以从遮罩对面听见一个女童重复的歌声。从城市中出发的铁轨一直向外延申,透过边缘的遮罩进入了看不见的终点。如果凝视这个遮罩,会在其上看到不计其数的流动的人脸,理智检定 0/1。

边缘的地面上埋置着众多的铜管乐器,它们的喇叭口从土地中露出朝向地面,构成了彼岸的边界线。肃穆、低沉的声音这些喇叭中奏响。如果仔细聆听其中的音乐,\textbf{艺术(音乐)}不低于 20\%的特工可以发现这就是和 St. Illness 中背景音乐完全相同的旋律,但是去除了其中的装饰音,并且节奏相较于那个背景音乐慢了很多,呈现出一种庄严感。

岸边聚集着几十名穿着灰袍戴着面具的人,他们在这里围绕成圈,在埋置的铜管乐器中穿行,并在口中念着某种类似颂歌的东西。他们心无旁骛地在这里歌唱着,完全没有注意到特工的到来。如果特工一一揭开他们的面具,会在其下发现一些陌生人和一些熟悉的面孔。“悖论”三人组都在这里,其中甚至还有一些特工已经逝去的亲人。此处在“现实”中对应着湖滨公园。

\subsection{摩天大楼}

特工意识到自己正站在彼岸最高的摩天大楼的顶楼上,在这里可以清晰地看见覆盖在整个城市之上的巨大黄印。巨型的飞行器和水生生物在水体中漂浮,浮在整个城市的上空。透过水面可以看见其上悬挂的双月。这里有形形色色的人正在交谈,碰撞着自己的酒杯。特工可以在这里找到\textbf{巴特·伯格曼}。

巴特会认出特工,并和特工开始寒暄,但他所聊的事只是一些生活琐事。巴特同样会为特工拿到邀请函表示祝贺。如果特工问起他任务的事情,巴特会露出不解的神情,“哈哈,你又在开无聊的玩笑了是吧。”

如果特工从这里跳下,他会带着失重感重新出现在彼岸的某个随机的位置,仿佛这一切都没有发生过。

\section{现实}

破碎而疯狂的幻象会干扰特工在现实中的旅途,和彼岸平静祥和的氛围形成鲜明的反差。当特工疯狂发作之时,需要进行一次\textbf{腐化值}检定,如果检定成功,特工的意识就会再一次进入彼岸或者触发彼岸事件。此外一些特殊的场合也会引发这个检定。

\subsection{奥克兰中央广场}

中央广场的四个巨幕上突然切换到了一个疑似公园的场景,背景中的草坪上放置着巨型章鱼等雕塑:巨幕中的画面正如实地转播着“悖论”的演出现场,巨幕的音响也转播着现场的声音。

广场中的人群会驻足观看这个奇特的表演,并且很快开始呼应演出,在广场中央围绕成圈,以相仿的节奏踩跺着地面。大厦的灯光开始采用相同频率的闪烁。只有恐慌的宠物犬时不时的几声喊叫破坏这个节奏。

中央广场开始和“剧院”建立联系,使得这个场景中的环境腐化值陡升。随着演出的进行,幻象开始越发夸张:火车的轨道浮现于空间之中,空间中的物体边缘出现明显的颤动黑线,更远处的环境变得模糊,突然降下的大雨将淹没地面的水平面提升到了接近十五层楼的高度。

\subsubsection{停止转播}

特工可以破坏这四块巨幕,或者闯入中控室,破坏中控室的电脑、解除皮特的控制等。管理人员不知道发生了什么,一边照着节奏敲击着电话按键,一边惊慌失措地拨打电话联络皮特·戴维斯。一个成功的\textbf{说服}或者\textbf{行政}检定就可以让慌乱而着迷的管理人员为自己开启这些中控室的门。

但是终止奥克兰中央广场的演出只能缓解燃眉之急,而真正的演出仍然在湖滨公园继续进行,并顺着互联网慢慢地蚕食现实——特工可以在 YouTube 上通过搜索关键词毫不费力地发现这个事实。

\subsection{奥克兰市的其他角落}

在这个都市的其他角落,还有少量幸运观众在互联网上看到了这个奇异的演出,他们开始在自己的房间中制造类似节奏的响动,并且开关家中的顶灯。强烈的转发欲望让黄印在互联网上迅速扩散起来。不过万幸当时的互联网并没有那么成熟,因此这一股势力短时间内很难造成恐怖的影响——尽管花不了太久。

如果特工此时走在城市之中,可以听见四处传来类似节奏的鸣笛——这一部分来自于自己的幻象,一部分来自现实。

\subsection{CLOSE TO THE EDGE}

通往湖滨公园的路程是对特工的巨大挑战:漫天的幻象掩盖着现实,摩天的黑色巨石撑破了破旧的房屋拔地而起,一些死去的贝类和藻类尸体在街道的转角累积。特工可能看不见街道上行驶的车辆,只能隐隐约约地感受到一些无形的物体冲过带起的一阵湍急的漩涡;特工眼中的人类像是玩偶一样在水中扭动,他们所发出的声音也像是透过液体传递到自己的耳朵之中,模糊而粘腻。

这些幻象会对特工的前进带来异常痛苦的阻碍,例如撞上自己已经看不见的现实世界的建筑或者车辆。你或许可以利用特工被擦身而过的车辆撞至昏厥或者别的什么机会,将特工引入“彼岸”(参见\textbf{彼岸事件:再会亚当森})。

\subsection{湖滨公园}

此时的湖滨公园看起来如同往日那般平静,“悖论”正在靠着加州湾的小广场上演出。那种熟悉的节奏从那里辐射出来,影响特工的心智。特工需要进行一次\textbf{腐化值+10\%}的检定,如果检定成功,他的腐化值+5,并被拉入彼岸。

\textbf{建议。}此时你可以明示特工的腐化值了。

\subsection{终止演出}

演出会在毕宿五从地面升起之时开始。本次演出的结构和并不相同:吉姆只身一人位于舞池的中央,除了约瑟芬和皮特,其余所有特工在酒吧中见到的顾客都会出现在这里。

在舞蹈开始时,他们身边的观众就开始自发开始那种特殊的节奏,绕着圈运动。很快更多的旁观者参与进来,但这一次完全没有疯狂的要素,庄严肃穆的氛围在空气中弥漫。富鲁格号加入进来,在这庄严之中参入了明亮色彩。由衷的喜乐攀上了所有观众的脸,而这一切也被直播的摄像机忠实地记录了下来。

在舞蹈过程中,吉姆身上的黑线开始诞生出人的形状,它会逐渐凝结成达夫。当达夫完全成型时,这里的人和物就永远与彼岸结合。特工需要赶在这之前破坏现场的摄像机,捣坏与互联网的链接,并且除掉可能会重启这场直播的人。

对仪式参与者展开攻击或者别的什么干扰,都不会激起他们的反抗——他们全身心地投入其中无暇顾及其他——在整个过程中,只有特工会成为施暴者。

只要吉姆死亡,仪式就会结束。庞大骇人的幻象从空间中褪去,参与者都会立刻瘫倒在原地。

\textbf{注意。}我们在这里设计了一些终止演出流程的展现,表现了特工来到演出现场、挤入人群、射杀吉姆的过程,但这只是结束模组的一个例子。你可以根据特工的行为构建他专属的流程,例如他打算在远处狙击这里,或者将这里直接爆破,你总是可以用一些其他要素来构建最终场景的幻觉。

\subsubsection{挤入人群}

吉姆所在的位置被其他仪式的参与者所环绕,从外面无法对吉姆开枪。特工需要挤入这些拥挤的人群前往仪式的中心。在这个过程中,他会被四面八方的声响触动,感受到一种轻松的情绪。

特工会突然注意到周围的人群都变成了披着灰袍、头戴面具之人,自己也同样如此,自己的配枪不知所踪。特工在这一瞬间可以察觉到来自另外若干个“自己”的感受在自己的心中回荡,仿佛这些“自己”就是此刻站在身边的这些灰袍人。

这是彼岸边缘,吉姆·格林伍德站在边缘之上欢迎特工的到来,其余灰袍人则主动让开了位置。他展开对于这个世界的讨论。简单来说,吉姆希望劝说特工放弃对于自己的阻挠。

吉姆会解释说现在的彼岸是不完善的,而这次演出会完善彼岸,从而达到自己的夙愿,完成一个永远存有一致目的的、毫无分歧的世界,将一切都暴露在永恒存在之下,接受其宠信,并被他所祝福。现场的人的存在会从“现实”所有人的认知中消失,并活在那个永恒幸福的世界;而不在这里的人很快就会恢复。一个成功的\textbf{人源情报}检定会发现吉姆由衷地说着这些话。

这段交谈结束后,特工会发现自己重新回到了湖滨公园的人群中。

\subsubsection{射击}

如果特工打算射杀吉姆,他不会反抗。在特工射击吉姆的过程中,我们可以放缓这段时间,插入一些别的叙事。

例如让特工短暂地回到了彼岸。掌局者可以继续塑造彼岸的温馨场景,并且让其中的时间流动越来越缓慢。特工可以在这里回到整个故事的开头——黑线行动的现场,或者是2月12日之夜。你可以在这些节点插入任意的内容,将其进行部分修改,让它呈现出温暖、感人的质感。将这些场景和吉姆被射击的场景交错在一起,直到最后一枪。

\subsubsection{叙诡:最后一枪}

这是将彼岸与“现实”经历纠缠的一个例子,用以描述在”现实“中,特工朝着吉姆·格林伍德开最后一枪之时。

此时,我们可以突然将叙述视角转回彼岸的边缘。“你感受到一种异常真实的剧痛在全身游走,你中弹了。你的眼前的景象不再是那个海边的小公园,而是紧邻着无边虚空的悬崖峭壁之上,埋在地面的铜管发出低沉的哀鸣声。你发现你的手捂着的腹部,从那里涌出黑色的油墨在水体中逸散,沾染了自己穿着的灰色长袍。这剧痛之后是什么东西倒下的声音——你躺在了地面上,看见了站着的那位持枪之人——是【特工】。你觉得自己的嘴在嗫嚅着什么,但是发不出声音。”

然后将视线再转回特工在现实的视线,“‘我好怕’,吉姆的声音在你的脑海中响起。这个声音和突然出现的画面只停留了一眨眼的时间,公园的场景重新回到了你的视线,你看见那个舞者在你面前倒下了,伴随着旁边手足无措的人员。你看见吉姆的身上升腾起一阵黑线,组成了一个扭曲的印记,然后就弥散了。”

\section{一些别的情况}

\subsection{拥抱希望}

特工可以选择同意吉姆的提议,离开这里不再干扰仪式,或者成为仪式的一员。如果他选择离开这里,参见\textbf{绿色三角洲的主动介入},他的腐化值+10。仪式会继续顺利举行。如果特工留在现场参与仪式,他的存在会从“现实”中被抹除,而永远呆在彼岸(参见\textbf{尾声:持续到永恒的愉悦})。

\subsection{联系绿色三角洲}

特工可以考虑联系自己的主管,并且在提供一定证据的情况下,由主管启用非自然事件侵染互联网的预案,给特工争取针对互联网或者电力系统的权限用以解决这次案件,如果有必要,甚至是一些非绿色三角洲的人手(如果时间允许的话,也可以从其他地方调来其他绿色三角洲小组)。申请预案权限和调用当地人手需要 2d6 * 10 分钟的时间,调集另一支绿色三角洲小组则需要 3d4 小时的时间。

抵达的小队会立刻镇压现场,使用催泪弹打断这个仪式,联络当地的警署将现场的人控制起来。参见\textbf{尾声:历史的阴影}。

\subsection{绿色三角洲的主动介入}

如果特工的行动不够迅速,黄衣之王的影响已经被奥克兰的友方注意到,他们会联络自己的接头人,接头人会迅速启用非自然事件侵染互联网的预案(这需要 2d6 * 10 分钟的时间。):他会使用专线取得极高等级的权限,以输电装置损害的名义短暂停用整个奥克兰市的电路和网路,并在这暂停之中进行迅速的排查。

特工此时会接到来自特工坎蒂的联络,以获得更加详细的情报。如果此时特工至少明确了演出的地理坐标,坎蒂会被迫调用绿色三角洲以外的权柄(此时召集绿色三角洲已经太迟)来控制这个事态的扩散,例如调用警员一同摧毁这两个地点。而如果特工此时提供了错误的情报,或者一无所知,则会让排查的时间更长。

这一切会给特工坎蒂的明面立场带来巨大的压力,也会给绿色三角洲此后的扫尾带来许多麻烦。舆论会直指停电等骚动是为了掩藏政府的一些不能启口的阴谋等等。但相比让非自然事件在互联网上迅速扩散,这样的代价是值得的。总之,绿色三角洲会想办法处理这一切,尽管十分狼狈地付出高昂的代价。

\textbf{建议}:绿色三角洲的介入只是作为模组最后的一个应急手段,以将这次风波的范围限制在不至于破坏整个世界设定的区间中。如果特工仍然没有放弃,冒着难以忍受的折磨仍然在试图拯救一切,那么我们不妨将最终期限放得稍微宽一点,让特工独自一人解决这一切。

\section{尾声}

\subsection{历史的阴影}

无论结局如何,这场灾难对于奥克兰市和其他区域的影响并不会“非常严重”。

奥克兰中央广场的行人:如果特工成功关闭了转播,那么在奥克兰中央广场的人会幸存。反之这些人的情况则会和在演出现场的人相同。

湖滨公园:如果吉姆的仪式顺利举行了,那么世界会如同吉姆所说的那样,忘记这一次仪式受害者的存在。连同整个公园都从认知中被抹除——直到一两年后的时间,这里被唐突地改建为一个新的游乐设施。如果特工还留在现实中,他会记得这些人,成为这段黄衣之王入侵历史之后的铭记者。而这个公园可能以后出现在某个历史的角落里,成为新的沟通卡尔克萨与现实的通道,吞噬着其他人。

如果仪式被中途打断,那么在演出现场的人都会成为精神病患者,被奥克兰精神病院受理。其中的部分人(比如约瑟芬和皮特)会从监禁自己的房间中消失。更严重的,这个精神病院会被黄衣之王继续污染,变成下一个闹钟酒吧。

世界的其他角落:黄印的阴影从此开始埋在互联网的角落里,成为某种意义不明的模因,并随着时间的流逝逐渐扩散开来。当日的一些群体幻觉现象得到了绿色三角洲的注意,但是无法将这个事件和特工联系起来。

\subsection{毫无起色的未来生活}

如果特工没有选择投身彼岸,且没有将现状暴露给绿色三角洲,他只会接受绿色三角洲的一次例行询问,以判断奥克兰的群体幻觉事件与特工无关;相反,面对他的会是绿色三角洲长期的监禁和盘问,他的详细经历会被要求“以某种形式”全部记录下来,但不会被任何人阅读,作为“SILENCE\#07”机密文档的一部分被绿色三角洲封禁起来。

此外,根据特工在结束时的个体腐化值不同,其结局也略有区别。

腐化值低于30:噩梦出现在他的梦境中,他偶尔会回到“剧院”之中,但它会随着稳定的精神治疗而逐渐潜藏起来,隐藏到特工的心理深处。直到很久之后,某个特工意志薄弱的时刻,不知从何而来的黄印又悄悄爬上了自己的手稿,特工就会知道自己的噩梦又回来了。

腐化值介于30和50之间:特工还可以度过一段勉强具有人形的生活。间歇的幻象、噩梦和难以停用的药品和精神治疗困扰着特工。可能几年后,特工会再次在生活中发现黄印。此后若干天的梦里,“剧院”中人物的脸孔出现在特工的面前。“剧院”的门继续朝特工敞开着,那也是他的归宿。

腐化值高于50,即便事件结束,特工的情况也不会有什么好转,他不再能够区分现实与幻象,无数的黑线在他的视野中游荡;尽管自己的手仍然在物理世界的空间中机械地运动着,他的思想已经回到了“剧院”之中。他忘记了自己曾经经历的生活,他在“剧院”之中获得了新生,并且仿佛他只在这里活过。他以为自己正在将这次经历记录下来,并作为机密档案被封存起来,但实际上他只是在背诵《黄衣之王》,成为下一个灰暗时刻的源头。

\subsection{持续到永恒的愉悦}

如果特工愿意留在彼岸,那么他会如愿乘上“剧院”前往黄衣之王的宫廷,怀揣着永不磨灭的对于黄衣之王的渴望的极乐情绪,直到一切可能性的尽头。

由于特工在现实中的存在被抹除,他在现实中的亲朋好友或许会表露短暂的疑惑情绪,但这并不会停留太长时间——与永恒愉悦的岁月相比,这些只是如同放在大海中的一粒原子,根本觅不见痕迹。但是谁知道呢,或许特工会借着另一条前往现实的通道,再一次出现在所有人的面前。

\subsection{绿色三角洲是否介入}

如果其他绿色三角洲介入了此次事件,那么在这次事件后漫长的时间中(直到今日),绿色三角洲都维持着对涉及这次事件的人群的暗访,确保不再有更多的风险。这同时意味着绿色三角洲往后更多的牺牲,不过这也是后话了。