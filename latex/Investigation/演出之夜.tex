如果特工在一切开始之前没能挫败吉姆一行人的计划,那么“悖论”的最终演出会在湖滨公园举办,而现场的情况会传输到奥克兰中央广场的大屏幕上。

\section{确定演出地点}

特工可以在演出开始前就试图确定演出的地点。特工有这几个着眼点:

首先,这是一个文字游戏:吉姆的房间中的家庭照片显示吉姆等人正位于一个水上公园(Waterpark),而绘本的全名为“瓦普内克剧院”(Teatro Warpneck)。这两个词组使用了完全相同的字母——从剧院的全名中删去 Waterpark,可以将剩余的字母拼凑出单词 Nocte,从而获得 Nocte Waterpark 这个答案。如果特工注意到了这两个词的关联,可以通过一个成功的\textbf{智力*5-20}的检定得到这个结论。

其次,在整个模组流程中,特工会抵达以下五个被黄衣之王腐蚀的地点:自己的家、亚当森的住所、加州鹿角传媒公司、闹钟酒吧和格林伍德家族宅邸。这五个地点以及水上公园,在地图上构成了黄印图腾的所有转折点。如果特工在抵达这些地方之后检查奥克兰地图,会在发现黄印在地图上浮现,其中那个特工没有拜访过的顶点上写着“湖滨公园”。

此外,特工的腐蚀值超过 25 后,可以在奥克兰市上空中看到一条若隐若现的光带,这条光带在以上五个被腐蚀的地点会发生一次折向。特工可以沿着这条光带抵达了湖滨公园。

最后,特工可以在演出开始之前跟踪吉姆一行人。由于水上公园目前并不是被黄衣之王腐蚀的地点,他们必须通过现实世界才能抵达那里——这对他们来说也并不容易。

特工没能注意到上面的任何线索也是正常的,他仍有很多弥补的机会:中央广场的大屏幕上的转播或者一些观众上传到 YouTube 的视频,都会显示出吉姆正在一个背景全都是水上娱乐设施的地方;天空中的那条带有转折光带会在中央广场的上方非常显眼。掌局者甚至可以安排特工坎蒂发过来一张奥克兰的地图,要求特工指出非自然发生的所在地,特工会在目击这张地图时看见扭曲的黄印。

\subsection{此刻的奥克兰中央广场}

中央广场的四个巨幕上突然切换到了一个疑似公园的场景,背景中的草坪上放置着巨型章鱼等雕塑:巨幕中的画面正如实地转播着“悖论”的演出现场,巨幕的音响也转播着现场的声音。

广场中的人群会驻足观看这个奇特的表演,并且很快开始呼应演出,在广场中央围绕成圈,以相仿的节奏踩跺着地面。大厦的灯光开始采用相同频率的闪烁。只有恐慌的宠物犬时不时的几声喊叫破坏这个节奏。

中央广场开始和“剧院”建立联系,使得这个场景中的环境腐化值陡升。随着演出的进行,幻象开始越发夸张:火车的轨道浮现于空间之中,空间中的物体边缘出现明显的颤动黑线,更远处的环境变得模糊,突然降下的大雨将淹没地面的水平面提升到了接近十五层楼的高度。

特工可以破坏这四块巨幕,或者闯入中控室,破坏中控室的电脑、解除皮特的控制等。管理人员不知道发生了什么,一边照着节奏敲击着电话按键,一边惊慌失措地拨打电话联络皮特·戴维斯。一个成功的\textbf{说服}或者\textbf{行政}检定就可以让慌乱而着迷的管理人员为自己开启这些中控室的门。

但是终止奥克兰中央广场的演出只能缓解燃眉之急,而真正的演出仍然在湖滨公园继续进行,并顺着互联网慢慢地蚕食现实——特工可以在 YouTube 上通过搜索关键词毫不费力地发现这个事实。

\subsection{此刻的格林伍德宅邸}

幻象阈值 40。

这栋建筑现在像是一个游动在空中的巨型鲸鱼,房屋的内部湿滑粘腻,某种细胞组织构成了墙壁,原本是窗户的地方构成了鲸鱼的若干个眼睛,门廊组成了鲸鱼张开的嘴。离开建筑会发现鲸鱼的尾巴上连接着众多线状的东西——这是在水中飘动的铁轨——数条铁轨在身后拖出黄印的形状,并最最终与地面相连。特工可以凭借想象飞到原本是门窗的位置处,从这里进入宅邸——它通往“剧院”。

\subsection{奥克兰市的其他角落}

在这个都市的其他角落,还有少量幸运观众在互联网上看到了这个奇异的演出,他们开始在自己的房间中制造类似节奏的响动,并且开关家中的顶灯。一种难以压抑的转发欲望使得这种狂热的节奏在互联网上迅速扩散起来。不过谢天谢地,由于当时的互联网并没有那么成熟,因此这一股势力短时间内很难造成恐怖的影响——但这花不了太久。

如果特工此时走在城市之中,可以听见四处传来类似节奏的鸣笛——这一部分来自于自己的幻象,一部分来自现实。

\subsection{CLOSE TO THE EDGE}

如果特工此前没有前往过海滨公园,那么特工必须通过现实世界抵达终点。一些宏大的、歇斯底里的幻象可能对特工的前进带来异常痛苦的阻碍,例如被拜亚基追踪、或者撞上自己已经看不见的来自现实世界的建筑物以阻挠这一段旅程。

我们在这里提供一点简单的从现实世界抵达目的地的示例:奥克兰与哈利湖开始融合,一场一直没能停息的雨冲刷着奥克兰的街道,并逐渐积累起来。此时特工眼中的世界完全沉浸在水中,水平面不断升高。摩天的黑色巨石撑破了破旧的街道和房屋拔地而起,一些死去的贝类和藻类尸体在街道的转角累积。特工可能看不见街道上行驶的车辆,只能隐隐约约地感受到一些无形的物体冲过带起的一阵湍急的漩涡;特工眼中的人类也像是玩偶一样在水中扭动,他们所发出的声音也像是透过液体传递到自己的耳朵之中,模糊而粘腻。

在路途中,可能看见两条若有若无的光带浮在上方,在海滨公园所在的位置交汇。

\subsection{演出开始}

随着这一天毕宿五从地面升起,湖滨公园的演出开始了。本次演出的结构和并不相同:吉姆只身一人位于舞池的中央,除了约瑟芬和皮特正在护卫着摄像机和电脑,其余所有特工在酒吧中见到的顾客都会出现在这里。

在舞蹈开始时,他们身边的观众就开始自发地开始以那种特殊的节奏开始运动、挥舞着自己的四肢,绕着圈开始运动。舞者也比上一次更加迅速地投入状态。这忘情的舞蹈很快卷入了路过的行人,瞬间变成一场放肆而疯狂的巨型仪式:土地从地面脱离,一只巨大的鲸鱼将所有参与的观众托举起来漂浮于空中,空间变得昏暗而粘滞。

铁轨从鲸鱼的腹部穿过,鲸鱼的背部的骨刺穿过皮肤而出形成了“剧院”的车厢,这列车厢继续上升者。特工会看见在这无底的哈利湖之中,一座直通天际的城堡出现在了视线之中,仿佛已经触手可及。而自己正沿着环形的铁轨围绕着这个城堡飞速地移动,特工在围绕着它的过程中可以看见它辉煌的大门,特工所能期望的一切从大门中流淌而出;城堡上昏黄的点缀在液体的折射中摇晃,昭示着窗户的位置和幸福的未来,而其中的一扇即将拥有自己的永恒的主人;无数托着精致食物的玩偶在维多利亚式的大厅轨道上四处滑动;沁人的花香和高贵的乐声填满了足以令万物失色的花园;在大厅的尽头,一个带着苍白面具的皇帝已经举起了自己的双手准备着特工的到来。

四周是无边无际的液体以及漂浮于液体中的黑色尖锐石块,头顶不知距离处泛着的浪花织成灰白相间的渔网。一只漂浮于极远处的巨大鲸鱼隐隐约约在远处晃动,隐约可见它的尾巴上垂下几根线(那是格林伍德宅邸)。除了特工所在的这条轨道,还能看见无数的轨道围绕着这栋庞大的城堡,旋转着、变换着、交错着。“那个宴会就在前方,永恒的欢愉与幸福。”

疯狂的喜乐攀上了所有观众的脸,但是仍旧维持着自己的舞蹈,而这一切也被直播的摄像机忠实地记录了下来。

\subsection{终止演出}

在舞蹈过程中吉姆身上的黑线开始诞生出人的形状,它会逐渐凝结成达夫。这个黑线凝结的程度象征着现实世界与“剧院”的联系,当达夫完全成型时,这里的人和物就永远被卡尔克萨所吞噬。

而希望结束“剧院”对现实世界的腐蚀,就是赶在这一切之前终止这场亵渎的直播——停止广场的屏幕,破坏录制这一演出的摄像机,捣坏将其上传到互联网上的电脑,并且除掉会想办法重启这场直播的人,最后阻止它继续在互联网上扩散。当然特工也可以在一切开始之前就完成上述所有条件,使这场演出永远不会出现在现实的世界。

对仪式参与者的攻击或者别的什么干扰,这都不会激起他们的反抗;除了当特工破坏了现场或者直播时,他们会想方设法地使一切恢复运作。他们只有一个目的:完成这场表演。他们全身心地投入其中无暇顾及其他。正在进行演出的吉姆不会被玩家干扰,而只是忘情地在舞台上进行自己的仪式;在一旁负责直播的皮特和约瑟芬则只会将注意力放在直播之上,他会竭尽全力地维系吉姆的布道。

在特工开始攻击电子设备或者仪式参与者的两回合后,有翼怪物闯入仪式的现场,对特工发起攻击。特工每回合都需要进行一次成功的意志检定来摆脱仪式对自身的控制。此外,在这个特殊的空间中被破坏的电器开始受到吉姆等人的想象影响,而只需要(1D6-想象人数)的回合就能自动修复(特工可以使用自己扭曲现实的能力与之对抗)。

现场吉姆死亡且电脑或摄像机中的一个被破坏,就会结束整个仪式。庞大而骇人的幻象开始从空间中褪去,参与意识的人都仿佛被抽取了庞大的精力瘫倒在原地,有翼生物就像从未出现过一样变成一滩黑线离去——但是特工自身腐化值带来的幻象并不会消失。

\subsection{联系绿色三角洲}

特工可以考虑主动联系自己的主管,并且在提供一定证据的情况下(流程可以参考\textbf{突发事件:特工联系了绿色三角洲}),由主管启用非自然事件侵染互联网的预案,给特工争取针对互联网或者电力系统的权限用以解决这次案件,如果有必要,甚至是一些非绿色三角洲的人手(如果时间允许的话,也可以从其他地方调来其他绿色三角洲小组)。申请预案权限和调用当地人手需要 2d6 * 10 分钟的时间,调集另一支绿色三角洲小组则需要 3d4 小时的时间。

\subsection{绿色三角洲的主动介入}

如果特工的行动不够迅速,整个城市已经开始被黄衣之王严重的影响,那么位于奥克兰的友方会积极地展开行动(奥克兰广场和直播网站上的线索都是友方们判断的线索)。他们会联络上自己的接头人,接头人会迅速启用非自然事件侵染互联网的预案(这需要 2d6 * 10 分钟的时间。):他会使用专线取得极高等级的权限,以输电装置损害的名义短暂停用整个奥克兰市的电路和网路,并在这暂停之中进行迅速的排查。

特工此时会接到来自特工坎蒂的联络,以获得更加详细的情报。如果此时特工至少已经精确地确定了问题的根源同时在于格林伍德宅邸和闹钟酒吧,充分的证据会让坎蒂被迫调用绿色三角洲以外自己的权柄(此时召集绿色三角洲已经太迟)来控制这个事态的扩散,例如调用警员和特工一同摧毁这两个地点。而如果特工此时提供了错误的情报,或者一无所知,则会让排查的时间更长。

但是这一切会给特工坎蒂所持有的明面地位带来巨大的压力,也会给潜伏的绿色三角洲扫尾带来许多难以言喻的麻烦,舆论会直指停电等骚动是为了掩藏政府的一些不能启口的阴谋等等。但相比让非自然事件在互联网上迅速扩散,这样的代价是值得的。总之,绿色三角洲会想办法处理这一切,尽管十分狼狈地付出高昂的代价。

\textbf{建议}:绿色三角洲的介入只是作为模组最后的一个应急手段,以将这次风波的范围限制在不至于破坏整个世界设定的区间中。如果特工仍然没有放弃,冒着难以忍受的折磨仍然在试图拯救一切,那么我们不妨将最终期限放得稍微宽一点,让特工独自一人解决这一切。即便绿色三角洲主动插手,也应该试图让特工在这次行动中位于“中心”的地位,作为对玩家的正面反馈。

\subsection{再来一次}

掌局者或许也可以考虑应用“剧院”的性质,在一切都无法挽回时,暗示特工“剧院”扭曲时空的性质,在特工的意愿下“回到”更早一些的时间,在事情开始之前将一切萌芽扼杀——但这个拯救的究竟是原来那个世界吗?

\subsection{尾声}

\subsubsection{劫后的奥克兰}

对于整个城市来说,这场灾难对于奥克兰市和其他区域的影响由直播持续的时间决定。

如果这场演出的直播顺利地持续超过 5 分钟,就会对整个世界带来难以磨灭的创伤,黄印的阴影从此开始埋在互联网的各个角落,并随着时间的流逝逐渐从其中扩散开来融入所有人的生活。

而随着直播的继续,奥克兰市的这一部分就会开始与卡尔克萨开始融合。在最坏的情况下,在两个地标从地球上所有没有被黄衣之王腐蚀的人的认知中被抹除,连带着这批受到影响的平民。绿色三角洲其他部队会在此时介入将灾难控制在这个范围之内。

\subsubsection{绿色三角洲是否介入}

如果其他绿色三角洲的小队介入了此次事件,那么绿色三角洲的管理人员会直到这个风险的存在。在这次直播事件后漫长的时间中(直到今日),绿色三角洲都维持着对可能被这一次直播事件波及的人群的暗访,试图确保不再有更多的风险。这同时意味着绿色三角洲更多的牺牲,不过这也是后话了。


\subsubsection{幸存特工会面临什么}

如果特工侥幸活到了最后,那么面临他的是来自绿色三角洲的长期监禁和盘问,他的详细经历会被要求“以某种形式”全部记录下来,但不会被任何人阅读,作为“SILENCE\#07”机密文档的一部分被绿色三角洲封禁起来。

此外,根据特工在结束时的个体腐化值不同,其结局也略有区别。

如果特工腐化值低于15,那么噩梦会出现在他的睡眠中,但它会随着稳定的精神治疗而逐渐潜藏起来,隐藏到特工的心理深处。直到很久之后的某个特工意志薄弱的时刻时刻,不知从何而来的黄印又悄悄爬上自己的手背时,特工就会知道自己的噩梦又回来了。

如果特工的腐化值介于15和30之间,那么他还可以度过一段勉强具有人形的生活。如影随形的幻象、噩梦和难以停用的药品和精神治疗困扰着特工。但不管如何,特工要不了多久就会发现自己的电子设备中出现了黄衣之王的印记,并且以自己作为发信人发给与直接存在联系的一切人物。若干天的梦里,达夫会出现自己的梦境中,带着所有“剧院”中人物的脸出现在特工的面前。“剧院”的门继续朝着特工敞开着,那里也是他的归宿。特工周围的人也伴随着巨大的被污染的可能性,但这一切就交给黄衣之王决定了。

如果特工的腐化值高于30,那么他甚至可能等不到自己的经历被完全记录下来:在记录到一半的时刻,他就不再能够区分现实与幻象,无数的黑线在他的视野中游荡;尽管自己的手仍然在物理世界的空间中机械地运动着,他的思想已经回到了“剧院”之中。他忘记了自己曾经经历的生活,忘了自己为城市做出的牺牲和贡献,他在“剧院”之中获得了新生,并且仿佛他只在这里活过。这本他自己以为是经历记录被封存于机密档案中的文件,实际上只是自己背诵的《黄衣之王》。

\subsubsection{持续到永恒的愉悦}

如果特工死亡在了被黄衣之王腐蚀的地界,或者愿意留在“剧院”之中,那么他会乘坐着这列前往黄衣之王宴会的列车,怀揣着永不磨灭的对于黄衣之王的渴望的极乐情绪,直到一切可能性的尽头。

而他的亲朋好友,或许会表露出短暂的疑惑情绪,但这并不会停留太长时间——与永恒愉悦的岁月相比,这些只是如同放在大海中的一粒原子,根本觅不见痕迹。但是谁知道呢,或许特工会借着另一条前往现实的通道,再一次出现在所有人的面前。
