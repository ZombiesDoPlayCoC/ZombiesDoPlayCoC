如果特工在一切开始之前没能挫败吉姆一行人的计划,那么“悖论”的最终演出会在湖滨公园举办,而现场的情况会传输到奥克兰中央广场的大屏幕上。而另一方面,特工严重的失真症正在将他不断拽入另一个神秘的世界,特工对于现实的感触在逐渐被彼岸世界展现的现实所替代。

如果特工想要完成绿色三角洲的职责,他在终幕中需要依次完成这几个事件:破坏奥克兰中央广场的转播、从彼岸世界逃回现实世界、在湖滨公园的现场击杀高腐化值的人物。如果特工可以在短时间内完成第三步,那么忽略第一步不会带来太大的问题。

\section{此刻的奥克兰中央广场}

中央广场的四个巨幕上突然切换到了一个疑似公园的场景,背景中的草坪上放置着巨型章鱼等雕塑:巨幕中的画面正如实地转播着“悖论”的演出现场,巨幕的音响也转播着现场的声音。

广场中的人群会驻足观看这个奇特的表演,并且很快开始呼应演出,在广场中央围绕成圈,以相仿的节奏踩跺着地面。大厦的灯光开始采用相同频率的闪烁。只有恐慌的宠物犬时不时的几声喊叫破坏这个节奏。

中央广场开始和“剧院”建立联系,使得这个场景中的环境腐化值陡升。随着演出的进行,幻象开始越发夸张:火车的轨道浮现于空间之中,空间中的物体边缘出现明显的颤动黑线,更远处的环境变得模糊,突然降下的大雨将淹没地面的水平面提升到了接近十五层楼的高度。

\subsection{停止转播}

特工可以破坏这四块巨幕,或者闯入中控室,破坏中控室的电脑、解除皮特的控制等。管理人员不知道发生了什么,一边照着节奏敲击着电话按键,一边惊慌失措地拨打电话联络皮特·戴维斯。一个成功的\textbf{说服}或者\textbf{行政}检定就可以让慌乱而着迷的管理人员为自己开启这些中控室的门。

但是终止奥克兰中央广场的演出只能缓解燃眉之急,而真正的演出仍然在湖滨公园继续进行,并顺着互联网慢慢地蚕食现实——特工可以在 YouTube 上通过搜索关键词毫不费力地发现这个事实。

\section{确定演出地点}

特工可以在演出开始前就试图确定演出的地点。特工有这几个着眼点:

首先,这是一个文字游戏:吉姆的房间中的家庭照片显示吉姆等人正位于一个水上公园(Waterpark),而绘本的全名为“瓦普内克剧院”(Teatro Warpneck)。这两个词组使用了完全相同的字母——从剧院的全名中删去 Waterpark,可以将剩余的字母拼凑出单词 Nocte,从而获得 Nocte Waterpark 这个答案。如果特工注意到了这两个词的关联,可以通过一个成功的\textbf{智力*5-20}的检定得到这个结论。

其次,在整个模组流程中,特工会抵达以下五个被黄衣之王腐蚀的地点:自己的家、亚当森的住所、加州鹿角传媒公司、闹钟酒吧和格林伍德家族宅邸。这五个地点以及水上公园,在地图上构成了黄印图腾的所有转折点。如果特工在抵达这些地方之后检查奥克兰地图,会在发现黄印在地图上浮现,其中那个特工没有拜访过的顶点上写着“湖滨公园”。

此外,特工的腐蚀值超过 25 后,可以在奥克兰市上空中看到一条若隐若现的光带,这条光带在以上五个被腐蚀的地点会发生一次折向。特工可以沿着这条光带抵达湖滨公园。

最后,特工可以在演出开始之前跟踪吉姆一行人。由于水上公园目前并不是被黄衣之王腐蚀的地点,他们必须通过现实世界才能抵达那里——这对他们来说也并不容易。

特工没能注意到上面的任何线索也是正常的,他仍有很多弥补的机会:中央广场的大屏幕上的转播或者一些观众上传到 YouTube 的视频,都会显示出吉姆正在一个背景全都是水上娱乐设施的地方;天空中的那条带有转折光带会在中央广场的上方非常显眼。掌局者甚至可以安排特工坎蒂发过来一张奥克兰的地图,要求特工指出非自然发生的所在地,特工会在目击这张地图时看见扭曲的黄印。

\subsection{奥克兰市的其他角落}

在这个都市的其他角落,还有少量幸运观众在互联网上看到了这个奇异的演出,他们开始在自己的房间中制造类似节奏的响动,并且开关家中的顶灯。一种难以压抑的转发欲望使得这种狂热的节奏在互联网上迅速扩散起来。不过谢天谢地,由于当时的互联网并没有那么成熟,因此这一股势力短时间内很难造成恐怖的影响——但这花不了太久。

如果特工此时走在城市之中,可以听见四处传来类似节奏的鸣笛——这一部分来自于自己的幻象,一部分来自现实。

\subsection{CLOSE TO THE EDGE}

通往湖滨公园的路程是抵达尾声的最后一段旅途,但同时也是对特工理智的巨大挑战:漫天的幻象掩盖着现实,摩天的黑色巨石撑破了破旧的街道和房屋拔地而起,一些死去的贝类和藻类尸体在街道的转角累积。特工可能看不见街道上行驶的车辆,只能隐隐约约地感受到一些无形的物体冲过带起的一阵湍急的漩涡;特工眼中的人类像是玩偶一样在水中扭动,他们所发出的声音也像是透过液体传递到自己的耳朵之中,模糊而粘腻。

这些幻象会对特工的前进带来异常痛苦的阻碍,例如撞上自己已经看不见的现实世界的建筑或者车辆。但这种现状不会维持太长的时间——当特工注意到从四面传来的那个熟悉的声音时,或者被擦身而过的车辆撞至昏厥时,那个熟悉的敲门声就又一次响起了。

\section{彼岸}

这是\textbf{尤里卡时刻}所揭露的世界,位于“剧院”之外的空间。但和“剧院”不同,它更加靠向卡尔克萨一侧,因此难以与“现实”直接产生联系。它是吉姆和达夫创造的理想世界的雏形,并在混沌的时光中收留了众多旅者。

腐化值过高的失真症患者可能无意识间就发现自己来到了这个世界,并且从此在其中漫游。

从地理条件上来看,彼岸和奥克兰极为相似,但是完全浸没在水下。这个世界保留了“剧院”那种维多利亚式风格,有一些近未来和超现实的成分:带有维多利亚式装潢的铁制悬浮电车在街道之间游走,复杂的浮雕装饰镶嵌在摩天大楼的棱角处,众多铜质喇叭插在漂浮在天空中的电灯上。此外,这个世界没有白天——双月永恒悬挂在水面之上。

特工在腐化值超过40时可能会不知不觉地抵达这里。同样的,我们提供一些事件以供参考。

\subsection{建议:运行彼岸}

特工进入这个世界时,自己属于“现实”的记忆会开始模糊,并且大量属于这个世界的信息会涌入。特工在这个世界中引导着一个和“现实”区别不大的生活,不同的是,其中的不幸和矛盾尽数移除了,自己也不再是绿色三角洲的一员。

彼岸世界应该极力地渲染“正常”的氛围,仿佛一切最开始都是如此,也最终如此,所有人都认为它如此。

我们建议让特工踏上前往湖滨公园的道路时就被拉入这个空间(例如通过\textbf{事件:再会亚当森}),并且让他短暂地在这个空间中进行探索,让他对这里有初步的印象。

在特工返回现实之后,你可以将特工在彼岸中的经历与“现实”中的经历缠绕在一起,像是正在观看不断来回切换底片的两部电影(例如\textbf{事件:枪击})。因此你不必将所有的内容都在特工第一次造访彼岸时揭晓,而是保留一些内容放在之后的意识流中,当特工在“现实”中涉及到某个关键词时,再将他拉入彼岸世界。

\textbf{注意}:这种叙事并不意味着特工反复在彼岸与现实中穿梭,而是如同\textbf{叙诡:二月十二日之夜}中那样,是特工时空认知紊乱的一种呈现——只不过现在更加严重了。因此,在组合两个世界时不需要遵从事件发生的先后顺序。但是为了保证玩家的主动性,我们建议只颠倒彼岸中事件的先后顺序(它们的先后顺序是\textbf{既定}的),而让现实发生之事按照客观的时间呈现。最后,这种切换应该仅仅作为渲染终末氛围的一种调料,而不是故意切碎叙事的休止符。

\subsection{离开彼岸}

彼岸之人需要先从这里进入“剧院”,才能再一次回到“现实”。特工只有两种方法离开彼岸:在彼岸世界找到“剧院”的站台,进入正停靠那里的“剧院”,并按照“剧院”的规则回到现实;或者特工可以在这个世界找到那台造梦机或者类似的东西,借助造梦机回到现实(\textbf{参见剧院事件:造梦机})。

\subsection{再会亚当森}

这个事件可以作为特工进入彼岸的引子。它像本模组开头那样,使用敲门声打断特工本来的思绪,像是打断了梦境一般,将其引入当前的世界。同样是夜晚(这里没有白天),但这一次的来访者是亚当森·考克斯。他手上拿着一个小礼盒和一张卡片,穿着他标志性的格子衫,大大落落地站在门口。亚当森会提到自己收到了前往舞会的邀请函,并且也打算前往那里。

他把手上的东西交给了特工,表示自己愿意在舞会的时候也见到他。这张卡片和特工此前在亚当森卧室书桌上找到的完全一样,礼盒中装着的则是《瓦普内克剧院》绘本。如果特工问起他这两个东西是怎么来的,亚当森会回答这是吉姆·格林伍德寄给自己的,并且吉姆也委托自己把这两个东西都交给特工。亚当森会和特工拥抱之后就和特工告别。

这里的亚当森不是特工在绿色三角洲的同事,而只是“朋友”关系。在告别的拥抱中,特工可以回忆起两人度过的无数愉悦而舒适的时光。这段友谊可以持续到天涯的尽头。

如果特工问亚当森他为什么要去那里,他会耸耸肩,“这可是那位王的舞会。”如果特工提出要和亚当森一同前往,亚当森会感谢特工的美意然后以想和家人一同前往作为托辞拒绝。如果特工跟踪亚当森,他会在亚当森登上一列电车之后失去他的踪迹。特工可以问起这位吉姆是谁,“这里的大祭司”。

\subsection{琐事}

特工确实在这里“生活”,他会在这里找到若干自己生活和存在的痕迹:他的电子邮箱里可能塞满了同事生日派对和婚礼的邀请,附近的街坊邻居也会热情地和特工主动攀谈……在特工在彼岸的街道上行走时,可能会被某个路人拦下,他亲切地和特工打招呼:先是用手拍拍肩膀,再拥抱一下,“听说你拿到了邀请函!我的天!祝贺你,祝贺你。祝福我们的王。”特工获得邀请函的消息仿佛已经被传开了。

\subsection{格林伍德宅邸}

这栋建筑现在是一个巨大的鲸鱼堡垒,房屋的内部湿滑粘腻,某种细胞组织构成了墙壁,原本是窗户的地方构成了鲸鱼的若干个眼睛,门廊组成了鲸鱼张开的嘴。一条铁轨穿过鲸鱼的尾部,连接着天际。当特工进入格林伍德宅邸时,会听见从外部传来的火车汽笛的声音,然后特工会发现“剧院”就停在铁轨之上。

\subsection{站台}

如果特工在彼岸中抵达了现实中闹钟酒吧的位置,他会发现这个酒吧现在是一个富丽堂皇的车站。一条连接着天际线的铁轨穿过这里,宽阔而明亮的大门欢迎着一切来者。形形色色的人潮在这里进进出出,售卖报纸和饮料的机械玩偶在门前来回游走。此时可以从两端看到一列列车正停在站台里。站台找不到时刻表。如果特工提出想要乘坐“剧院”,侍者会请求出示邀请函,在确认无误后会引领着特工进入“剧院”。

\subsection{但丁,记忆收集者}

这是一个瘦得有点皮包骨的老人,穿着一套袖口有些磨损的老旧西装。他在站台大门旁放了一个小木桌和凳子,此时他正用两手杵着拐杖坐在凳子上。桌子上摆着一台奇特的蒸汽装置(参见\textbf{剧院事件:造梦机})。

他自称但丁·阿利吉耶里,一名来自罗马的失恋人,造梦机的发明者。他本拿着王的侍者俾德丽采交给自己的邀请函,但他不小心把它遗失了。于是俾德丽采交给了他一张清单,上面写着众多人物的名字,但丁如果能将造梦机兜售给这些人物,他就能取回邀请函,前往天堂与俾德丽采重逢。

他会首先进行自我介绍,然后就询问特工的名字,\textbf{人源情报}不低于40\%的特工会从他的脸上抓到一丝转瞬即逝的快乐情绪。他会相当模糊地为特工介绍造梦机的功能,并且表示愿意提供一些胶囊让特工体验造梦机,前提是获取特工的一些血液。

但丁会以特工的遗传物质和一段回忆作为交换造梦机的筹码。如果特工同意了,他就会从腰包里拿出一个脏兮兮的古典样式的针管,从特工的手臂抽取大概200毫升的血液——这个针眼很快就自动痊愈了。特工在被抽血时会感到一种难以言喻的不安感,如同一只手正攀上了自己的心脏。(特工可能在这个过程中将意识转换回“现实”。)

交易完成后,但丁就会将造梦机交给特工,然后拿出清单在上面划上一个勾,收起自己的摊位前往下一个站点。特工会在清单上看见自己的名字,以及一些别的名字,比如特工的重要之人。

\subsection{铜管声}

每隔一段“时间”,特工可以听到从某处传来几声若有若无的铜管乐器的声音,紧随其后就能听见从所有电灯装置着的喇叭上播放出的音乐声,周围的行人都驻足并高喊:“祝福我们的王!”特工很快就意识到这个声音也在从自己的嘴中喊出。循着最初的声音的方向,特工可以抵达彼岸的边缘。

\subsection{彼岸的边缘}

彼岸的边缘像是一个巨大海底沟壑的岸边,其旁边的海沟深不见底,黑暗得如同虚空。更远处有一层像遮罩一样的隔层若隐若现,像若干面镜子,可以看见在对面有着另一个彼岸的痕迹。成功的 \textbf{警觉} 检定可以从隔层的对面听见一个女童重复的歌声。从城市中出发的铁轨一直向外延申,透过边缘的遮罩进入了看不见的终点。

边缘的地面上埋置着众多的铜管乐器,它们的喇叭口从土地中露出朝向地面,构成了彼岸的边界线。肃穆、低沉的声音这些喇叭中奏响。如果仔细聆听其中的音乐,\textbf{艺术(音乐)}不低于 20\%的特工可以发现这就是和 St. Illness 中背景音乐完全相同的旋律,但是去除了其中的装饰音,并且节奏相较于那个背景音乐慢了很多,呈现出一种庄严感。

岸边聚集着几十名穿着灰袍戴着面具的人,他们在这里围绕成圈,在埋置的铜管乐器中穿行,并在口中念着某种类似颂歌的东西。他们心无旁骛地在这里歌唱着,完全没有注意到特工的到来。如果特工一一揭开他们的面具,会在其下发现一些陌生人和一些熟悉的面孔。“悖论”三人组都在这里,其中甚至还有一些特工已经逝去的亲人。此处在“现实”中对应着海滨公园。

\subsection{叙诡:枪击}

这是将彼岸与“现实”经历纠缠的一个例子,用以描述在”现实“中,特工朝着吉姆·格林伍德开最后一枪之时。

此时,我们可以突然将叙述视角转回彼岸的边缘。“你感受到一种异常真实的剧痛在全身游走,你中弹了。你的眼前的景象不再是那个海边的小公园,而是紧邻着无边虚空的悬崖峭壁之上,埋在地面的铜管发出低沉的哀鸣声。你发现你的手捂着的腹部,从那里涌出黑色的油墨在水体中逸散,沾染了自己穿着的灰色长袍。这剧痛之后是什么东西倒下的声音——你躺在了地面上,看见了站着的那位持枪之人——是【特工】。你觉得自己的嘴在嗫嚅着什么,但是发不出声音。”

然后将视线再转回特工在现实的视线,“‘我好怕’,吉姆的声音在你的脑海中响起。这个声音和突然出现的画面只停留了一眨眼的时间,公园的场景重新回到了你的视线,你看见那个舞者在你面前倒下了,伴随着旁边手足无措的人员。你看见吉姆的身上升腾起一阵黑线,组成了一个扭曲的印记,然后就弥散了。”

\section{落幕}

特工从彼岸归来之后,就能再一次踏上前往终点的旅途。破碎而疯狂的幻象(其中一部分来自彼岸)会继续干扰特工在现实中的旅途,和彼岸平静祥和的氛围形成鲜明的反差。当特工疯狂发作之时,需要进行一次\textbf{腐化值}检定,如果检定成功,特工的意识就会再一次进入彼岸。

\subsection{演出开始}

演出会在毕宿五从地面升起之时开始。本次演出的结构和并不相同:吉姆只身一人位于舞池的中央,除了约瑟芬和皮特正护卫着摄像机和电脑,其余所有特工在酒吧中见到的顾客都会出现在这里。

在舞蹈开始时,他们身边的观众就开始自发开始那种特殊的节奏,绕着圈运动。很快更多的旁观者参与进来,这里瞬间变成一场庄严的巨型仪式,一曲传递着憧憬和希望的颂歌:土地从地面脱离,一只巨大的鲸鱼将所有参与的观众托举起来漂浮于空中,空间变得昏暗而粘滞。

铁轨从鲸鱼的腹部穿过,鲸鱼的背部的骨刺穿过皮肤而出形成了“剧院”的车厢,这列车厢继续上升者。特工会看见在这无底的哈利湖之中,一座直通天际的城堡出现在了视线之中,仿佛已经触手可及。而自己正沿着环形的铁轨围绕着这个城堡飞速地移动,特工在围绕着它的过程中可以看见它辉煌的大门,特工所能期望的一切从大门中流淌而出;城堡上昏黄的点缀在液体的折射中摇晃,昭示着窗户的位置和幸福的未来,而其中的一扇即将拥有自己的永恒的主人;无数托着精致食物的玩偶在维多利亚式的大厅轨道上四处滑动;沁人的花香和高贵的乐声填满了足以令万物失色的花园;在大厅的尽头,一个带着苍白面具的皇帝已经举起了自己的双手准备着特工的到来。

四周是无边无际的液体以及漂浮于液体中的黑色尖锐石块,头顶不知距离处泛着的浪花织成灰白相间的渔网。一只漂浮于极远处的巨大鲸鱼隐隐约约在远处晃动,隐约可见它的尾巴上垂下几根线(那是格林伍德宅邸)。除了特工所在的这条轨道,还能看见无数的轨道围绕着这栋庞大的城堡,旋转着、变换着、交错着。“那个宴会就在前方,永恒的欢愉与幸福。”

由衷的喜乐攀上了所有观众的脸,但是仍旧维持着自己的舞蹈,而这一切也被直播的摄像机忠实地记录了下来。

\subsection{终止演出}

在舞蹈过程中吉姆身上的黑线开始诞生出人的形状,它会逐渐凝结成达夫。这个黑线凝结的程度象征着现实世界与“剧院”的联系,当达夫完全成型时,这里的人和物就永远被卡尔克萨所吞噬。

而希望结束“剧院”对现实世界的腐蚀,就是赶在这一切之前终止这场亵渎的直播——停止广场的屏幕,破坏录制这一演出的摄像机,捣坏将其上传到互联网上的电脑,并且除掉会想办法重启这场直播的人,最后阻止它继续在互联网上扩散。当然特工也可以在一切开始之前就完成上述所有条件,使这场演出永远不会出现在现实的世界。

对仪式参与者的攻击或者别的什么干扰,这都不会激起他们的反抗;除了当特工破坏了现场或者直播时,他们会想方设法地使一切恢复运作。他们只有一个目的:完成这场表演。他们全身心地投入其中无暇顾及其他。正在进行演出的吉姆不会被玩家干扰,而只是忘情地在舞台上进行自己的仪式;在一旁负责直播的皮特和约瑟芬则只会将注意力放在直播之上,他会竭尽全力地维系吉姆的布道。

在特工开始攻击电子设备或者仪式参与者的两回合后,有翼怪物闯入仪式的现场,对特工发起攻击。特工每回合都需要进行一次成功的意志检定来摆脱仪式对自身的控制。此外,在这个特殊的空间中被破坏的电器开始受到吉姆等人的想象影响,而只需要(1D6-想象人数)的回合就能自动修复(特工可以使用自己扭曲现实的能力与之对抗)。

现场吉姆死亡且电脑或摄像机中的一个被破坏,就会结束整个仪式。庞大而骇人的幻象开始从空间中褪去,参与意识的人都仿佛被抽取了庞大的精力瘫倒在原地,有翼生物就像从未出现过一样变成一滩黑线离去——但是特工自身腐化值带来的幻象并不会消失。

\subsection{联系绿色三角洲}

特工可以考虑主动联系自己的主管,并且在提供一定证据的情况下(流程可以参考\textbf{突发事件:特工联系了绿色三角洲}),由主管启用非自然事件侵染互联网的预案,给特工争取针对互联网或者电力系统的权限用以解决这次案件,如果有必要,甚至是一些非绿色三角洲的人手(如果时间允许的话,也可以从其他地方调来其他绿色三角洲小组)。申请预案权限和调用当地人手需要 2d6 * 10 分钟的时间,调集另一支绿色三角洲小组则需要 3d4 小时的时间。

\subsection{绿色三角洲的主动介入}

如果特工的行动不够迅速,整个城市已经开始被黄衣之王严重的影响,那么位于奥克兰的友方会积极地展开行动(奥克兰广场和直播网站上的线索都是友方们判断的线索)。他们会联络上自己的接头人,接头人会迅速启用非自然事件侵染互联网的预案(这需要 2d6 * 10 分钟的时间。):他会使用专线取得极高等级的权限,以输电装置损害的名义短暂停用整个奥克兰市的电路和网路,并在这暂停之中进行迅速的排查。

特工此时会接到来自特工坎蒂的联络,以获得更加详细的情报。如果此时特工至少已经精确地确定了问题的根源同时在于格林伍德宅邸和闹钟酒吧,充分的证据会让坎蒂被迫调用绿色三角洲以外自己的权柄(此时召集绿色三角洲已经太迟)来控制这个事态的扩散,例如调用警员和特工一同摧毁这两个地点。而如果特工此时提供了错误的情报,或者一无所知,则会让排查的时间更长。

但是这一切会给特工坎蒂所持有的明面地位带来巨大的压力,也会给潜伏的绿色三角洲扫尾带来许多难以言喻的麻烦,舆论会直指停电等骚动是为了掩藏政府的一些不能启口的阴谋等等。但相比让非自然事件在互联网上迅速扩散,这样的代价是值得的。总之,绿色三角洲会想办法处理这一切,尽管十分狼狈地付出高昂的代价。

\textbf{建议}:绿色三角洲的介入只是作为模组最后的一个应急手段,以将这次风波的范围限制在不至于破坏整个世界设定的区间中。如果特工仍然没有放弃,冒着难以忍受的折磨仍然在试图拯救一切,那么我们不妨将最终期限放得稍微宽一点,让特工独自一人解决这一切。即便绿色三角洲主动插手,也应该试图让特工在这次行动中位于“中心”的地位,作为对玩家的正面反馈。

\subsection{再来一次}

掌局者或许也可以考虑应用“剧院”的性质,在一切都无法挽回时,暗示特工“剧院”扭曲时空的性质,在特工的意愿下“回到”更早一些的时间,在事情开始之前将一切萌芽扼杀——但这个拯救的究竟是原来那个世界吗?

\subsection{尾声}

\subsubsection{劫后的奥克兰}

对于整个城市来说,这场灾难对于奥克兰市和其他区域的影响由直播持续的时间决定。

如果这场演出的直播顺利地持续超过 5 分钟,就会对整个世界带来难以磨灭的创伤,黄印的阴影从此开始埋在互联网的各个角落,并随着时间的流逝逐渐从其中扩散开来融入所有人的生活。

而随着直播的继续,奥克兰市的这一部分就会开始与卡尔克萨开始融合。在最坏的情况下,在两个地标从地球上所有没有被黄衣之王腐蚀的人的认知中被抹除,连带着这批受到影响的平民。绿色三角洲其他部队会在此时介入将灾难控制在这个范围之内。

\subsubsection{绿色三角洲是否介入}

如果其他绿色三角洲的小队介入了此次事件,那么绿色三角洲的管理人员会直到这个风险的存在。在这次直播事件后漫长的时间中(直到今日),绿色三角洲都维持着对可能被这一次直播事件波及的人群的暗访,试图确保不再有更多的风险。这同时意味着绿色三角洲更多的牺牲,不过这也是后话了。


\subsubsection{幸存特工会面临什么}

如果特工侥幸活到了最后,那么面临他的是来自绿色三角洲的长期监禁和盘问,他的详细经历会被要求“以某种形式”全部记录下来,但不会被任何人阅读,作为“SILENCE\#07”机密文档的一部分被绿色三角洲封禁起来。

此外,根据特工在结束时的个体腐化值不同,其结局也略有区别。

如果特工腐化值低于15,那么噩梦会出现在他的睡眠中,但它会随着稳定的精神治疗而逐渐潜藏起来,隐藏到特工的心理深处。直到很久之后的某个特工意志薄弱的时刻时刻,不知从何而来的黄印又悄悄爬上自己的手背时,特工就会知道自己的噩梦又回来了。

如果特工的腐化值介于15和30之间,那么他还可以度过一段勉强具有人形的生活。如影随形的幻象、噩梦和难以停用的药品和精神治疗困扰着特工。但不管如何,特工要不了多久就会发现自己的电子设备中出现了黄衣之王的印记,并且以自己作为发信人发给与直接存在联系的一切人物。若干天的梦里,达夫会出现自己的梦境中,带着所有“剧院”中人物的脸出现在特工的面前。“剧院”的门继续朝着特工敞开着,那里也是他的归宿。特工周围的人也伴随着巨大的被污染的可能性,但这一切就交给黄衣之王决定了。

如果特工的腐化值高于30,那么他甚至可能等不到自己的经历被完全记录下来:在记录到一半的时刻,他就不再能够区分现实与幻象,无数的黑线在他的视野中游荡;尽管自己的手仍然在物理世界的空间中机械地运动着,他的思想已经回到了“剧院”之中。他忘记了自己曾经经历的生活,忘了自己为城市做出的牺牲和贡献,他在“剧院”之中获得了新生,并且仿佛他只在这里活过。这本他自己以为是经历记录被封存于机密档案中的文件,实际上只是自己背诵的《黄衣之王》。

\subsubsection{持续到永恒的愉悦}

如果特工死亡在了被黄衣之王腐蚀的地界,或者愿意留在“剧院”之中,那么他会乘坐着这列前往黄衣之王宴会的列车,怀揣着永不磨灭的对于黄衣之王的渴望的极乐情绪,直到一切可能性的尽头。

而他的亲朋好友,或许会表露出短暂的疑惑情绪,但这并不会停留太长时间——与永恒愉悦的岁月相比,这些只是如同放在大海中的一粒原子,根本觅不见痕迹。但是谁知道呢,或许特工会借着另一条前往现实的通道,再一次出现在所有人的面前。
