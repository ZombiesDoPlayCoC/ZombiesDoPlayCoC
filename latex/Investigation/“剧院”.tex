剧院是本模组中连接卡尔克萨与“现实”的通道。具体来说,它是永远穿梭在哈利湖底的一列蒸汽火车。这列列车没有头部和尾部,它的车厢以随意且并不固定的模式连接在一起。但毫无疑问的是车上所有的乘客都围绕着卡尔克萨的王宫徒劳地运转着。

列车的内部装潢和维多利亚时期的豪华火车类似:靠窗的一侧是供人通行的走廊,另一侧是带有小门的诸多隔间。华丽而繁复是火车内部空间的主要基调,车厢的小门上是一些关于海底动植物的工笔绘画(如果特工读过《卡尔克萨王族食谱》,那么他可以认出这些生物),厚实柔软的红白图案的地毯铺满了整个走廊。

朝窗外看去,会看见注意到这是一列行进在水底的“蒸汽火车”:列车的前方不知从何处喷出的类似蒸汽一样的物质在水中漂动,带出一条模糊的黑色轨迹;晃动的车轴互相摩擦发出规律的“哐当”声,在迟钝的液体中回荡。

随着模组的推进,这个列车脚下的轨道变得越发不规则,并最终脱离地面,在液体中自由地飘动;但列车的内部始终保持着稳定的景象,仿佛完全不被外界影响。

对于被腐蚀者来说,剧院的构造是稳固的,他们不再被支离破碎的现象折磨,不再听见莫名其妙的耳语,不再需要疑神疑鬼,一切事物在这个区域中都是如此自洽而宁静。特工在这个空间中不会感到饥饿与疲惫,然而在应该出现食物的地方总是会堆满食物。

这个空间作为卡尔克萨的延申,具有一定的变化法则:它们禁锢着乘客,又受到乘客思想的影响。被腐蚀者具有影响剧院的能力。他们可以在这片区域中“心想事成”,但总是以意想不到的形式呈现。但对剧院的改造意味着在这个泥潭里更深一步(适当地增加腐化值)。

\section{建议:运行剧院}
进入剧院是本模组的重要节点,应用这个场景控制模组的节奏是一个不小的挑战,我们在这里给出几个简单的建议。

我们建议将特工进入剧院的次数控制在2-4次,且控制每次在其中停留的时长——过量信息只会让玩家疲倦。

你可以将剧院的遭遇与“现实”的探索糅合:剧院的遭遇可能会将特工从剧院扔回“现实”,也以幻象的形式出现在“现实”中,并将特工带入剧院(遭遇的示例参见\textbf{剧院事件})。

在闹钟酒吧和格林伍德宅邸,你有诸多机会将特工不知不觉地地引入或者带离剧院。比如让特工第一次进入剧院时,只是草草瞥见了那个梦幻的走廊,就被迅速带离了这里(特工或许会燃起主动拜访这里的欲望)。

并且,进入剧院都\textbf{需要}获得有效信息,比如一些值得深挖的关键词,或者揭示出急需特工解决的来自俗世的危机(例如提示特工自己正在被绿色三角洲监视),让他们在离开剧院之后仍有事情需要完成——否则他们只会觉得剧院是某种可有可无的调味料,或者傻乎乎地等着第二天再一次盲目地冲入剧院。

最后,进入剧院必须带来可察觉的代价,例如和其他人沟通越发困难,幻象出现的频率大幅上升。对特工来说,剧院应该是一个未知且有巨大风险的空间,而非可以随意进去闲逛的市民图书馆。

\section{离开剧院}
特工可以被动或者主动地离开这里。

特工可以在遭遇剧院事件后被动地离开剧院。例如特工在剧院中遭遇\textbf{剧院事件:吉姆的幼年}之后,就可能会发现自己进入了“现实”中的吉姆卧室。

特工也可以主动离开这里。当特工希望离开时,他会发现自己打开的下一个包厢中映射出扭曲的现实光景。穿过这扇门,特工就会回到“现实”,或者至少是特工相信的那个“现实”(参见\textbf{剧院事件:虚假的离场})。

主动离开这里的特工会抵达自己心中希望到达的坐标。在没有明确指向的情况下,特工会回到自己的家或者亚当森的家,或者是回到进入剧院的位置。

但无论如何,特工只能通过剧院回到被黄衣之王腐蚀的“现实”地点,也无法进入自己从未造访过的“现实”地点。这个限制对“悖论”同样成立。

\textbf{可选:时间的把戏。}剧院是若干可能性叠加态的碎片,它按照不固定的顺序拼接着若干时空,因此进出剧院对应的“现实”时间并不一致。这是一个使用时间叙诡的好机会,让特工体验一把时间倒流或者时间跳跃。如果特工带有强烈的回到过去或者造访未来的想法,那么剧院出口对应的时间会略微地朝着特工希望的方向偏移。

\textbf{可选:老套的平行时空。}特工回到的“现实”只是卡尔克萨蕴含的万千可能性中的一种,而\textbf{不是}原来那个。它和原本的世界很像,但存在着些微的差异。这个新的世界与特工的认知基本一致,但存在一些和原本世界微妙的出入;如果特工有一个良好的记忆,他们会逐渐发现一些和自己的记忆有所出入的细节。

例如离开剧院的特工路过熟悉的花店门前,花店的老板微笑着问到,“你这件衣服真好看,新买的吗?”但事实上在特工的印象中,这位老板在这天的上午就说过这句话了。如果特工向老板提出这个疑问,只会迎来,“啊,这样的吗?可能我忘记了。”这样模棱两可的回答。

如果特工意识到自己抵达的并不是原来的世界,那这意味着他开始更加深刻地理解卡尔克萨:自己所处的世界只是卡尔克萨无数浮沫、万千可能性的极一小部分,卡尔克萨的旅客永远回不到出发的地方。这种极其可怕的虚无主义可能会消灭特工继续解决问题的决心,请掌局者谨慎考虑是否要使用这样的设计。

\section{乘客}
剧院中的乘客来自诸多不同的时空。这些乘客保留着理性的残渣,怀着无尽的空虚徘徊在剧院之内。他们成为了剧院的一部分,死亡或生命这些世俗的观念在这里毫无价值。

他们不能真正地被杀死——他们只是一些信息,被破坏的部分会以难以理解的形式恢复原状。他们裸露的伤口之下不是血肉,而是棉花和齿轮(这些填充物之外的部分则是黑洞洞的)。

有一些信息是这列火车上乘客的“常识”,他们会对这些信息毫不怀疑:
\begin{itemize}
    \item[\#] 自己所在的世界名为卡尔克萨,这里的国王名为黄衣之王,没有人知道他的真名是什么。
    \item[\#] 自己所处的具体位置名为剧院,是一个行驶在哈利湖湖底的一种用于旅行的机械结构。
    \item[\#] 自己正在朝着一个化妆舞会前进,但没有人能说清楚这列火车什么时候会抵达目的地,或许“明天”就到了。
    \item[\#] 世界上的所有人都出生在这列火车之上。 
\end{itemize}

这些乘客在走廊中走动、坐在靠窗的一侧发呆,或者在包厢中干一些满足欲望的事情。如果特工想找人,他们会推荐特工前往餐车,或者去询问“卷心菜国王”,但同时神秘地建议特工考虑好措辞。如果特工希望离开这里,他们会表示奇怪,“因为世界除了‘这里’没有‘那里’了”。

在走廊、餐车中可能遇见“悖论”的成员,他们的应对和在\textbf{夜间的闹钟酒吧}中类似。出现在闹钟酒吧中的其他人物也可能会出现在这里,例如那个末世论者。

除了上述人物,特工可以遇到两个很特殊的人物:“自己”,以及亚当森·考克斯。

\subsection{迷途羔羊,特工自己}
特工可能会发现匆匆忙忙在乘客中穿梭的“自己”。

这个“自己”是以下可能性中的某一个:1)失忆之前闯入“剧院”的“自己”,和特工同属于一个“现实”但位于时间轴上的不同位置;2)来自其他“现实”的“自己”,和特工出于相似的原因出现在这里,这两个特工之间可以交换情报;3)特工幻象构建出的“自己”,说话的语气会和剧院中出现的亚当森别无二致;4)迷失于剧院的“自己”,无论是否阻止“悖论”成功,由于腐化过深最终出现在这里。

掌局者可以根据情况,借助“自己”之口揭示一些特工尚未发现的信息,例如一些关于“悖论”的事实,或者是特工对于剧院的理解。如果特工和过去的“自己”交流,那么他会发现一些传递给过去的“自己”的信息开始出现在自己的脑海和现实生活中,仿佛一些笼罩着“现实”的谬误得到了修正。

这些信息不一定“正确”,但它们必须有助于找到“正确”的解释。这个“自己”也可以帮助揭露部分黑线行动的事实。和“自己”交流会失去 1/1d6 的理智值,并腐化值+4。

\textbf{可选:一种表现手法}

你可以在这里尝试使用一种奇特的表现手法,来传递不同的“自己”的统一性:特工在和“自己”交流时,会感到自己的主体在这两个实体之间不断地穿插:在特工说话时,这个感受和正常的交流无异;在“自己”发言时,特工则进入了“自己”的视角,视线中出现的是特工,他能感到到“自己”嘴唇的移动,但无法控制“自己”的嘴唇,但在“自己”说出口的一瞬间,特工就会产生一种“确实是自己说过这句话”的感觉,这种感觉甚至早于自己听见“自己”发出的声音之时。

\subsection{迷途羔羊的羔羊,亚当森·考克斯}

剧院中的亚当森是复杂思想的结合:特工记忆中的绿色三角洲的同伴,特工保护自己远离黄衣之王腐蚀的屏障,以及黄衣之王引诱特工揭开过去的诱饵。

亚当森可以鼓励特工与非自然对抗,也可以劝阻特工趁为时未晚前立刻停止调查,回到自己的生活,也可以对特工讲述关于卡尔克萨的知识,使得特工在泥淖中越陷越深。甚至在上述几种情况中快速切换。

\section{剧院事件}
在剧院之中可能存在着诸多超现实的现象,但这些现象在剧院之中都是事实。特工可以在剧院内遇到任何进入过或者将要进入这里的角色。我们在这里给出一些例子。调用这些遭遇的建议参见\textbf{运行剧院}。

\subsection{餐车}
一个和维多利亚餐车别无二致的车厢,可以在\textbf{《火车的历史》}中介绍东方快车的部分找到和这个车厢几乎完全一致的图片:一个L型的吧台在车厢的一端,一个动作机械的侍者在吧台后摇晃着调酒杯。两排长方形桌子分列两侧。特工可能在这里遇到任何可能出现在“剧院”中的人物。

\subsection{吉姆的幼年}
“我喜欢火车。它有自己的轨道,它的乘客都将抵达最开始的那个目的地——无论他们多么意见不合。乘客没有跳下火车的权力,观看窗户外的景色直到终点是他们登上火车时就决定好了的事实。”

在剧院中推开某扇门时,特工会注意自己的视线变得低矮——他进入了1995年失踪前吉姆的视角,出现在夜间的格林伍德宅邸之中,抱着玩偶坐在一个狭小的卧室床上,听着门刚被摔上的回响。“自己”正赤裸着身子,衣物被丢在床上的另一个角落。深蓝色的窗帘遮住整个窗户,月色透过窗帘穿进来,把整个房间照射出一种幽暗而宁静的蓝色。

特工可以听见从隔壁房间传来玻璃砸碎的声音,女性啜泣并咒骂的声音和男性的哼哼声。如果特工打算离开自己的床前往门前,会听见房间中传来说话声——这个声音来自自己怀中的玩具。

\textbf{达夫}
这是一个30厘米高的狮子玩偶,它陪伴吉姆度过了童年,吉姆给它起名为达夫。它在卧室中目睹了这个家庭的悲剧,它所认知的世界是吉姆告诉它的世界。

在剧院中,这个说话的玩偶是吉姆幻想的产物。它会带着青年变声时的那种傲慢而温和的嗓音自称为“达夫”,特工可以从这个玩偶的口中知道“自己”是谁(“格林伍德十世”),或者现在是什么时候(“吉姆历10年”),但都是以一种搞怪式的回复。

在对话结束后,他会从肚子里掏出一本《瓦普内克剧院》递给特工,“来,你快画完了。”

如果特工在现实世界中来过这间屋子,那么他在离开这个房间时会离开剧院进入“现实”,特工也可以借由这个事件从“现实”进入剧院。在离开这个房间时特工的视线会恢复正常,并且发现手中正拿着这本未完成的《瓦普内克剧院》。

\textbf{未完成的《瓦普内克剧院》}
这是一本尚未完成的绘本,它的内容会随着故事的推进而轻微地变化(其它绘本副本的内容会默契地和这本绘本保持一致)。如果特工试图在这个绘本中绘画,他会感觉自己手上的线条正在撞击着自己的笔触,让它偏离特工期望的轨道(腐化值+4,理智检定0/1)。特工需要通过一次成功的\textbf{腐化值}检定来在这个绘本下画上自己设想的内容——并且特工会注意到自己绘画的内容一部分出现在了剧院之中。

这个未完成的绘本实际是和剧院联系的实体化,特工借由这个绘本实现的“心想事成”,和特工在剧院中的“心想事成”其实是一回事。但这个绘本的存在可能会诱导特工更加主动地实践自己的想象。

\subsection{渐淡的伤疤}
这是特工自己的思想与剧院融合的产物。推开门,特工会回到亚当森·考克斯死亡的现场。门外是一片超自然的景象:门外是无边无尽的海床,一个废弃的工业园区位于海床之上,围绕着这扇打开的门。视线的正中是一个正倒塌一半的工厂,刚卷起沙石的风压从闪烁着的爆炸中心扑出来——这一切都像是被暂停一般停止在原处。

与之相对的,这些液体则可以自由流动,但不会涌入剧院。从水底往上看,可以看到水面上的波光组成了一张飘动的网。

在爆炸中心附近,特工会发现被炸得腾在空中的亚当森·考克斯;如果特工朝着工业园区外部移动,则会发现在园区门口持枪待命的特工“自己”。

亚当森和“自己”都被定在原处,但他们可以和特工交流。特工可以从他们口中得知大部分关于黑线行动的事实。二人的交互逻辑参照\textbf{迷途羔羊,特工自己},\textbf{迷途羔羊的迷途,亚当森·考克斯}。

不久,从远处传来朦胧而模糊的救护车和消防车的声音。然后特工会感觉到这种声音突然逼近,一阵眩晕后,特工发现自己回到了剧院。(这里或许也可以回扣一次开头的梦境,将当时救护人员将特工运出爆炸现场的真实场景重现一遍。)

\subsection{婴儿床}
这是剧院中的格林伍德宅邸一层的储物室。这是一个不足十平方米的幽闭空间,除了一扇门外全都是光秃秃的墙。

正中的地板上孤零零的婴儿床中蜷缩着一具已经有些腐坏的男性尸体,黄褐色的液体顺着婴儿床流下,滴在在周围的地板上。这个满脸酒疮的男性头颅上戴着一个巨大的鹿角头冠,头颅上有一个被子弹钻出的大孔。如果特工见过马里亚诺的照片,那么他可以认出这个男人是谁。

尸体的肚子有一条巨大的缝,缝中连接着一个子宫一样的器官。子宫里本来应该是婴儿卧躺的地方,塞着狮子玩偶达夫。

“这是一个谎言!” 这具尸体中蕴藏着吉姆对自己追求的怀疑,吉姆将其舍弃在这里,从而创建出具备“完美”概念的达夫·格林伍德。这具尸体会主动与特工交流——它裂开的子宫一张一合,仿佛是一张挪动的嘴,尸体传达的内容则会以吉姆的声音直接传入特工的脑海。

从尸体这里,特工可以得知吉姆的愿景:创造一致的、永不能抵达的目标。尸体会告知特工有一个名为“达夫”的怪物在这里诞生,它是那位不可直呼名称的王的碎片,是纯粹的可能性的集合,达夫不能被瓦解,因为被瓦解这个可能性本就是它的一部分。只有将剧院与“现实”的联系切割开,才能暂停这场噩梦,但他并不知道该如何切割。

尸体否认吉姆所追求的“永不能抵达的目标”,它认为这只是人造的虚无而脆弱的欢愉,是自欺欺人的谎言,是将苦痛背负于自身的自我满足,以及剥夺可能性的罪。尸体会带着和剧院格格不入的强烈感情,请求特工帮助吉姆和约瑟芬解脱,阻止他们的仪式。

\subsection{造梦机}
特工会在某个车厢中发现一台富有蒸汽时代风格的机械装置:古旧的二色曲面显示器位于机械的中央,下方是两个喇叭和一个投币口,右侧安装着一个带有圆球的扳手,上方有一个刚好能使得纸张通过的长条缝隙,一张很长的纸带从中间伸出来,左侧挂着一副可以摘下来的像是望远镜的笨重眼镜。尝试理解这个机械的构造和功能会丧失 0/1 的理智。

这个机械以不知道何种原理运行着一个微型世界(并\textbf{不一定}是特工所处的那个),输入的纸带会影响这个微型世界的运算,投入的带有遗传物质的胶囊会决定模拟的对象,使用者可以戴上左侧的眼镜以模拟对象的身份体验和影响这个微型世界。这个机械不能逆向计算,并且在遗传物质耗尽后会自动暂停计算。

\subsubsection{纸带}

纸带上以话剧台本的形式事无巨细地记录着特工从追踪“悖论”开始的所有行动,一直到特工最后一次进入“剧院”前结束。这份记录中含有长篇累牍的对于动作和环境的描写。特工可以从这个台本中得知失忆之前的大部分信息(但其中存在错误)。这本台本默认读者知道故事开始前发生的一切,因此台本即便会涉及部分比这更早的事情(比如黑线行动),也不会对其进行任何解释。

\subsubsection{运行造梦机}

拉动右侧的拉杆可以造梦机继续工作:伴随着拉杆咔嚓的响动,拉杆开始以不稳定的频率摇晃。在喇叭发出“已匹配”的声音后,机器开始正式运作:上方的缝隙开始吐出更多纸带,显示器上出现逐渐清晰的图像,喇叭播放出一些刺耳的音响效果。此时戴上那一幅眼镜会被强制拉到“现实”继续自己的行动,并且进行一次\textbf{意志*5-20}检定。如果检定成功,特工在“现实”中行动时可以清晰地知道自己“正在使用某个机械进行观察”,并可以主动地回到这个车厢中来。但无论如何,在“现实”的剧情只会向前发展半天,特工就会被强行拉回这个房间,随后喇叭会以生硬的机械合成音提示到:“请植入一枚新的胶囊。”。

\subsubsection{投入新的胶囊}

在这台机器旁的罗筐中还有不少的空白纸带,以及一些圆形透明小盒(每一个小盒上标注着某个特定对象的姓名、样貌,以及一串用以确定起始随机值的编码。这上面的名字大多是特工的重要之人或者本模组中登场的角色。如果后续通过科学手段检验这个小盒的内容物,会发现这是对应人物的染色体溶液)。

投入这些小盒,造梦机就会开始模拟这些人物在“现实”中的生活,例如被黄衣之王腐蚀的好友正在大街上插科打诨,或者特工的伴侣由于与特工的亲密关系值下降而正在行不忠之事等等。其展现的生活\textbf{大多}是真的。此时戴上眼镜,特工会以这些人的视角进入他们的世界(同样需要进行意志检定)。但特工最多能在这些世界中行动两个小时,然后就会被强行拉回这个房间,并伴随着合成音:“警告。胶囊匹配度过低,已弹出。”

使用这台机械会使得腐化值 +4。

\subsection{咣物瓜王国的国王}

特工可能会进入这样的一个空阔的车厢:它其中没有其他车厢中的包厢和桌子,而只有一片开阔的场地。地面上摆放着各式各样的玩具。此外,绘画着毛线球和屎壳郎的纸团在这个房间中扔得到处都是。房间四面墙上使用水粉绘画着日常的图案——看上去像是把格林伍德家宅邸的娱乐间平面化画到墙上并反色的结果。

房间的中间摆放着一个张由纸做成的长桌,其上手绘的图案象征着这是一个台球桌。一个同样手绘而成的卷心菜正坐在这个台球桌的中央。

\subsubsection{卷心菜国王,喀巴奇·劳·四肢纤细}

喀巴奇·劳·四肢纤细身体是一页页片状的结构,它底部支出的四根宛如菜叶的触须构成了四肢,和地球上的卷心菜极为相似。一个金黄的纸质皇冠戴在它的头顶。

喀巴奇·劳来自被黄衣之王征服的安塔里斯(Antares,心宿二),它曾经是这个星球上一个名为咣物瓜(Gkanghughuaa)的小国的国王。在永不停歇的舞曲终于逼疯了王后之后不久,喀巴奇·劳也终于成为了黄衣之王的俘虏。喀巴奇·劳总是带着居高临下的口气和不可一世的傲慢,使用诘屈聱牙的措辞与乘客交流。吉姆在旅行的过程中遇到这位漂泊的国王,并把它画进了绘本中。

在见到特工的时候,喀巴奇·劳就会询问特工是否认为自己正在一本书中,如果特工回答是,那么这位国王就会展现友好的态度;反之则会让这个国王突然发起攻击。

特工可以和喀巴奇·劳进行敏捷对抗以逃过追击。如果特工在与国王的战斗中落败,那么他会被吞下并排泄出来。一次成功的\textbf{魅力}检定会让特工变成一个毛线球,反之则会变成一个屎壳郎。特工在剧院中每经过两个车厢可以进行一次腐化值检定以摆脱这种状态(每一次尝试+2腐化值)。

喀巴奇·劳知道很多剧院相关的知识,特工可以从他这里得知以下信息:

\begin{itemize}
    \item[\#] 它是马里亚诺谋杀案时吉姆的同伙,他能够重述马里亚诺谋杀案的所有细节。但他和吉姆并不太熟,无法提供关于吉姆的太多信息。
    \item[\#] 他说王的使者就行于这列车之中。
    \item[\#] 卡尔克萨是“无数陨落王国的尸骸与温床”。
    \item[\#] 剧院是存在于卡尔克萨中的一本书,又是“现实”与卡尔克萨之间的通道。王腐化的众人共同创造了这个想象的空间,被腐化者的思想可以对它进行干预,使它朝着希望的方向发生变化。“如果你希望离开这里,那么你可以试着想象离开这里。”
    \item[\#] “死亡”在这个空间中的居民不会真的死亡,而是永远停留在这个夹缝之中,寻找前往卡尔克萨的道路。在这个空间中的疼痛和死亡会提供被伤害者瞥见“那个舞台”的机会。
    \item[\#] 一些关于咣物瓜王国的信息。
\end{itemize}

\textbf{建议。}掌局者可以让喀巴奇·劳说出很多和剧院机制相关的东西,但不要太过直白,也不要太早。请将其更多作为间接验证特工部分猜想的途径。

杀死喀巴奇·劳,可以从它纸片化的身体上发现关于咣物瓜王国的信息,从这个国王诞生一直到它的妻子在其身体上完整刻下《黄衣之王》剧本结束(腐化值+6)。

\subsubsection{咣物瓜王国}
这个位于安塔里斯星上的国度存有对于强有力藤蔓的疯狂崇拜,并且拥有在同族身上刻下咣物瓜文字的习俗。等到同族死亡后,他们会把同族的尸体展开晒干,然后当作书籍使用。而就在卡尔克萨王国的外交团到来后不久(使官人人都戴着苍白的面具),咣物瓜的居民逐渐开始崇拜卡尔克萨,他们在同族的身体上刻下《黄衣之王》的台词,这些台词逐渐覆盖了原本刻下的这个民族的文化和科技的文字。到最后,这个不明所以的剧本替代了整个国家的过去,一个王国覆灭了。

\subsection{通往愉悦之路}

\textbf{注:}如果特工已经经历过\textbf{剧院事件:八次鞭子},我们建议您略去这一事件。

特工可以在一个车厢之外听见鞭子抽打声和嚎叫声。推开门可以看见在一个布置得和娱乐室几乎完全相同的车厢,其中是正在举行仪式的“悖论”三人组。

赤裸全身的约瑟芬和吉姆蹲坐在台球桌上,纠缠在一起。约瑟芬正发出尖锐的叫声,血液从她佩戴着金属饰品的伤口流出,将乱糟糟的台面染得更加令人反胃。约瑟芬的肉体上刻满了各种不同形状的黄印。皮特会在一旁以朝圣者的尊敬态度向吉姆递去新的工具。两人会在台球桌上进行血腥的交媾、喊叫和舞蹈直到筋疲力尽。他们沉浸在这仪式之中,完全不会意识到特工的存在。即便受到特工的攻击,也只会发出由于喜悦而发出狂欢的大叫(并随后复活);也或者他们注意到了特工的闯入,然后进入\textbf{剧院事件:八次鞭子}。

如果特工会由于目击暴力失去 1/1d4 的理智值。在仪式结束后,这三个人都会陷入短暂的呆滞状态中。此时亚当森会从不知道什么地方冒出,拽着特工朝着门外跑去,并听见亚当森的声音“不能,他不能直视。”

如果特工没有离开这里,那么他会看见从吉姆的身体中分裂出来一个同样赤裸着的男性,这个男性本该长着头颅的位置是一团不断跳动的黑线。特工会在目击这团黑线的瞬间回忆起来自己遭遇的一切,并目击达夫创造的“静止”奇景。目击达夫会遭受 1/1d10 的理智损失,并增加 6 的腐化值。吉姆会主动介绍这个新出现的人物是达夫·格林伍德,但达夫并不会对特工的行为有任何反应。

\subsection{八次鞭子}

吉姆会盛情邀请特工加入自己与神取得联系的仪式,仪式分为两个阶段。

阶段一:特工会遭到连续8次鞭打。每两次鞭打,特工受到1d2的伤害,并需要通过一次体质检定,否则特工会晕倒过去。这八次鞭打会在特工的后背烙上黄印形状的伤痕,特工增加等同于理智损失值的腐化值。参与这一仪式会令特工失去 1/1d10 的理智值。如果特工发现自己身体上刻下的是黄印,则额外失去1/1d3的理智,并增加2点腐化值。

特工每遭受两次鞭打,可以看见后续场景中的一个:

\begin{itemize}
    \item[\#] todo
\end{itemize}

阶段二:如果特工撑过了仪式的前奏,那么他会被询问是否愿意继续仪式,这种诱惑仿佛来自脑海的深处并难以拒绝(意志检定),如果特工检定失败,那么他们会和特工一起在这个房间中开始野蛮的舞蹈(失去1/1d3的理智,以及增加2点腐化值)。在舞蹈之中可以看见的景象:todo。

如果特工成功地拒绝了这一请求,他们会摇摇头离开这个房间。如果特工昏迷过去,醒来的特工会发现这两人已经不在这里了——他们并不在意特工的生死,只是对特工感受不到美而遗憾。

在这个过程中,他们不会响应特工的一切询问和抗议:约瑟芬在整个过程中都一言不发,吉姆除了诱惑性的询问之外不会交代任何事情。吉姆沉浸在一种忘我的感情中,一个可供审美的对象,一种灵感的刺激。而在约瑟芬的眼中,特工是一个延申他们欲望的玩具。在整个流程结束后,特工可以和逐渐平静下来的吉姆展开一些交流。

\subsection{虚假的离场}

特工可能在想着离开剧院时,腐化值检定失败,因而并没能真正离开剧院。他会发现自己来到了一个和现实世界无异的地方,失真症完全消失了,特工可以在这里和日常生活中的朋友和亲人度过一顿愉快的晚饭,甚至继续开展自己的调查。这里的一切都正常地运作着,直到随着一声鲸鱼的声音贯穿天空,一条铁路从天边浮现过来,自己四周的景象都像是画布一样投射到车厢的墙面上,然后消失了色彩和声音——特工发现自己一直没能离开剧院。理智检定 1/1d6。