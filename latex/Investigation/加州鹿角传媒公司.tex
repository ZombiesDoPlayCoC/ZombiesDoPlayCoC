
\section{加州鹿角传媒公司}
加州鹿角传媒公司是一个小型的广告公司。互联网上可以轻松找到这家公司的信息:它于 2005 年在奥克兰本地成立,以迅速、专业、优质的售后服务广受好评。其营业范围为商圈电子屏幕的广告位出租,以及为这些广告投放者提供技术服务。公司的注册信息显示这是一个不足 10 位员工的小型公司,在奥克兰一共拥有十五块用以招租的屏幕。奥斯丁·罗德里格是公司的创始人和目前的负责人。

公司为每块屏幕都安排了专门的技术顾问,以保证广告投放方的需求可以及时得到响应。皮特·戴维斯就是奥克兰中央广场的技术顾问。

\subsection{前往鹿角传媒}

鹿角传媒公司位于 todo 大厦第九层,和另一家小型广告公司共用这层写字楼。

腐化值超过25时,特工可以注意到天空中漂浮着的两条巨大的光带在这里形成了一个折角。

一进门就能看到公司清爽简洁的办公区域:一个小型的大厅中放置了四个客服人员的卡座,两侧则是几个独立的办公室供技术顾问和负责人奥斯丁使用。特工可以在进门的走廊处看到员工的职业照片墙,其中就有皮特·戴维斯。

如果特工第一次前来鹿角公司,他刚进入房间就可以听见从里面传来的一个中年男人的声音,“好的,我们会尽快派人员处理。嗯,是的。”随着电话挂断的声音,男人怒焰爆发了出来,“该死的皮特小子在哪里!”这就是奥斯丁·罗德里格。

\subsection{奥斯丁·罗德里格}

奥斯丁是一个看起来五十岁左右的中年男性,穿着休闲西装,轻微的啤酒肚,一头有些秃的棕色头发,眼睛美丽而温和的蓝色给这幅平庸的皮囊提供了一些点缀。他正坐在自己的长条办公桌之后对着空气大发脾气。他刚刚接到了来自奥克兰中央广场屏幕广告投放方的投诉:商家在两周前提出了更换广告的诉求,但这一变动一直没能落实。

奥斯丁奉行“顾客至上”的古朴服务理念,在特工走入之时就会立刻收起自己的怒气,并为自己的失态而道歉。奥斯丁为自己的员工感到自豪,也会很愿意和特工分享自己的治理技巧。

如果特工问起关于这个“皮特小子”的事情,他会如实回答皮特最近一两个月总是心不在焉,尤其是最近的两周,甚至开始忽视顾客的请求,这两天甚至没来上班。奥斯丁完全不知道原因。

如果特工希望进入皮特的办公室,奥斯丁会面露难色,以这是皮特的私人空间而婉拒。如果特工亮出警方等官方身份,可以令奥斯丁立刻松口同意——难以掩饰的担忧浮现在他堆笑的脸上。

\textbf{一件小怪事}
奥斯丁可能会提及近期和皮特的一次交谈:10号那天夜晚,自己正要下班时听见皮特的办公室中传来一些响动,于是去敲了敲皮特办公室的门,听见里面传来皮特·戴维斯的声音。奥斯丁在门外询问皮特出了什么事,皮特只是简单地回答顾客有着急的要求需要处理。奥斯丁没有多问就离开了办公室。再次提及这个事情的奥斯丁突然露出疑惑的表情:“但是那天好像没有人看见皮特小子走进办公室……而现在顾客的需求也没能得到解决。”

这一天皮特从“剧院”回来取自己的电脑,开始为预演做准备。

\subsection{皮特·戴维斯的办公室}
环境腐化值:5。

皮特的办公室位于这层写字楼的一角,一边是可见街景的落地窗,另一面连接着公司大厅。靠窗的那一面本来整齐地码放着皮特收藏的磁带架子,现在磁带架子倒下,磁带散落一地。皮特的办公桌上看起来反常地空空荡荡。而根据奥斯丁的描述,本应放在办公桌上的电脑现在不翼而飞。

在他的抽屉里可以找到一沓照片。照片中的一部分是从较近的距离拍摄了一个持有相机的男性正在和一位女性交合的画面。奥斯丁辨认不出图片中的女性;如果特工已经见过了约瑟芬,那么他可以从图片中辨认出来。照片角落的拍摄时间显示这些照片都拍摄于两三个月前。

照片中的另一部分则看起来都是夜间从一栋房屋(格林伍德宅邸)的窗户外偷拍得到的照片。照片中总是一个人正在向另一个人疑似施虐的景象。但是由于夜间的厚重噪点和过远的距离,难以从照片上辨认出主角。其中一张照片拍下了这栋建筑旁的路牌,路牌显示这栋建筑位于罗伯勒多大道。特工可以凭着这张照片找到格林伍德宅邸。照片角落的拍摄时间显示这些照片都集中拍摄于一个半月前左右,并且最近一个月没有照片。

\subsubsection{从这里进入剧院}
幻象阈值:21。

皮特在被黄衣之王腐蚀之后的一段时间仍然在这里工作到深夜,将办公室和“剧院”建立了联系。特工会发现巨大的落地床窗户变成了一扇木质推拉门,推开这扇门就会直接走进“剧院”的走廊。