\section{格林伍德家族宅邸}

这栋建筑位于奥克兰的东南部的罗伯勒多大道,紧邻一条穿过市中心的铁道。这是一个三层楼高的建筑,有一个独立的车库和不太大的庭前花园。花园中不对称的位置放置着两个大理石喷泉,被歪歪倒倒的杂草团簇着。

整栋建筑采用米白色的墙面喷漆。房屋的正面是一条门廊和正对花园的窗户。这些窗户上都进行了复杂的雕花,但看起来已经有些时间未经打理了——整栋奇怪的建筑在现代化的都市中显得格格不入,但是又奇怪地并不被周围的人注目。大家只是觉得这是一个衰败的家族。

特工可以依靠在皮特·戴维斯办公室找到的不雅照片,或者调查约瑟芬相关的信息找到这里。特工腐化值超过25时,可以看到天空中漂浮着的两条巨大的光带在这里形成了一个折角。

\subsection{调查格林伍德宅邸}

\begin{itemize}
    \item[\#] 在城市记录不难找到这栋建筑的历史。这栋建筑由建筑师阿萨·达利邦迪受格林伍德家族委托设计,于 1925 年完工。它的上一代主人是马里亚诺·格林伍德,现在由其妻子约瑟芬·莱特所有。
    \item[\#] 向街坊邻居打听现任主人:街坊不太愿意和约瑟芬打交道,只是同情又厌恶着这个女性。街坊会嚼舌头说约瑟芬不守妇道,经常带不同的男人回家。不过最近这位男性好像固定了(这个男人是皮特·戴维斯),尽管街坊认为这个只是暂时的。一些街坊会神秘地透露说,自己从来没见过这个男性从屋子内走出来。他们表示约瑟芬迟早会把他吃干抹净。如果特工拿出在皮特那里发现的不雅照片,他们会指出照片中的女性就是约瑟芬。
    \item[\#] 向街坊邻居打听宅邸往事:老一些的街坊邻居会提及马里亚诺自杀案和吉姆失踪案两件怪事,并暗示这两件事都归功于约瑟芬——街坊认为约瑟芬嫌弃自己的前夫马里亚诺,但又觊觎他的财产。她在杀死马里亚诺后用一大笔钱打通了关系使自己脱离嫌疑;又顺手处理掉了可能目击了过程的儿子吉姆。
    \item[\#] 前往社区管理部门:特工可以得知这栋建筑的水电和互联网都在正常地运作,尽管业主已经由于迟缴费而被警告了数次。最近一次缴费在1月底,负责收费的人员表示约瑟芬看起来既兴奋又睡眠不足。收费人员难以掩藏对于这个女人行为的无奈。
\end{itemize}

\subsection{监视格林伍德宅邸}
\begin{itemize}
    \item[\#] 这栋屋子门窗一直紧闭,窗帘也拉得死死的。白天没有人或车辆进出,房间内也不像有人活动的痕迹。
    \item[\#] 到了接近半夜的时刻,房屋内的灯会亮起(“悖论”进入了宅邸)。
    \item[\#] 午夜之后,可以从二楼的主卧听见鞭打声,以及一个女性和两个男性隐隐约约的喊叫声。如果特工和约瑟芬、皮特或者吉姆交谈过,特工可以分辨出他们的声音。这个声音会一直持续大概两小时。
\end{itemize}

\subsection{造访这座宅邸}
环境腐化值:10。

这栋别墅一层门廊左侧是一个独立式车库,门廊进去之后是一体式的厨房、餐厅和客厅,储物间放置在通往二楼楼梯转角的位置,卫生间在它的隔壁。二楼是主卧、马里亚诺的书房、客卧、卫生间,和与门廊在同一面的阳台。顺着楼梯继续向上是一个儿童卧室(吉姆的卧室),娱乐室以及和门廊同向的露台。

房屋中的家具被摆得七零八落,原本放置在客厅里的沙发桌子都被堆在了角落,在客厅中间留出一大片空地。速食包装盒和没有及时处理的腐败食物打包在垃圾袋里,堆在了厨房的洗手台中,一些褐色的液体粘连在地面上。整个房间中都是一种腐败的气味。但并不难看出有人在这里生活的痕迹:这个巨大的空地上有一些新鲜的像是被剐蹭过的深褐色血迹。

这栋建筑的其他角落也是一团糟:卫生间已经丧失了它的功能,沾着血迹的纸团被扔得到处都是,把马桶堵得满满当当;原本光洁的墙面现在布满了黑色或绿色的不明缘由的划痕;主卧的床看起来至少有两周没有被使用过,浓厚的灰尘覆盖在掀开的被套上;客卧的床上空空荡荡,有很长时间未被使用的痕迹。

\subsubsection{储物间}

一层储物间的门被用木条钉上了,撬开木条的钉子进入这个房间,会闻到一股浓烈的尸臭,理智检定 0/1。储物间的地面上留有一圈人形的黄色污渍(这是达夫·格林伍德从马里亚诺·格林伍德的子宫中诞生的地方)。如果特工的腐化值达到30,可以在这里\textbf{剧院事件:达夫的诞生}。

\subsubsection{书房}
书房仍保留着当年繁盛的一丝痕迹,一个气派的书桌背靠着窗户。但曾经摆放在这里的各国的奇珍异宝都被销售一空,只留下一些放在地面上不明所以的底座和光秃秃的墙面。

这是马里亚诺·格林伍德被杀死的房间,这里可以安排一些特殊的幻象,例如让卷心菜国王和吉姆同时出现,还原杀死马里亚诺的一幕,或者让特工看见自己从窗户中跳出推开了自己消失在了走廊里等等。

\subsubsection{吉姆的卧室}
环境腐化值 15。这里是现实与“剧院”连接最紧密的地方,是吉姆的疯狂想象诞生的角落。

吉姆的卧室位于顶楼,由于房顶倾斜的形状显得压迫而低矮。卧室的正门正对着从二楼上来的楼梯,因此能听清楚二楼的动静。房间正对道路的一侧是一扇巨大的落地窗,其余几面墙都被油漆漆成了海蓝色。墙上有一些海草和水母图案作为点缀。在距离地面不太高的一些墙面上有突兀和独立的水粉颜料笔触。

倾斜的天花板上安置着一个带着浅蓝色灯罩的白炽灯。开灯的时候,从灯里散出的光线会为房间罩上一层更轻盈的蓝色,灯罩上的一个鲸鱼形装饰物会在墙上投下一个鲸鱼形状摇晃的阴影。

一张长约一米五的床摆在房间一角。在床的斜对面是一张靠着书柜的书桌。书桌上有各种水粉颜料的笔触,被一大堆揉在一起的纸团覆盖,一些水粉笔和早已干掉的颜料放置在工具盒中。

\textbf{团圆照。}一张家庭照片放置在书桌上,但是属于父亲的那三分之一已经被撕去。照片中三个人正站在一个有不少水上游乐设施的地方,背景中可以看见被撕去了一半的招牌 “……aterpark”。这张照片背后写着:“终点站”(Terminal)。

\textbf{纸团。}纸团上是使用不同的笔写的“吉姆·格林伍德”,有一些纸条上会落款时间。可以发现这些纸团从1996年开始一直到2006年都有,但没有2007年的。如果特工的\textbf{法证学}不低于40\%,他可以认出这些笔迹都来自同一个人(约瑟芬·莱特)。如果特工此时还没有获得《瓦普内克剧院》,那么特工可以在这里的纸团之下发现一本。

\textbf{书柜。}书柜中摆放着大量书籍,从百科全书到爱情读物一应俱全,出版日期囊括1920s至2000s这个时间段。

\textbf{《无门之界》。}一本出版于 1936 年的小说,作者为艾米琳·F·菲茨罗伊。书中描写了一个无门的国度,这里的所有人都由于听到一种特殊的音乐而不能停下自己的无尽重复且无聊的生活。小说的主人公是来自无门国度之外的一名叫艾比的小女孩,她击败了音乐的源头,希望将人们解救出这种生活。但是事与愿违,陷入寂静的城市开始崩溃,女孩不得不成为了新的歌唱者,将世界重新引入了无尽的重复。书中的一页写着“我想和她交朋友,和艾比,和艾米琳,和……”。阅读这本书使腐化值+4。

如果调查这本书的作者,会得知她12岁出版了这本书,并在26岁于家中失踪,直到五年后由她的哥哥宣布死亡。(关于这本书和作者更详细的内容,可以参照《不可思议的风景》第一幕。)

\textbf{《火车的历史》。}这本书发行于 1930s,它的装订都被翻烂得不成样子。封面内的第一页上写着“奥克兰铁路总局赠格林伍德。”,在这句话的下方是两行歪歪扭扭的字:“喜欢火车。它有自己的轨道。乘客们都只有一个目的。大家可以在上面打牌,看风景。火车上没有争吵,因为大家有目的。我希望找到那永不能实现的目的。”(如果特工此时腐化值不低于 25,他可以听见幼年的吉姆在耳自己的背后说起这一段话,后接\textbf{剧院事件:吉姆的幼年})。

\textbf{《卡尔克萨王族食谱》。}这是一本属于卡尔克萨世界的书籍,它记录的是卡尔克萨被黄衣之王腐蚀前王族的食谱,主要是对于一些水生生物烹饪流程的描述。书籍上标注自己出版于“AKY 15”(After the King in Yellow 15 年,实际不是英文)。书的排版和百科全书很像,有着大量的水生生物插图以及放置在插图旁的文字。这些文字实际是按照韵歌的形式记录的料理这些水生生物的方法,例如“斝砉鱼沾蓖葺酱”。书中文字形似楔形文字,但如果检索相关材料,一个成功的\textbf{历史}检定会表明这不是人类已知的语言。如果特工的\textbf{科学(生物学、植物学)}不低于40\%,他会发现这种特殊结构的生物很难在地球上生存。

如果特工的腐化值不足 25,他甚至不能看懂标题,可能只能猜出这是一本介绍水生生物的百科全书;相反,特工可以毫不费力地阅读这本书,甚至无法意识到这并不是英文。这本书的序言中会表述“这是对不再存在的食材和前代王朝宫廷菜谱的整理”,旁边有一段手写英文批注“一个奇怪的推销员硬要给我的”。此外,书中一页关于贝肉的食谱旁边有另一段英文批注:“我尝试割自己成三角形的小块,但它会长好,如果我想。这有点疼,但也很快乐,我有一种感觉。我尝了几块,是甜的。”

\textbf{《黄衣之王》。}幻象阈值 45。这是吉姆曾经从那个书店中获得那本的《黄衣之王》剧本。红色的皮质封面闪烁着油腻的光彩,看上去和普通的书别无二致,封面上什么没有写下名字。

在夜晚降临时,这是皮特、吉姆和约瑟芬进入格林伍德宅邸的通道。特工可能通过\textbf{剧院事件:吉姆的幼年}抵达这个房间,或者从这个房间进入这个剧院事件(幻象阈值 30)。

\subsubsection{娱乐室}
娱乐室是吉姆和约瑟芬举行爱之仪式的地点,这里充满血、精子和汗液的腥骚味。房间的中央是一个标准大小的台球桌,旁边放置着几把靠椅,几根台球杆放在角落里。一根沾血的鞭子挂在门把手上,一些奇形怪状的宛如刑具的金属制品,和一些粗细不一的绑带被挂在原本放置台球杆的架子上。台球桌被褐色的血迹染得非常斑驳,有些痕迹还非常新鲜,构成了一个黄印。

特工在午夜时分会在这里发现正在举行仪式的吉姆和约瑟芬。详见\textbf{剧院事件:通往愉悦之路}。

\subsubsection{前往“剧院”}
幻象阈值:21

特工注意到吉姆的卧室和书房的窗户外边并不是应有的街景,而是一扇离地半米高的门,这是两个通往“剧院”的通道。宅邸一层储物间的柜门也连接着“剧院”。