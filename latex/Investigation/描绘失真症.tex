\section{幻觉与现实的边界}

在特工调查的过程中,他能够看见的内容是幻象和现实的混合。这里幻象既可能是特工曾经见到的内容,也可能是黄衣之王影响下产生的内容,也可能是自己希望见到的内容。幻象可能无缝地出现在特工的视野里,并在游戏的前中期尽量持续且自洽(但并不需要完全自洽,甚至玩家的一些对于剧情的不合理指控都是可以接受的)。本模组会在有必要出现幻象的场合,描述对应的幻象,当然你总是可以按照自己的理解和推进的情况发挥出新的幻象。

因此特工可能会逐渐发现自己可以一定程度地按照自己的期望操作“现实”,但是尽量让这一切都显得巧合而自洽,成为一个隐秘的伏笔(这也是可用于“叙述性诡计”的一个工具)。在玩家看来特工目击或遭遇的非自然在驱使自己的特工迈向疯狂,但实际上只是在封禁的记忆唤起他已经疯狂的神经。