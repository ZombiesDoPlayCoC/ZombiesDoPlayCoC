\section{“凌晨十二点闹钟”酒吧}

闹钟酒吧位于辛普敦大厦的负一层,毫不引人注目。这家酒吧前身是罗伯特古董书店(就是吉姆找到\textbf{那本书}的书店)。1996 年,约瑟芬买下此地并将之改造成闹钟酒吧。在约瑟芬独特的管理风格下,酒吧满溢的荷尔蒙和腐败的气息吸引着这一片区中的颓丧人士。

\textbf{建议。}闹钟酒吧是本模组的核心,是大量人物和信息交汇的场景。特工会在这里找到故事的核心人物、众多被动进入“剧院”的契机,以及找回自己丢失的记忆。一股脑地将这些都交给特工不是明智的选择。如果特工不愿意主动离开这里调查,你可以在适当的时候利用外界的突发事件将特工引出这个酒吧,让闹钟酒吧揭露的真相呈现出层次性。

特工在腐化值超过25时,可以看见天空中漂浮的两条巨大的光带在这里形成了一个折角。

\subsection{调查闹钟酒吧}

特工可以借助IP地址调查到辛普敦大厦的位置,St. Illness 预告片的内容提示了酒吧位于地下。特工可以借助一些其它的调查确定这家酒吧的位置。

\begin{itemize}
    \item[\#] 谷歌地图:在辛普敦大厦的位置标有“凌晨十二点闹钟”酒吧的图钉标签,信息显示这家酒吧位于地下一层,24小时营业,且目前正在营业中。
    \item[\#] 城市记录:城市记录可以确认辛普敦大厦建造于 1985 年,其中负一楼同年售卖给了罗伯特·R·罗伯特,由他开办了罗伯特书店。罗伯特书店在 1996 年转让给约瑟芬·莱特,约瑟芬建立闹钟酒吧营业至今。
    \item[\#] 询问街道上的其它商户和行人:“有这么一个店吧?但是不是很早以前就倒闭了?”他们的回答总是模棱两可,透露出一种不确定感。
    \item[\#] 询问辛普敦大厦的管理人员:他们对这家店毫无印象,甚至对是否存在地下一层这个事实都感到疑惑。但在简单的文件搜索后,他们会惊讶地发现这家酒吧清清楚楚地写在自己的文档之中,甚至总是及时提交租金。
    \item[\#] 电话:如果通过谷歌地图或者辛普敦大厦管理人员找到了这家酒吧的电话,那么一个自称约瑟芬的大嗓门女性会接起这个电话。她表示酒吧还在营业。
\end{itemize}

\subsection{BAR NOT FOUND}
在故事一开始,闹钟酒吧就和“剧院”建立了深刻的联系,它就像是从世界割下的一个碎片,以模糊的姿态存在于现实世界之中。尽管这个酒吧仍然被纸质材料忠实地记录着,但它已经开始从附近居民的认知中被抹除。

对于腐化值低于15的特工来说,负一楼的酒吧看上去只是一个正在转让的老玩具店铺,以非常微妙的存在感躲在那里。它俗套过时的装潢不会激起任何人的好奇心,临街玻璃积着厚重的灰——看上去任何一个城市的落败街角都存在的那种庸俗的商铺。

如果靠近这个店铺,特工能够听到里面传来的架子鼓声和对话声,但透过店铺布满灰尘的玻璃看过去,里面空无一人,且看上去很久没有营业。如果凝视这个玻璃的时间太长,他会注意到玻璃上厚重的灰上有一些被擦去的痕迹,而这个痕迹组成了一个熟悉的符号(腐化值+4)。

如果特工执意要进去(比如破坏“玩具店”的门窗闯进去),那么特工在进入这个“玩具店”的一瞬间,会发现自己来到了一个“热闹非凡”的酒吧:老板正在柜台上托着自己的下巴,懒散地看着这个突然闯入的客人,酒保在桌椅间穿梭,零零星星的客人坐在自己的座位上。由于无力感失去1/1d4的理智值,腐化值+4。

对于腐化值不小于 15 的特工来说,这个酒吧显然正在营业,并且温馨的气息从其中洋溢出来:负一楼的门面装置着一个有些坏点的霓虹灯,上面写着这个酒吧的名字:“12:00 a.m. Alarm”,名字的下方有一个高脚杯样式的霓虹灯条,在夜晚它会发出不均匀的光。

酒吧的正门是两扇对开的木质门,门上有一个刻着 1996 的金属铭牌。从门内传出来的交谈声和玻璃杯碰撞的声音表示着这家店仍在营业。如果特工在见过它未营业的状态后,看到这个营业中的酒吧,会由于无力感失去1/1d4的理智值。

\subsubsection{幻象:他/我造访过这里}
幻象阈值:15。

这是特工第二次进入这家酒吧。第一次进入酒吧时,他在这里目击了二月十二日的演出。在进入酒吧的一瞬间,特工会从酒吧门上玻璃的反射看见自己的倒影,但是倒影上是亚当森的脸。理智检定 0/1。这个幻象停留一瞬间就消失了。一个成功的\textbf{智力*5}的检定可以帮助特工理解镜子中的亚当森正在和自己说些什么,掌局者可以在这里放入一些暗示特工记忆的信息。

\subsection{监视闹钟酒吧}
监视这个酒吧会是一个谨慎而朴素的选择。白天这家酒吧不会有顾客进入;太阳完全落下之后,一些零星的客人开始往里面走。一直到了第二天太阳升起,才会有人稀稀拉拉地从这家酒吧离开(并且这个数量看上去比进去的人数更少)。

如果特工在监视的过程中由于黑夜的影响使自身的腐化值超出了 15,那么他会看见这家酒吧的灯光在黑夜降临的过程中逐渐亮了起来,窗户上映着移动的人影——尽管自己没有看见任何人进入过这家店铺。特工不会在监视过程中看到约瑟芬、皮特或者吉姆从这里离开——他们都通过“剧院”移动。

\subsubsection{幻象:他/我闯了进去}

幻象阈值:20。

大约在黄昏的时刻,特工会在监视的过程中看到一个和亚当森很像的人小心翼翼左顾右盼地进入了这个酒吧。这是当时自己进入酒吧时的情形,掌局者可以根据游戏进程填充这里的内容。

\textbf{示例:如果特工的腐化值很高}

如果特工的腐化值已经来到了30或更高(这已经几乎抵达了接近故事的尾声),你可以构建一些更加超现实主义的幻象,掺入难以遏制的性与狂野的气味,让特工意识到自己步入泥淖不浅。

我们在这里给出一个简单的示例,特工和幻象中的亚当森在雨水中交合:就在特工注意到那个进入酒吧的亚当森的时刻,无云的天空开始下雨,雨水在街道上迅速堆积。腐化的贝壳和死鱼虾顺着正潜伏着的特工的靴子边流过。特工感受到自己的肩膀被一个人从后面拍击着,这正是幻象构成的亚当森。亚当森会附在特工的耳朵旁,以潮湿的语气诱惑着特工,“你不能从那个夜晚逃脱自己的命运,如果没有我。你欠我很多东西。”

如果特工不做任何反抗,亚当森会和他缠绵在一起,两个人卧倒在充斥着烂鱼虾的漩涡中合二为一。特工可以通过摆脱幻象的检定来反抗——如果理智检定失败,自己就会像被固定在了泥潭之中动弹不得,只能任由这种温暖而冰冷的渴望舔舐着自己。

特工在第二天醒来时会发现自己正赤身裸体地躺在随便什么地方,并且注意到一些特殊的疼痛和自己性器官被粗暴使用过的痕迹(例如找到几根沾着血的树枝,或者刮着血迹的墙缝和井盖洞眼)。理智检定1/1d6。

\subsection{酒吧之内}
环境腐化值 10。

在 2 月 12 日第一场演出之后,整个酒吧都成为了“剧院”的一部分,并保留着那天演出的影响。但是在白天,这个酒吧会显得“较为”正常。

这是一个长方形的大房间,木桌子、沙发和隔断看似随机地摆放在其中。房间中间留出了一个小小的舞池,正对着一条贯穿整个酒吧的吧台,舞池旁放置着一套架子鼓。酒吧的天花板上悬挂着几台电视,一些安装在承重柱上的灯向下射出紫色和绿色的柔光,除此之外是各种粗犷的走线。

酒吧的四面墙低矮处贴着红黑相间的瓷砖,高处则是镶嵌着不同颜色金属块的平滑水泥墙,墙上不规律地挂着几幅笔触粗犷的油画肖像。

这个大房间连接着老板的办公室、后厨、地窖和卫生间。除开老板外,这家店还雇有三个长期员工和一个临时员工。如果特工留心数一数,他会发现这个酒吧中的顾客包括自己共有十二个(吉姆、皮特、马蒂亚、托马斯、巴特中每少一个人,该数字减一)。

\subsection{酒吧中的线索}
酒吧中的一部分线索被黄衣之王的腐化制造的幻象掩盖着,特工需要一定的腐化值才能注意到。你可以在这个酒吧塞入一些违反常理的元素来塑造违和的氛围,例如在柜台里发现不属于这个世界的人头钞票、浸泡着巨型甲虫的果汁等等。

\subsubsection{计算机与网络}
除了前台用来收账以及监控用的台式电脑,酒吧中看不见别的电子设备。如果出示相关的证件,并通过成功的\textbf{行政}检定,可以获得检查这台电脑的许可。特工可以确定这台电脑上不存有任何和“悖论”相关的信息。并且由于约瑟芬不会使用电脑,这台电脑除了收账相关的文件以外也没有其他的东西。不难从账本上发现这家酒吧夜晚的营业情况比白天好上不少。特工也能很容易地发现这里的监控设施已经很久没有运行了。

一个成功的\textbf{计算机科学}检定可以通过检查这台计算机的路由,发现还有一台计算机正连接着酒吧的网络(藏在地窖中的计算机)。排查酒吧中的数据线路可以发现一条通往地窖的线路。

\subsubsection{幻象:血迹与暴力痕迹}
幻象阈值 15。

舞池中有一大摊新鲜的血迹,其中一些溅到了装饰画上。一些坏掉的凳子和沙发仍然放置在地面上。试图采集这些血液会发现它们变成了腐败的污水。

如果特工带着这些东西去质问约瑟芬,约瑟芬会解释说这是前几日喝醉了顾客打架导致的——一个倒霉鬼被另一位砸了脑袋送了医院。一个成功的\textbf{人源情报}检定会发现她显然是在胡说。如果询问其他员工,他们只会耸耸肩建议去询问约瑟芬。

\subsubsection{幻象:St. Illness 的海报}
幻象阈值 20。

这是一张宣传舞剧 St. Illness 的海报,其中央绘制着一辆行驶在海底的火车的横截面,火车轨道在海底自由飘动,一些海带和贝类挂在轨道上。轨道飘动的线条构成了那个熟悉的符号。一个鲸鱼形象的阴影若有若无地出现在远处的昏暗之中。一对戴着灰色方盒的赤裸男性肢体缠绕,站在火车里的小桌上。海报上标记着预演的演出时间和正式的演出时间。

\subsubsection{酒吧中的摄像机}
特工可以通过一个成功的\textbf{搜寻}检定发现藏在酒吧角落杂物里的摄像机。它目前并没有工作。至少20\%的\textbf{工艺(电工)}可以判断出这是拍摄 St. Illness 预告片使用的摄像机。顺着该摄像机可以发现一条经过掩藏的数据线一直连接到地窖的门。

如果播放摄像机中的内容,可以在视频开头看见皮特多次出现在摄像机前调整拍摄角度,此外还有一个中等身高的男人裤腿出现在摄像头中,但他的脸并没有出镜。

在视频的结尾处,皮特再一次出现在视频中,看起来像是在调整摄像机,此时可以听见一位男性和皮特对话的声音。男性:“很顺利,快要结束了。”皮特:“嗯,吉姆,我想这样一定没问题了。你确定到时候不在这里?这里会更安全。”男性:“嗯,有一个很早以前就决定了的地方。”皮特:“听你的。”

这台摄像机已经被黄衣之王腐蚀了,因此可以在视频中看见一些期望之外东西。此外,使用这台摄像机会引起“悖论”的注意,但他们都会不主动开启与特工的交流。

\subsection{“悖论”成员}
“悖论”三人拥有较高的腐化值。白天黑夜变化对他们认知的改变相当有限。

\subsubsection{约瑟芬·莱特}
这是一个眼神忧伤的中年妇女,嗓门很大。她留着金褐色的短发,穿着墨绿色的宽大袍子,戴着一副灰色镜片的墨镜。在她的锁骨附近可以看见一些伤痕。她有时在吧台之后,有时在自己的办公室里(这里能找到约瑟芬藏起来的花里胡哨的性玩具)。特工偶尔能看见她在与顾客调情。

如果看过预告片,那么成功的\textbf{智力*5}检定,或者重新检视那个录像会使特工回忆起这个女性在预告片中出现过:她当时站在靠近舞台一圈的位置,带着毫不掩饰的笑容注视着舞台。

如果观察约瑟芬锁骨处的伤痕,一个成功的\textbf{医学}检定可以判断出这是在两天之内留下的。问起约瑟芬那是什么,她会随意地说这是情人留下的,并轻佻而真诚地说“你也想给我留下一些吗?”如果特工问这位情人是谁,约瑟芬会说“耶稣”。

约瑟芬无条件支持吉姆和皮特的计划,但对其细节知之甚少。约瑟芬不会承认“悖论”成员的身份。如果特工询问关于“悖论”相关的事情,约瑟芬会如同其他顾客表示的疑惑,表示对这个事情并不知情。一个成功的\textbf{人源情报}检定会发现她的话中有一些迟疑。如果特工询问约瑟芬关于网站的事情,约瑟芬会表示自己不懂那些高级东西。

在约瑟芬的语言中,会出现一个代词“他”。约瑟芬坚信“他”将会把他们带往理想之地,瞥见那位王的面容,获得永恒的幸福。无论如何,约瑟芬不会愿意离开被黄衣之王腐蚀的地界。如果特工通过拘捕等方式将其强行带离闹钟酒吧,也不会使约瑟芬的情况有什么好转(见\textbf{事件:当患者被拘束})。

如果特工成功引诱了约瑟芬,她会同意和特工溜去办公室中鱼水一番(无论男女,但只能是白天)。如果特工看到了更大面积的伤口,则可以发现这些伤痕实际上是无数重叠的黄印——以凸起的鞭痕和齿痕的形式长在这位女性的肉体上。约瑟芬会自言自语:“我应该介绍你给吉姆看看。”

在白天,约瑟芬表露出明显的心不在焉,和她的交流总是上句不接下句:她有时候在和特工说话,有时候则自顾自地说起一些酒吧运作的心得,甚至是突然把特工看作几天前的情人开始调情并动手动脚。

在晚上,约瑟芬还是会在酒吧中,但不再似白天那般放荡和激情——仿佛有什么更强烈的东西在吸引着她。一个成功的\textbf{人源情报}检定可以发现她止不住地朝着吉姆看去。在午夜来临时,她会通过铁门溜入“剧院”。一个成功的\textbf{警觉}检定可以发现约瑟芬的这一行动。

\subsubsection{皮特·戴维斯}
在白天,皮特总是戴着褶皱贝雷帽子,裹着咖啡色毛毯窝在酒吧的一个小角落。这个瘦削的中年男人总是皱着眉头露出一副不满的表情,并且神经质地抖动着自己的腿,用手在桌面上敲出节奏。他耳朵里的耳机顺着线连接到被他攥在手里的破旧 WM-GX410 随身听。他的桌面上放着一摞磁带和一杯苏打水。他看上去正沉浸在思索中,不想搭理任何对话。他腰上的钥匙扣上别着前往地窖的钥匙。

皮特会在每日下午两点左右进入地窖检查奥克兰中央广场屏幕的程序是否能正常运作,其余时间则都在酒吧大厅中。

特工可以试图和皮特展开友好的对话。如果从他的兴趣入手,这个尝试会变得更容易一点。如果\textbf{艺术(音乐类专攻)}大于等于20\%可以从他桌面展开的磁带盒判断出他正在收听的音乐是中性牛奶旅店(Neutral Milk Hotel)乐队1998年发布的专辑《In the Aeroplane Over the Sea》。

如果询问皮特关于“悖论”网站的事情,皮特会装成完全没听说过的样子,甚至掩藏自己的程序员身份,称自己只是一个磁带收藏家。成功的\textbf{人源情报}检定会通过他停下抖动的双腿判断出他正在撒谎。\textbf{计算机科学}大于等于30\%时可以利用电脑相关领域旁敲侧击,让技术上自信的皮特漏出自己了解计算机技术的马脚。

如果借助这些漏出的马脚,或者出示事先收集的皮特的背景资料,会让皮特变得紧张。尽管他还是会一口咬定自己不知道此事。这个时候一个成功的\textbf{警觉}检定会发现他的眼睛有一瞬间看向了地窖的方向。

如果特工提供了确凿的证据以证明皮特的身份,皮特会立刻停止自己抖腿的行为,严肃地承认自己是“悖论”的成员。如果特工表现出任何程度的攻击性,他都会拒绝为特工提供更多的信息;相反,皮特可以成为将特工介绍给吉姆的中介。皮特不会透露除了演出时间之外的更多信息。

即便皮特对特工的身份起疑,或者自己身份被迫暴露,也不会主动对特工发起攻击,而是保持自己的行事步调,并在晚上时将特工的存在告诉吉姆·格林伍德。无论如何,皮特不会愿意离开被黄衣之王腐蚀的地界。如果特工通过拘捕等方式将其强行带离闹钟酒吧,也不会使皮特的情况有什么好转(见\textbf{事件:当患者被拘束})。

\textbf{皮特的随身听}
随身听已经被黄衣之王腐蚀了。煞有介事地和皮特交流音乐(\textbf{艺术(音乐类专攻)}大于等于20\%),或者一个成功的\textbf{魅力*5}或\textbf{说服}检定可以让皮特愿意分享自己正在聆听的音乐。特工可能从中听到任何幻象可能构造的对象的声音(例如亚当森的声音,或者失忆之前的自己的声音,或者是皮特录下的约瑟芬被鞭打时欢乐的喊声)。理智损失 0/1d3,腐化值 +2。从皮特手上强硬地夺过这个随身听会造成他的不满,但他并不会和特工打起来。

\textbf{可选:皮特的舞会}
皮特在夜晚会更加活跃,甚至偶尔会走进舞池找一个舞伴,一边高声唱着歌一边迈着滑稽的舞步。那个曾经出现在预告片中的节奏再一次回响在这个房间中,并且引起其他听众的加入。这会诱发失真症突然发作,这一事件可以作为特工“第一次”进入“剧院”的契机。

\subsubsection{吉姆·格林伍德}
腐化值 60。卡尔克萨的旅客。

皮特不会在白天出现在酒吧之中。当夜晚来临,吉姆会推开铁门进入闹钟酒吧,带着平静的微笑独自坐在吧台旁喝酒。他戴着一条黑红方格围巾和一顶滑稽的牛仔帽,穿着风尘仆仆的皮夹克。吉姆看上去已经有至少30岁,特工无法通过童年的照片将其认出。

吉姆会以温和宽厚的态度和所有人交流,他不会由于任何发生的事情生气,甚至会想办法阻止一些即将发生的矛盾。他说话的声音听起来有点闷闷的,措辞也略微奇怪,例如谓词经常出现在不该出现的位置,偶尔会蹦出来一些不明所以的名词。

如果询问他谁是吉姆·格林伍德,他会回答这个人在很多年前就已经离开了。如果给他展示 St. Illness 的影片,他会惊叹到:“这真是录得很不错,我没有想到。”他不会刻意隐瞒自己是录像中舞者的事实,但他反常的措辞可能会让特工以为他正在否认这个事实。即便是特工直接询问录像中的是否是他,他也会回答:“嗯,如果一次机会再有,可能是。”,或者说“我也希望那是我。”在他的理解中,出演预演的并不是自己,而是某个“自己”,而他并不能保证这个“自己”和正在和特工说话的这个自己一致。并且由于类似这样的原因,他总是以不知可否的语气回答这类问题。

只有两个例外。如果问起马里亚诺是怎么死的,他会毫不犹豫地承认是自己杀了他;以及如果问是否是他对约瑟芬展开了性虐,他也会毫不犹豫地点头,“当我施加或感受到疼痛的时候,我感觉我回到了一致。”这种一致感使他确信行使这一切的就是自己。他甚至会非常细节地描述这两个事件的细节,就像他正在做这个事情一般精确。

并且由于吉姆对于卡尔克萨的理解,他会认为特工是受到当下时空限制的一个脆弱的现象,而非一个时空上连续的实体。例如他对于特工的称呼不是“你”,而是“在这里的你”,或者“在此时的你”等等。他使用的措辞听起来奇怪,但意思上仍然通顺。

如果特工希望和吉姆讨论卡尔克萨中相关的事情,吉姆则会如同孩童一般滔滔不绝,他会描述自己旅行中遇到的各式各样的人,包括喀巴奇·劳国王等等。他也不会掩饰自己对于卡尔克萨的崇拜。

最后,如果特工询问他的目的,他会先岔开话题,举出特工在生活中遇到的幸事,来讨论这些事件是否真正给特工带来了幸福,并会列出一些特工在模组中遭遇的不幸经历(他提起这些事情时就仿佛自己亲眼所见一样)。并在最后抛出一个唐突的问题结束这个问题:“我们是否构建一个永远幸福的状况?”

\subsection{其他顾客}
这里大部分人是 2 月 12 日在 St. Illness 预演现场的观众。普通观众的默认个体腐化值为 5,即他们白天时的腐化总值为 15,在夜晚时为 25。他们仍然可以离开酒吧,但总是会下意识地回到这里(就像特工那样),直到成为“剧院”中的居民再也不能离开。随着夜晚到来,他们对于世界的理解也会发生变化。

在白天,他们仍然保留着一部分现实世界的思维,并且也如同特工那样忘记了演出的事实。他们其中有人会隐隐约约地记得自己曾见过特工,但想不起来具体的时间。

而到了晚上,这些顾客就会由于夜晚的来临变的混乱不堪。他们的言语开始变得神经质,并掺杂一些来自卡尔克萨和“剧院”的术语,比如“邀请函”、“舞会”等。

此外还有一些从其他时空前来被腐蚀者,他们曾经由于不慎沾染了黄衣之王而最终沦落此处,例如来自某精神病院的病人,或者在某绿色三角洲的行动中失踪的倒霉特工,或者曾经出演黄衣之王的舞台剧演员——甚至是未来的特工自己。这些人需要一定的腐化值才能看见。

我们在这里给出一些示例方便掌局者使用。

\subsubsection{马蒂亚·哈尔斯}
腐化值 4。岌岌无名的恐怖小说作家。

这个年轻的作家正在靠窗的台子上奋笔疾书,看上去还不到30岁。她栗色的长发盘在头上,戴着一副有点书呆子气的方框眼镜。她的眼神里有一种稚气未脱的神态,让人觉得她很难成为一个优秀的恐怖文学作家。

如果和她交流,她会认为这个酒吧有诡异又诱人的魅力,总是令自己充满了灵感。她会出于好奇同意特工的一些请求,例如帮助特工分散约瑟芬或者其他店员的注意力;如果特工只是简单地警告马蒂亚这个地方很危险,反而会激起这位恐怖小说作家的好奇心;如果特工遇到危险,她会向特工伸出援手;如果特工向马蒂亚提供了实在的证据证明此地的异常,那么特工和马蒂亚的腐化值均+3。

马蒂亚可能会无意中提起和一位年轻客人聊天的经历:这个客人认可了自己的作品,并提供了一些简单的建议,“如果你想塑造一个恐怖的故事,不如试试使用一些神秘主义的符号?那种从你的脑海里浮现出来,但你不知道其含义的符号,然后对其做出意义不明的诠释。我想这会让人恐惧。”

如果追问马蒂亚关于这个客人更多的情况,她会说自己记不清了(她是在夜间见到的吉姆·格林伍德,在白天时这些回忆变得幽邃而不清晰)。如果特工请求她展示她正在完成的作品,她会委婉地拒绝这个建议。

在晚上,她会变得不那么拘谨,同意展示自己的工作:她正在完成一本叫做《失灵》的小说。一个巨大的黄印占据在手稿的中央。小说的详细内容见\textbf{附录:人物数据-马蒂亚·哈尔斯}。马蒂亚可以回想起来和自己对话的人的样貌。当吉姆正在现场时,她会为特工指认吉姆。

\subsubsection{托马斯·汤姆森}
腐化阈值 30。

腐化值 40。末世论者,来自“1840年的英国”的“埃及”学家。

这是一位穿着中东服饰的中年男性,戴着一副玳瑁色圆框眼镜,蓄着厚厚的胡子,总是用一副玩味地表情微低着头,从眼镜片的上方看着与自己说话的对象。如果不和他搭话,他就会一直喝着酒在自己的草稿纸上写写画画,时不时哼唧一声。靠近他会闻到一股浓烈的体臭。

托马斯在埃及工作期间接触到古埃及祭司留下的神秘学石板,并由此接触到了黄衣之王,进入了“剧院”。他扬言自己从这些文献中发现了世界末日存在的证据。为了表达自己的观点,他会罗列一大串和古埃及历史相关的事件及其对应的时间,并以此作为基础在手边的草稿纸上进行繁复而跳跃的计算。

他同样使用阿拉伯数字进行运算,但其使用方式和现代人类的习惯截然不同,矛盾而费解的等式在推算中频频出现,推算看上去只是胡乱的文字游戏。托马斯的结论:这个世界信息传递的效率正在逐渐变慢,缓慢到支持高等动物思维的器官都会退化,最终进入一种所有生命都没有感觉、混沌到仿佛一切都静止的世界。在那时,世界会进入形式上没有毁灭,但在认知上毁灭了的状态。

如果特工的\textbf{历史} 达到20\%,那么他会发现托马斯描述的埃及历史和当下公认的埃及历史截然不同,比如一些重要的时间节点和当下主流学者的观点有很大的出入,但是仍然能够毫不费力地辨别出他所描述的确实是那个建造大金字塔的古代王朝。如果特工的\textbf{科学(物理、天文学或行星科学)} 达到 20\%,那么他会发现托马斯的结论和1852年开尔文勋爵提出的热寂假说相似,尽管证明的方式十分天方夜谭。

托马斯是一个严重失真症患者的代表。他认为自己是一个早就离开了大不列颠、正在漫长路途中赶路的旅客。现在的闹钟酒吧是一个“站台”,这个站台建设在“剧院”之上,它既独立于“剧院”,又随着“剧院”一同行动。“剧院”的终点是伊提的宴会。

\subsubsection{巴特·伯格曼}
腐化阈值 21。

腐化值 15。这是另一个以为自己正在执行由“特工坎蒂”发布的名为“停机行动”的特工,他从另一个可能性的世界通过“剧院”来到了这里。

巴特会像所有玩家扮演的特工那样向特工主动提问。巴特不会轻易地暴露自己绿色三角洲的身份,但是在提供惊人一致性证据的情况下,他会被特工说服,并交流彼此获得的信息。

巴特正在追踪一个和特工同代号名的绿色三角洲特工A,特工A在汇报了自己被腐化的事实之后就从房间中消失了。特工A,“亚当森·考克斯”和巴特是另一个现实世界小队的三人组。那个世界的亚当森·考克斯在一个名为“黑线行动”的行动中牺牲了。(这可能会引发特工的幻象)

巴特无法成为特工在被腐化之地之外的助力——他会在迈出这个空间的一瞬间就进入别的“现实世界”。至于特工是否能在剧院中再次遇到他,遇到什么时间线的他,就请掌局者决定。

\textbf{另一种可能的设计:}巴特正在追踪一个和特工长得一模一样的绿色三角洲特工A。特工A汇报了自己被腐化后就失踪了,与之同时,“亚当森·考克斯”死在了自己的房间中。嫌疑落在落在了特工A身上。绿色三角洲派出巴特前来追踪特工A。这种设计会让特工与巴特的矛盾难以调和,甚至很可能以巴特偷袭作为开头展开这一次遭遇。

\subsection{幻象:地窖}
幻象阈值 20。

这个地窖如同酒吧一样被幻象隐藏着。在特工幻象阈值小于 20 时,连接着地窖的门在特工的眼中只是一幅绘制着门的装饰画。如果特工决定掀开这幅装饰画,他会发现这幅画变成了一扇木质门。腐化值 +4,理智损失 0/1d3。

这扇门并没有上锁。如果特工进入其中时被“悖论”发现,他们不会前来阻止;如果向约瑟芬请求进入地窖,她也会毫不犹豫地同意。但是只要“悖论”发现特工进入其中,他们就会在特工进入后偷偷锁上地窖的门(参见\textbf{事件:被困在地窖})。

这是闹钟酒吧用来存储各种饮料的地方,房间内部也很简单:地窖中留有一条两人宽的通道,通道两侧是放满瓶瓶罐罐的架子。一个房间临街一侧的墙上有一个窄小的通风口连接着街道。地窖看起来有频繁的使用痕迹。一个成功的\textbf{搜寻}检定可以在这里发现皮特藏在这里的电脑。

\textbf{电脑}:皮特的手提电脑,被六位数密码加密。酒吧摄像机的数据线连接着这台电脑。一个成功的\textbf{计算机科学}检定可以破除这个密码。这台电脑里存有以下信息:

\begin{itemize}
    \item[\#] “悖论”的宣传材料。这些材料也可以在“悖论”的官网或者 YouTube 上找到。
    \item[\#] 一个代码项目。至少30\%的\textbf{计算机科学}可以花费 1D3 小时理解这段代码的功能。这是皮特用来劫持奥克兰中央广场屏幕的代码,它会将摄像机正在录制的内容实时传输到奥克兰中央广场大屏幕上。代码中设定在2月18日晚自动开始运行。
    \item[\#] St. Illness 中富鲁格号使用的乐谱。乐谱的开头相对正常,但后面的谱面则变得越来越杂乱无章,甚至出现了一些莫名其妙的标记。如果尝试哼唱或者演奏这个谱面,都会使得腐化值 +4。
    \item[\#] 一些工作文档。这些文件揭示这台计算机的主人属于皮特·戴维斯,一个受雇于加州鹿角传媒公司的电器工程师。
    \item[\#] St. Illness 预演的完整版。见\textbf{酒吧中的摄像机}。
\end{itemize}

\subsubsection{事件:被困在地窖}

如果特工被锁在地窖中,他会被独自晾在这里直到午夜。以下是可以利用的要素:

\begin{itemize}
    \item[\#] 地窖之中靠近通风口的地方有手机信号,皮特的电脑也连接着网线,可以与外界取得联系。
    \item[\#] 通往酒吧大厅的木门残旧不堪,在使用钝器的情况下,可以通过一个成功的\textbf{力量*5}的检定破坏这扇木门。
    \item[\#] 特工可以通过通风口吸引外面路人的注意力,向他们求救,一个成功的\textbf{魅力*5}检定可以让他们同意进入酒吧营救或者报警。路人可能会表示:“可是这只是一家倒闭了的玩具商店?!”
\end{itemize}

如果特工叫来了奥克兰市警,他们可能会由于找不到前往地窖的门而认为这是假报警,或者约瑟芬会对营救人员宣称自己不知道里面有人,所以不小心锁上了地窖。“悖论”不会对逃出地窖的特工做出什么表示,最多露出一副遗憾的表情。

\textbf{注意。}无论特工是通过呼叫路人还是别的什么方式,使营救人员进入闹钟酒吧对自己展开救援,都会导致他们与黄衣之王接触。这种行为毫无疑问会将更多人拉入深渊,也是特工开始扩散黄衣之王的征兆。腐化值 +4。

如果特工在这里呆到半夜,由于环境腐化值的上升,他会发现四周的场景开始发生变化:奢华的装饰攀上房间的墙壁,潮湿气候造成的石灰墙面上的裂隙变形成了一扇扇窗户,透出外部黑暗而深邃的颜色。火车轮子与铁轨碰撞发出的规律噪声逐渐响了起来。一改此前颓败乏味的地窖风光,特工会发现自己恍惚间来到了一个豪华列车之上。这个过程不是瞬间发生的,而是会持续进行半小时左右。

在这个转变的过程中,特工可以选择立刻逃离这里回到现实世界。但如果在转变结束后,特工仍未离开,那么欢迎来到“剧院”,理智检定 1/1d6。到了午夜,约瑟芬和吉姆会来拜访特工,特工可以在此时夺门而出。吉姆具有强力的操作“剧院”的能力,他会呼唤出粗壮的蛇将特工绊倒,将其捆绑起来。特工可以借助自己心想事成的能力与吉姆对抗,或者一个成功的\textbf{敏捷*5}检定也可以躲过这些障碍物逃出房间。逃离的特工会发现自己站一列维多利亚式列车的走廊上。“悖论”不会对特工展开追击,而是待在房间中继续自己的仪式。如果特工没能逃脱,那么进入\textbf{剧院事件:八次鞭子}。

\subsection{前往“剧院”}
幻象阈值:21

特工会在闹钟酒吧中注意到一扇时不时有人进出的铁门,有的顾客会端着自己的酒杯直接走进去。这就是通往“剧院”的通道。如果特工的腐化值低于21,那他会发现这扇铁门只是一副钉在墙上的画。理智检定1/1d4。

