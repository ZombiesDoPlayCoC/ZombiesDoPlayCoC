\section{亚当森的住所}

亚当森的家位于西奥克兰林登街 2441 号,现在归亚当森的妻子温妮·考克斯名下。这是正面临街的两层小型建筑,采用了这条街上统一的的建筑风格:灰黑色的屋瓦和亮白色的墙面。两侧紧邻着一条单行道和另一栋房屋。其背面是一个小型的公园,其中有一些上了年头的秋千和滑梯。

在亚当森牺牲后,亚当森的妻子温妮搬离了奥克兰。临走前,温妮拜托特工照看这栋屋子并将钥匙交给了他,随后特工将亚当森的住所作为自己的安全屋(当然特工不记得这两件事)。

在特工的影响下,这间住所已经被黄衣之王腐化,因此即使在白天也能看见部分幻象。并且如果特工打算前往自己的安全屋或者回到家中休息,他可能会发现自己不由自主地走到了亚当森的住所。如果特工执意要回到自己的安全屋,那么他会发现自己的安全屋已经有一小段时间没有使用。

腐化值超过25时,可以在这里天空中漂浮的两条巨大的光带在这里形成了一个折角。

\subsection{叙诡:钥匙从何而来?}
如果特工想要进门,他会发现自己不假思索地从包里拿出了考克斯家的钥匙并自然地打开了门。这里可以应用叙诡,例如:掌局者可以假装轻描淡写地描述特工打开门之后看见内部的情况,将注意力从“需要使用钥匙打开门”这一常识性步骤上转移开。

如果特工指出了钥匙从何而来这个问题,掌局者可以回答:“噢,由于你和亚当森关系不错,不久前温妮\textbf{搬离这里的时候}就拜托了你帮忙照看这栋屋子。”故意露出一些破绽。

\subsection{进入住宅}

环境腐化值 5。

这栋住宅的所有灯光都关闭着,但是水和电仍然在正常运作。一进门就能看见通往二楼的楼梯。房屋内的主要生活区域有一定打扫的痕迹,但是看得出很多房间已经有段时间没有使用。空气中有一种略微腐败的潮味。几本书被摞成一摞放在桌面上,有《博尔赫斯小说集》、《纸牌大学(第二卷》(特工或许会在自己的家里发现它的第一卷)等。

\subsection{幻象:《瓦普内克剧院》绘本}

幻象阈值:15。

上述的某一本书会被特工看成一本名为《瓦普内克剧院》的儿童绘本。它也可以被翻译为《折颈剧院》。瓦普内克剧院的原名为 Teatro Warpneck,请一定交代本书的英文原名,这是一个文字游戏(见\textbf{终幕-确定演出地点})。

\textbf{建议。}这本绘本是模组的一个重要道具。特工第一次抵达时的腐化值会使他很难发现这本绘本,但这是正常的——我们希望特工在具有更高腐化值时获得这部绘本。除了亚当森住宅之外,模组中还有诸多可以给出这本绘本的场合,所以不需担心。

这是一本用棉线订成的正方形全手绘小画册,封面用黑色的涂鸦一样的字体写着《瓦普内克剧院》,并且其中各个地方都用相似的笔迹做了各式各样的批注和狂躁的修改。在三分之二的位置之后,绘本发生了巨大的风格转变,变得更加晦涩而张狂,有的时候一页纸被各种几何图形充满。绘本主要描绘了一个名为瓦普内克的剧院中上演的三个故事。

\textbf{故事 1:会说话的卷心菜}

有一个住在火车上的卷心菜,他大声地宣言自己正住在一本书里,但是他的朋友都不相信他。于是生气的卷心菜吞下了所有反对他的朋友,其中好看的被排泄了成毛线球,难看的被排泄成了金色屎壳郎。

\textbf{故事 2:狮子与蛇}

狮子与蛇是一对情侣,这一对情侣吵架了,吵架中的狮子突然痛苦异常。一只金色鹿一样的生物从狮子的肚子中钻出,剧痛杀死了狮子。从此鹿和蛇过上了幸福的生活。这个金色鹿一样的生物的形象有被用不同笔触的线条反复修改的痕迹——似乎作者拿不定它的样貌,乃至在故事的最后,这个小鹿头部的位置并不是头,而是长着鹿角的一大团的黑线。

\textbf{故事 3:害怕衣服的舞者}

有一个热爱跳舞的舞者,他在子宫里就开始活动自己的关节,自从出生就不停地舞蹈。他的家人为他设计了一个自动喂食的器械,让他在跳舞的时候都能够吃饭。但是他的家人无法设计在他跳舞时为他穿上衣服的器械(这太复杂了!),于是这个舞者一直赤裸着身体。这个舞者睡觉的时候在舞蹈,做爱的时候在舞蹈,他在自己家人的葬礼上舞蹈。最后,舞者不小心踢断了自动喂食器的电源,但没有人发现,他就这样饿死在了舞台上。他死后成为了舞蹈的神,他跳着舞命令所有人都必须不停地舞蹈,于是世界上的所有人都赤裸了。

这个绘本也同样在绘画舞台上的景色时,描述了观看这些故事的观众的反映——他们压根没有观看这些故事,他们在舞蹈,拍着自己的手,仿佛眼前的舞台并不存在。这是些带着面具的观众。

图书馆和互联网上没有这个绘本或者剧院相关的信息。如果特工希望将它提交给鉴证科进行鉴定,他可能不会意识到自己正在把一本别的杂书交给鉴证科。

\subsubsection{幻象:陌生之地}
阅读完这本绘本,特工会发现自己来到了一个陌生的地方:一个看起来温馨的儿童卧室,墙面上都有各式各样的图样,窗帘拉得死死的。伴随着从窗外传来的火车车轮撞击轨道的声音。特工一移开脚步,就会发现自己回到了原来所在的地方。理智检定 0/1d3,腐化值+3。

\subsection{亚当森的卧室}
亚当森的卧室位于二层。这个房间配备有一台座机和一台电脑,用来处理应急工作。

在特工进屋时房间的灯关着。打开卧室的灯可以看见:一些纸张胡乱地被扔在地面、一支笔尖歪掉的钢笔躺在一片四散的墨水之中、堆积的快餐盒散发出阵阵恶臭。电脑放置在书桌之上,但歪得不成样子,电源线被粗暴地拔出来了。床被移动到了窗户附近,掀起来挡住了整个窗户。(这是特工自己干的。)

\subsubsection{幻象:他仍在这里}

幻象阈值:11。

亚当森正坐在自己的椅子上,皱巴巴的领带,衬衫上沾着一些血迹,额头上有一片烫伤。他干枯而瘦弱的形象会突然让特工感到已经有很长时间没有见到过亚当森。

亚当森完全不会注意到特工的到来,在特工的干扰下,亚当森才会偶尔注意到房间中的特工。他会间歇地说一些夹杂着非英语的语言的句子作为回应,并夹杂着其它谵语,像是在同时和另外几个人对话。如果特工追问邮件中提到的“选择”是什么,亚当森只会摇头,“\textbf{寻求选择的意义就是选择本身}。”此外可以从他的胡话中抓到特工的名字,“温妮”,或者“瓦普内克”这样的关键词。

\textbf{扮演亚当森}
在本模组中,亚当森是若干存在的结合体,所以他说话的风格和内容会不断发生变化。他可以劝说特工停止调查,紧接着就吐露几句新的线索引诱特工调查;或者表达遗憾,“要是调查的时候,你也在身边就好了”;有的时候掺杂一些对于过去那个事故的讨论,或者说特工幻想出来的亚当森对自己的责备;有的时候就是一些胡言乱语。

亚当森的语气中可能透露出非常细微的反常的暧昧。你总是可以让亚当森的语言变成一团特工不能理解的乱麻,并且悄悄地透露特工忘却的两段过去。

\subsubsection{两张纸条}
亚当森的书桌上摆放着两张纸条。

\textbf{第一张纸条}
这是一张普通的打印用纸,看起来被水打湿过又晾干了,皱巴巴的。其上使用工整的手写字体写着一段乱码,特工能立刻认出这是加密的结果——亚当森曾经和特工约定过一种对本文加密的方法。按照约定的解密方式可以将其解密为:“我的电脑密码”,后面跟着一串英文和数字组成的乱码。

这是亚当森生前委托在银行保险箱中的字条,银行会在亚当森死后自动将这张字条以及一些别的东西交付给特工,特工记得这张纸条是黑线行动之后由组织交给自己的,尽管亚当森还活着。

\textbf{第二张纸条}
幻象阈值:6。

它看起来是一封邀请函,崭新且精致。上面打印着一个 YouTube 视频地址,其下使用打印的花体写着“前次分别匆忙,未能充分表达我们的敬意。我们于此诚挚邀请您前来晚会。”这张纸上既没有写邀请函的目标,也没有申明晚会的时间和地点。这个视频指向《St. Illness》的预告片(见\textbf{St. Illness 与芝诺悖论})。邀请函的结尾处打印着一个黄印。这封邀请函是纯粹的幻象,来自黄衣之王的诱惑,只能被失真症患者看见。

\textbf{关于腐化值的示例。}假设特工在白天抵达此处,此时他的个体腐化值为4,环境腐化值为5,因此他的腐化值为9。因为这达到了看见邀请函幻象的阈值,特工可以看见邀请函,但看不见亚当森,只能听见一些不连续的呢喃声。直到特工看到了这个黄印,他的腐化值涨到 13,达到了亚当森幻象阈值:特工会惊奇地发现亚当森悄无声息地出现在了自己没注意到的小角落里。

\subsubsection{亚当森卧室的电脑}
电脑的电源被从插口中拔了出来,连接上电源电脑会弹出非正常关机的提示,随后开始正常开机。使用纸条上的密码可以解锁这台电脑。这台电脑近段时间的使用痕迹都来自特工自己。

在亚当森死后,这台电脑成为了特工进行绿色三角洲工作的主要工具,并且在调查“悖论”之前的所有行动信息都出于保密考虑彻底销毁了。

检查浏览器记录:2月8日之前的浏览记录都被清除了。2月8日到2月12日之间有大量不同内容的浏览记录,看上去是执行绿色三角洲日常的网络监察职务的结果。值得注意的是,2月11日晚有一条搜索 IP 地址对应位置的浏览记录,但是网站没能记下检索的具体 IP 地址是什么。

如果特工检查“悖论”的个人网站和 YouTube 账号的搜索情况,可以得知它们最早出现在2月8日的检索内容中,并且后续相关的检索力度逐渐加大,直到2月11日晚后相关的浏览频率开始下降。

这台电脑上找不到亚当森发给特工坎蒂的那封邮件(因为它压根不存在)。2月12日则在傍晚之后直到今天都不再有任何检索记录。

如果特工检查最近打开的文件,可以注意一个名为“项目08”的文件夹。

\subsubsection{项目08}
至少30\%的\textbf{计算机科学}或\textbf{信号情报}可以花费 1D3 小时的时间看懂这个项目的功能。其有两个功能:1)给目标用户发送一封带有钓鱼邮件,该邮件中的链接会收集目标用户的IP地址;2)利用该IP地址,在设备上留下一个可以攻击这台设备的“后门”。

检视项目的执行报告可以得知,目前目标邮箱被设置为“todo@bar.com”,且项目已经被顺利执行。在根目录下可以找到执行的结果:一个名为 IP 的纯文本文件,其中记录着一串 IP 地址。简单的调查可以得知这个 IP 属于辛普敦大厦。

\textbf{这个“后门”能做什么?} 特工可以编写新的代码,利用这个后门监视或切断电脑的数据传输、从目标电脑上爬取信息、向该电脑放置任意文件、甚至执行这台电脑上的程序等。特工需要成功的\textbf{计算机科学}检定(记下该投掷值)以及 2d4 小时的时间编写一个能够成功利用这个后门执行上述功能之一的程序。特工每通过这种行为操作一次远程的电脑,皮特可以与上述记录的投掷值进行\textbf{计算机科学}对抗,以判断是否能发现这个后门。发现后门的皮特会关闭后门,甚至反追踪特工。

如果特工选择监视数据传输,他会发现这台电脑在和奥克兰中央广场的某一台设备进行间歇的数据交换。特工能够从皮特的计算机上爬取到的信息请见\textbf{酒吧中的线索:计算机}。

\subsubsection{亚当森死亡之后}
在“亚当森”死亡后,不会有任何人向警察报案亚当森之死。如果特工回到亚当森卧室时的腐化值低于11,他会发现亚当森的尸体已经不翼而飞,此前自己准备的各种材料都变成了潦草的图画,但房间仍维持着此前混乱的景象;反之,特工看见的情况和他上一次离开这里时无异:亚当森平静地品味着自己的死亡。

\subsubsection{从这里进入剧院}
幻象阈值:21。

在特工陷入失真症后就将亚当森的房间与“剧院”相连:这个房间的窗户变成了一扇木门,镶嵌着窗户的整面墙壁都变成了木质结构。推开这扇门,会发现门外是一列正在行驶的火车的走廊——而这就是“剧院”。

\subsubsection{叙诡:联系温妮}
如果特工寻找温妮获取信息,例如特工问温妮这段时间亚当森在做什么,温妮可以选择回答:“我不知道。我想他应该在做自己真正喜欢的事情”等等。通过这种方式营造出一种她还不知道亚当森死亡的假象。

\textbf{可选:遇到回家的温妮。}让特工与正在进入家门的温妮迎面撞见会是一个有趣且危险的尝试。特工或许会选择向温妮摊牌,在这种情况下,温妮会展现出惊人的包容度,甚至会反常地说“这不是你的错。你应该早点放下这件事”等等。并且特工会发现,即便温妮去了二层,也不会有任何惊讶的反应——她看不见“亚当森”。将这一事件安排在模组的开头会让模组的疑点变得太重;将其安排在更后面的流程,例如特工已经开始难以分辨幻象和现实时,或许会有更不错的效果。