\section{NPPD第九区办公室}
这里无法获得太多关于亚当森的信息——他的材料两个月前被绿色三角洲处理了干净,甚至前任领导都已经被调走。但考虑到这里很可能会成为特工自主调查的首选地点,因而是塑造特工前期平静日常生活的最佳素材。

亚当森作为 NPPD 第九区的初始成员之一,在这里有相当不错的待遇。他有一间独立办公室,与办公室大厅之间由一面隔音效果非常好的玻璃和百叶窗阻隔。这个房间目前还空置着,保留着极少量亚当森的东西。

\subsection{伯莎·史密斯}
伯莎是第九区办公室现任负责人,在此处调查需要获得伯莎的许可。黑色的短卷发,没有口红,穿着带有白色细条纹的黑色西装套装。伯莎一个月前从第八区调任到第九区任负责人。伯莎·史密斯被告知亚当森在对网上恐怖分子的联合调查行动中牺牲,但与之同时,也收到了不要随意回答亚当森相关问题的指令。

如果特工不在 NPPD 就职,伯莎会从来没见过特工。伯莎会以部门保密性为由,拒绝特工调查亚当森办公室的申请。伯莎会要求特工出示相关文件证明自己调查行为的正当性。特工如果亮出自己的官方身份,可以通过一个成功的\textbf{政治}检定说服伯莎。相反,伯莎会不愿和特工多嘴,直接将特工赶走。

如果特工让伯莎相信了自己的官方身份,问起亚当森相关的事情,伯莎会说自己才调任到这里一两周,对亚当森不太熟,并且他的东西已经被收拾过了,特工不会找到什么有用的材料。如果特工追问具体的时间,伯莎狐疑地看特工一眼,表示自己不便回答这个问题。

\textbf{如果特工也在第九区工作。}这是很可能发生的情况,这会让此处叙诡变得更加困难。我们可以假设第九区办公室的前任负责人在调任和伯莎接洽时,表示特工可能在和亚当森共同执行任务的过程中受到了精神上的冲击,需要尽量照顾一下她。并且特工作为这个办公室的老员工,受到刚上岗的伯莎的尊敬。

由于以上两个原因,如果特工问起亚当森的事,伯莎会表示遗憾,并表示出明显地回避这个问题的倾向——这个办事利落的女人并不擅长处理这种问题。此外,她注意到了特工从9号开始没有按照要求出勤,但同样出于上述原因(或者特工在调查前老老实实地请了假),她不会主动提起特工没有出勤一事。

亚当森的办公室此时已经被清理一空,完全没有人使用的痕迹,甚至有点清理得\textbf{过于}干净了一点,正像伯莎强调的那样:找不到什么有用的材料。

\textbf{联络前任负责人。}前任负责人由于受到高层的叮嘱,会对亚当森的事实守口如瓶,但他会认出特工的声音,因为特工就是当时前来收拾亚当森东西的人,“噢,是你。不,你知道的比我多不是吗?”更多的追问可能会导致前任负责人的怀疑,直接挂断电话。

\subsection{幻象。这里仍是他的办公室}
幻象阈值:16。

检查这个办公室时,特工会在脑海中闪过一段混杂着幻象的过去:特工的视角中有一个坐在椅子上埋头办公的“亚当森”,自己耳朵边传来“我来收拾你的东西了”的声音,视线中的“亚当森”点了点头。在此刻特工的潜意识中(如果特工此时并未回忆起所有过去),尽管自己执行正在执行收拾亚当森东西的任务,但那个时候亚当森还活着。这是自己真实记忆和现在虚假认知杂糅的结果。

桌子上摆了一台被格式化了的个人电脑,一个飞机模型,一个日历,一堆文件按照亚当森的风格散乱地摊在桌面上,一张和温妮的合照。
