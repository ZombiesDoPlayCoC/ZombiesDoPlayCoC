\section{序幕}

\subsection{幻象:扭曲的过往}

在故事正式开始以前,我们需要以梦境的形式为特工重现黑线行动的结尾部分。但是在讲述时修改其中的少量细节,让“亚当森·考克斯”还好好地活着。

这个序章不需要太复杂,只需要几句话勾勒出亚当森这个人物即可。例如塑造一个特工背着亚当森逃离绿色火焰的场景,亚当森可以向特工交代一些和行动相关的信息,特工可以和亚当森互动几句,背景中的救护车的声音若有若无。作为伏笔,你可以描绘在特工即将离开时看见背后的废墟之下隐隐伸出一只手,或者类似的不安暗示。

讲述这个故事时,特工可以感知到一些异常,例如重心不稳,或者视线扭曲。你可以把这个事件描述得足够逼真,让特工误以为这就是本场游戏真正的开场。随后短促的敲门声会打破这场梦境,把特工抓回“现实”。

\textbf{黑线行动。}醒来后,特工可以回忆起梦中这个场景是“黑线行动”,并且可以想起这场行动的大部分事实。但是特工在最关键的地方出了错——特工以为亚当森和自己一起从工厂逃了出来,被救护人员救走。

\subsection{幻象:停机行动}
晚上 12 点,绿色三角洲的联络人敲响了特工家的门。来访者是一个不太高并有点驼背的中年男子,他浓重的黑眼圈上架着两片圆形眼镜,表情不太愉快。特工可以认出这是自己的主管坎蒂。“晚上好,我无意打搅你。我们还是去车上谈吧。”他的声音非常低沉细小,听上去是一个文质彬彬的老人。

坎蒂会请特工来到他停在特工家门口的桑塔纳里,在关闭汽车的引擎和灯光后,开始讲述这次的行动内容及要求:

\begin{itemize}
    \item[\#] 亚当森·考克斯已确认精神崩溃,组织决定这位特工将要就此退场。组织近期并未给亚当森安排新的任务,因此并不知道其崩溃的原因。
    \item[\#] 坎蒂会交给特工过量的安眠药、一份伪造的遗书、一张伪造的心理疾病诊断书和几张安眠药开具发票。特工需要营造出亚当森是由于抑郁症服用安眠药过量而亡的假象。
    \item[\#] 特工需要明确楚亚当森精神崩溃的原因、这一事件是否存在非自然因素、以及自己被指名的原因。
\end{itemize}

关于特工的问题,坎蒂可能会这样回答:

\begin{itemize}
    \item[\#] 组织是如何得知此事的?亚当森在两个小时前从自己的住宅给坎蒂发送了专用信道的加密邮件,请求接受组织的处理。
    \item[\#] 组织为何只指名了特工一个人执行这个任务?不知道出于何种目的,亚当森在邮件中指名让特工完成这个任务。
    \item[\#] 组织是否能提供假身份以供掩护,以及能不能申请更多人手?组织暂时无法提供掩护身份或者更多人手。但是在提供足够线索的情况下联系他,他会考虑接受这项提议。
\end{itemize}

说完这些要求后,坎蒂就会示意特工可以尽快开始行动了,随后他会驱车离开这里。

这次简报不需要非常自洽,毕竟它是幻象的产物——这种发布任务的形式以及团队的组成都不符合绿色三角洲的惯例。你可以让坎蒂的举止看起来和往常并不完全相同,或者别的细节来塑造违和感。

\textbf{关于幻象产生的物品的处理。}“过量的安眠药……开具发票”等材料,会在特工腐化值低于 10 时露出端倪,变成乱涂乱画的废纸或者别的什么东西。例如,特工会突然发现安眠药开具发票上写的对象不是亚当森,而是特工自己。

\subsection{可选:对于亚当森的另一种处理}

你也可以采取一个更温和的开头:亚当森并没有给坎蒂发送请求处理的邮件,相反,绿色三角洲发现了亚当森的失踪,于是指派特工去寻找失踪的亚当森。这种处理会使得特工不会在亚当森的住宅中见到“亚当森”,但是桌面的字条仍然可以给特工指明探索方向。这种设计不会严重影响后续的发展,简单的修改就能使模组的运行下去。

这两种设计会带来截然不同的开局氛围。你可以根据特工(玩家)的性格和背景自行考虑采用哪一种设计,以提供给玩家更好的游戏体验。模组默认以“杀死亚当森”这一种情况展开后续情节。

\subsection{幻象:特工坎蒂}
特工的主管,真名奥德里奇·莱特福特。在特工的记忆中,这位自称坎蒂的主管在和特工打交道的时候,从未选择在任何光线明亮或者信号良好的地方(这是一次特例),也从未晚于特工们抵达碰头地点或者先行离场——他看起来有很明显的危机意识。

如果特工在之后的流程需要联系上这位特工,掌局者可以选择让特工和幻象的坎蒂联系,或者和现实世界的坎蒂联系。如果现实世界的坎蒂听到特工正在执行一项自己布置的任务,他并不会否定这个事实,甚至会给一些模棱两可的指导,然后调度其他绿色三角洲的探员来监视特工——“如果有必要的话,除掉他”。

\subsection{幻象:来自亚当森的邮件}

“我想我被**影响了…让【特工】来结束。他把它拾起的,我向你保证。通往解答的道路。不要再为这个事情自责,真的。这样…你才会拥有选择的机会因为,落入它之手,还是就这样逃出去。”

这封邮件是特工给自己提供的一个选择:揭开真相并重新落入黄衣之王腐蚀之中,还是保持失忆的状态生活下去。

如果特工好奇这个邮件是不是真的由亚当森发出,坎蒂会表示自己可以保证这个信息的“真实性”。

\subsection{幻象:给亚当森打电话}
特工或许会给亚当森打电话,至于是否接通、由谁接通(甚至可以是一个咒骂特工拨错了号码的陌生人),则由掌局者考虑。特工或许会在之后发现这通电话拨向了一个错误的号码,或者根本就没能拨通。

\subsection{叙诡:特工的家人与队友}
特工的家人或许会被深夜出门的特工吵醒,甚至可能目睹特工在和空气比比划划。但他们不会打断特工这种匪夷所思的行为——因为他们知道特工由于工作的原因,正在遭受精神上的折磨。他们会因此表露出宽容和亲切,并且强调特工需要得到充分的休息,“不会有事的”。如果特工的亲密关系多次目睹了类似的场景,他们可能会认为特工的病情加重,于是偷偷联络特工的心理医生取得帮助。

特工或许还会回想起巴特·伯格曼也曾经是自己小队的一员,但是不知道出于什么原因,巴特·伯格曼已经被调离了小队有一两个月的时间了。如果联络巴特,巴特只会对发生的一切感到遗憾,但他对现在发生的事一无所知,“噢,亚当森早就该去坟墓了。在很早以前他就开始握不稳咖啡杯的小耳朵柄,我都弄不清楚他是怎样握紧扳机至今的。向我们曾经的队友致敬。”