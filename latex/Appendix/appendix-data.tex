\chapter{人物补充信息及数据}

\section{亚当森·考克斯}

已死亡。前绿色三角洲特工,NPPD第九区行动组组长。

汉斯(Hans)是亚当森·考克斯在组织中的代号,曾经加州 H 班成员,在黑线行动中牺牲。特工已经和亚当森一同行动过多次,因此在玩家创建人物时就可以得知亚当森的部分信息(尤其是当特工和亚当森在一个部门工作的情况下)。

亚当森于 1962 年出生于美国洛杉矶。1989 年,他在麻省理工学院完成了计算机博士学位后前往华盛顿,在美国陆军的计算机应急部门工作。

2002 年,亚当森调职加入新成立的美国国土安全局,被编入其下的 计算机应急响应小组。在这里他和自己大学时期的恋人温妮·考克斯重逢,并很快收获了自己的婚姻。

2003 年,亚当森在接触了互联网上的非自然事件后,正式加入绿色三角洲。亚当森作为技术人员,接受组织 C 班的直接监督,正式开始了自己在互联网上寻找潜在非自然威胁的工作,有时也会接受特殊的指令去进行一些线下的收尾行动。

2007 年年初,DHS 进行了大规模的组织改建,成立了专门针对网络威胁的部门——国家保护和计划局(简称 NPPD)。在该部门成立后,亚当森被调入 NPPD 第九区办公室工作,因此举家迁往奥克兰。并在这里继续从事组织活动。

2006 年在“黑线行动”中牺牲。

\section{马里亚诺·格林伍德}

马里亚诺出生于1951年,并在1976年接手刚经过石油危机打击的家族企业。

\section{吉姆·格林伍德}

腐化值 60。卡尔克萨中的旅客。

吉姆·格林伍德于1985年作为早产儿出生于奥克兰市立医院。他度过了短暂快乐的童年,但这份幸福很快随着 1992 年格林伍德油漆厂的没落而烟消云散。家庭中弥漫着难以消散的沉默、暴戾和颓废的情绪,为了逃避这一切的吉姆选择了开始绘画创作。他把小熊玩偶达夫当作朋友,在自己的房间之中和达夫畅想幸福的世界。

他在 1995 年从一个神秘的书店获得《黄衣之王》后,以之为灵感创作了绘本《瓦普内克剧院》。随后吉姆在格林伍德宅邸中发现了前往“剧院”的门,并借助这个通道进入了马里亚诺的书房,驱使“卷心菜国王”捆住马里亚诺,伪造了自杀的场景,将现场布置为密室后从“剧院”离开了。

马里亚诺自杀案结案后不久,吉姆从自己房间的“剧院”消失了。由于没有留下任何痕迹,警察对这一失踪案毫无头绪,只能草草收尾。吉姆开始在卡尔克萨之中若干可能性中旅行和取材,在取材结束后重新回到了这个世界,创作了现代舞 St. Illness,希望借此实现自己的理想。此时的吉姆看上去已经有三十来岁,但实际的年龄则并没有什么意义。

吉姆想要获得永恒的幸福。吉姆希望利用大量人思想的共鸣改变“剧院”,将“剧院”推向无限趋近于抵达舞会的时刻——仿佛“静止”在了终点无限近距离的位置,但无法真正抵达那里。这种“静止”就是达夫·格林伍德的核心。而“剧院”中的乘客会始终沉浸在即将抵达舞会的欢愉之中,这份欢愉会消解一切日常的分歧和琐碎,将人们引导向真正的极乐。

但这只是黄衣之王给吉姆的虚假许诺——“剧院”只是卡尔克萨的一丝轻微的涟漪,而卡尔克萨的一致性存在于其永不停息的混沌与不能穷尽的可能性的叠加中,唯一一致的只有变化本身。对于卡尔克萨来说,“一致的目标”并不存在。无论吉姆选择什么样的目标,都会存在从卡尔克萨传递过来的新的目标,这一目标会揭示一种更高的喜悦,否定当下目标的含义,从而让当下目标所许诺的喜悦化为虚无。吉姆的理想世界对于卡尔克萨来说同别的世界一样卑微而琐碎,不多不少,仅此而已。

\section{“静止”,达夫·格林伍德}

达夫·格林伍德诞生于被想象出来的马里亚诺·格林伍德的子宫之中,是吉姆·格林伍德精神上的重生、吉姆真挚的愿望和感情,也是受到黄衣之王腐蚀的一面概念的实体。

但另一方面,达夫·格林伍德但又完全超越了吉姆·格林伍德的想象,是纯粹的黄衣之王概念的一角,他虚构的身体里流淌着万千的黄印,他是这个空间真正的主人。这种概念如同黄印一般,与之接触就会受到黄衣之王的腐蚀。他同时存在于与他接触过的所有意识之中,只要达夫的存在尚未被忘却,他就会不断地对他寄宿的受害者产生影响,并在受害者无意识的传播下分裂、扩散。

在被腐蚀空间之中,达夫拥有实体以及改变这个空间的绝对权限;而在这些空间之外,达夫则依附于吉姆·格林伍德的实体中。实体化的达夫·格林伍德的体型和吉姆类似,但是一大团球形的无规则振动和变化的黑色线条占据了原本应该是头部的位置,长在身体的脖颈之上。

直视达夫·格林伍德的头部是特工能做的最坏的一件事。这个黑团是可能性世界的集合,它们互相蕴含又无穷无尽:注视者会发觉自己的视线正在不由自主地聚集于其上,这些黑线在自己的视线中不断地放大:特工会发现每条黑线都是一条火车轨道,一些列车沿着这些轨道上按部就班地前进。继续放大:列车之中是形形色色的行人,这些行人在忙碌着做着各种各样的事情、天下所有可能的事情,而这些行人之中有一个自己正在盯着一团黑色的跳动曲线。继续放大,这团黑线仍然是轨道,上面运行着火车。然后一种失重的感觉朝着特工袭来——他看见自己正在一个巨大无比的、毫无规律的铁轨上,并随着无数平行且抖动着的其他巨型铁轨及铁轨上的火车,绕着一个城堡运动,不断地运动。而这些黑线,组成了无数抖动且交错的黄印。

达夫同样象征着“静止”这个概念。如果特工试图攻击达夫,特工会在产生这个想法的当下,立刻感到一种已经成功攻击过达夫的错觉,并紧接着发现自己实际上什么也没有做(例如弹夹还是满的,自己还站在原地)。但是当特工再次想要攻击时,特工会再一次产生这种错觉。特工无法真正抵达攻击达夫的事实,因为他总是以为自己已经完成了攻击——达夫是一种思维上的概念,只是特工在构想中攻击达夫时,达夫就已经遭受了特工的攻击,从而反过来影响特工的思想,使得特工产生了自己已经攻击了达夫的感受。但这如同沙砾一般思想造成的损伤不会在达夫的外表上有任何体现。发现这种悖论的特工遭受 1/1d6 的理智损失。

\section{约瑟芬·莱特}

个体腐化值 35。酒吧老板、吉姆·格林伍德的母亲与情人。

原名约瑟芬·莱特,在 1980 年在奥克兰与吉姆格林伍德的父亲马里亚诺·格林伍德结婚,婚后她放弃了自己的酒馆,改名为了约瑟芬·格林伍德。约瑟芬并没能享受很长时间的幸福婚姻生活,迎接她的是常年苦闷的夜晚。她很为自己的儿子自豪,但是无力踏出家庭的泥淖。她想方设法地创造一个看起来更好的生活氛围,但这一切都是无济于事。

在注意到吉姆绘画上的天赋之后,她竭尽全力地希望让世人接受自己的儿子,但只遇到一次次拒绝。另一方面,吉姆灰暗和疯狂的作品让这位母亲更加挣扎。这份疯狂也开始慢慢影响这个可怜的母亲。在马里亚诺死亡后,约瑟芬立刻意识了这是谁的杰作,并且清楚自己的儿子就要离开自己去远行,然后回来创造震惊世人的艺术。她协助警方草草定案两个案件,并恢复为原名约瑟芬·莱特。1996 年,约瑟芬售卖了马里亚诺·格林伍德除了宅邸之外的所有遗产,买下了罗伯特书店,并将其改造为了现在的闹钟酒吧。

作为《瓦普内克剧院》第一读者的约瑟芬已经性情大变,曾经的苦闷以狂暴的形式爆发出来,阴晴不定、脾气暴躁,她开始放荡而饥渴在酒馆中捕获满足自己肉欲的猎物以及等待着被猎人捕获,但正是这种古怪的性格让她开设的酒馆生意兴隆异常——顾客都说约瑟芬的酒更有味。闹钟酒吧逐渐成为了奥克兰落魄者的避风塘。

不仅如此,对于吉姆扭曲的崇拜在她的心中生根发芽,并且伴随着那种挥之不去的渴望,约瑟芬对幻想中的吉姆产生了新的、违反常理的感情。她从未怀疑吉姆将会再度回到自己的面前,救赎自己,将自己领到《瓦普内克剧院》所描述的那个世界。而就在吉姆重新出现后,约瑟芬成为了第二位阅读《黄衣之王》的人物。约瑟芬幻象的那种关系在她与吉姆之间确立了——她每夜都贪婪地从吉姆的鞭子与鞭子那里感受快感,自己也距离自己幻想的国度越来越近。约瑟芬·莱特,这个悲惨的女人在收获了这无与伦比的幸福的同时,完全成为了黄衣之王和吉姆的俘虏。

\section{皮特·戴维斯}

个体腐化值 20。加州鹿角传媒公司的电器维修员,网站的实际建设者,吉姆和约瑟芬特殊关系的崇拜者。

毕业于 xx 大学,毕业后从事程序员,他以收集各种恶趣味的黄色 CD 和磁带等内容作为私人爱好,并且喜欢拍摄各种违反常理和令人反胃的照片。皮特·戴维斯是闹钟酒吧的常客,也是约瑟芬情人们中的一个。他享受着和约瑟芬的这种随意、不求更多但无比热烈的感情。这是皮特稳定得毫无营养的生活,直到吉姆的闯入破坏了这个平衡。

吉姆的进入夺走了自己的情人,或者自己只能卑微地和约瑟芬在白天交欢——这不是皮特所喜爱的。嫉妒的皮特开始跟踪约瑟芬的夜晚生活,在格林伍德宅邸对面贪婪地窥视一切。

他开始享受观察这对母子(他现在还不知道这是一对母子,在他得知之后,他甚至从这一事实中获得了更猛烈的喜悦)的美丽仪式,并对吉姆产生了兴趣。直到有一天,皮特终于按耐不住更近距离观看的渴望。他尾随着吉姆和约瑟芬穿越了“剧院”,并以此为通道进入了格林伍德宅邸。吉姆和约瑟芬完全没有介意皮特的跟踪,并且邀请他成为仪式的见证者——他在那里见到了达夫·格林伍德,皮特完全成为了黄衣之王的信徒。他在此后阅读了完整的剧本《黄衣之王》。

他成为了吉姆的技术顾问,为这个虚无的剧团建立了私人网站,并如痴如醉地经营这一个电子神龛。他开始夜不归宿,忘记了自己的妻女与事业,并为最后的仪式做着准备——他开始筹划将话剧演出的现场与互联网相连,并且试图将其连接到奥克兰广场的巨大荧幕上。他收到了由特工发出的钓鱼邮件,并在夜间的混乱之中点开了这一封邮件——到了白天,他感到自己犯了什么错误,于是他抱着侥幸心理,将作为服务器的网站电脑草草藏在地窖之中。

\section{马蒂亚·哈尔斯}

个体腐化值4。旅居作家。

一个从事恐怖小说和剧本写作的岌岌无名荷兰旅居作家。她在荷兰完成了自己的学业,但由于自己的作品不被当地认可,于是只身前来美利坚继续创作并宣传自己的作品。她并没有参加十二日晚的舞会,她是十三日才抵达这里的。但她还是误打误撞地找到了这个酒吧,并且被这个酒吧颓废而野性的气味吸引了——尽管没有直接受到冲击,但黄衣之王的种子已经埋下。

在13日夜晚,她遇到了吉姆·格林伍德。马蒂亚和吉姆简单的交流之后,被这个学识丰富的年轻人震撼,她获得了无穷的创作的想法,开始创作一本名为《失灵》的小说。小说的大纲大概是一个自以为背着友人尸体的军人(事实上他背着的只是一袋法式面包),企图乘坐火车抵达宇宙尽头将友人埋葬的故事。马蒂亚希望让这个故事发生在一个名为卡寇莎的虚拟国度中,这个军人会在这辆火车上徒劳地打着转。她甚至构思好了这些途径的地点应该构成什么样的图形:一个黄印。

事实上,特工行走的轨迹(自己的家、亚当森的住所、加州鹿角传媒公司、闹钟酒吧、格林伍德家族宅邸、以及湖滨公园)也可能构成一个黄印,参见\textbf{终幕-确定演出地点}。

\textbf{援手}。马蒂亚是模组为特工提供的一个潜在的援手:她已经被黄衣之王影响,但仍然处在一个清醒的状态中。如果特工获得了马蒂亚的信任,马蒂亚会为特工在酒吧中的行动提供帮助,甚至在最后一夜的演出中成为特工的副手,完成阻止演出的一部分任务。如果特工没有获得马蒂亚的援助,那么她会逐渐被黄衣之王影响,并在最后演出的时刻进入疯狂与失忆(就像第一次目击演出的特工时那样)。掌局者可以考虑在模组进入特工没能顺利地阻止演出,并且陷入完全的疯狂结局时,让玩家短暂扮演失忆后的马蒂亚,给她一些自由行动和探索的机会,以她的视角来描述特工的结局。

\section{伯莎·史密斯}

\section{奥斯丁·罗德里格}

\section{马蒂亚·哈尔斯}

\section{托马斯·汤姆森}

\section{巴特·伯格曼}

\section{喀巴奇・劳・四肢纤细}