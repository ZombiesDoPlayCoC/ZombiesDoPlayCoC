\chapter{后记}

首先对您能忍受这些絮叨啰嗦的文本看到这里表示由衷的感谢。如今本模组的创作也接近尾声,我终于可以放下我久经折磨的脑部,来回忆一些轻松的创作模组的往事,权当作者在收尾时的喃喃自语罢了。

回头检查为了本模组创建的第一个文件已是22年8月22日。要想谈及这个模组创作的始动,得回到21年的冬天。彼时我对跑团已经有些疲软,但在无意中看到了绿色三角洲出版的《不可思议的风景》。这个战役给我的思想带来的巨大的冲击:“跑团的故事可以这么诡谲、不落俗套而性感!”于是我立刻召集固桌,开始了一场并不很顺利的旅途(部分原因归咎于我有一段时间没带团的缘故,对于故事的控制力已经开始下滑了。)这次不算特别好的经历没有打消我一开始就产生的想法:创作一个像IL那样够酷、够癫的故事那样的想法。

但我没有一个很好的切入点,直到大半年后的一天。在我如常听着我珍爱的专辑的时候,我注意到了这么一首歌,来自Radiohead乐队的《Climbing up the walls》。这种带有现代性的失真、错位而冷漠的情绪,就是我想要的东西。于是我决定了这个模组的基调——一个灰蓝色调的、克制但癫狂的宛如爵士的“氛围”模组,就像模组开头对卡尔克萨的描述那般。

模组决定围绕故障艺术,创作一个开局就疯了的特工,他的生活被各式各样的故障艺术所充斥。但这个创作过程并不很顺利。其实在22年10月,这个模组的大致结构,包括场景和人物设计就已经大致完成。在完成大概七成内容后,我开启了初版测试。测试过程中,除了这个模组确实传达出了我想要的氛围,其人物塑造、动机设计、线索排布,以及受这些内容影响的游戏性,我通通不满意。于是模组创作陷入了漫长的反复停滞期,其中大纲推导重来数次,文件夹建了又删。我从最初的单人模组设计改成了支持多人游玩的版本,最后又改回了单人。在后续的时间,尽管人物设定和细节在若有若无地发展,但我一直没能满意。

大概从今年7月,我开始在吃完晚饭后长时间地散步,考虑模组设计如何推进,收听各种乱七八糟的电台来试图找到灵感。但并没有什么用。在9月,似乎一切都很自然开始出现了变化,一些关键的结点联系在了一起,它们自发地演化起来,指涉、递归论和“复活”母题浮现在了脑海之中。我看见了众多被抛弃的灵感和模组的琐碎细节之间的关系,模组中相继糅入了另外几部音乐作品的影子,比如模组中提到的牛奶中性饭店的《In the aeroplane over the sea》,以及用在章节页的马勒《第二交响曲》。如是直到模组创作结束,我开始有点喜欢这个模组了。

另外,虽说在创作过程中我已经在尽量回避IL的影响,但在创作结束的时候回头来看,发现还是几乎没能脱离IL创造的故事基础,作为作者不得不感到非常遗憾(所以在创作快结束时和Pan兄聊,Pan兄说“你这个是大型同人啊”,我也只能苦笑了哈哈)。以及故事的结尾,我最终还是没能完全关照游戏性。尽管写出了自己最想看到的结尾方式,但其自由度和精彩程度可能和主持人或者玩家心中会有落差,所以权当抛砖引玉,请大胆地创造您的结尾吧。

啰啰嗦嗦这么大堆,主要是对这个漫长的创作过程的回顾。这个风格化的模组,从我决定让他以精神病作为主角的时候开始,就注定只能给很少一小部分TRPG玩家带来快乐(如果真的有这样的群体的话。这一原因也让我完全放开了手脚,完成了模组花里胡哨的发癫排版)。

如果您能喜欢这个模组,或者您带领某些玩家体会了这个故事并感到满足,我将不胜荣幸。再次感谢您阅读本模组。如果您们在游玩这个在模组的过程中有难忘的体验,请不要吝啬,将这些美好的回忆也分享给我,我永翘首以盼。

Josep\_h

2022年11月30日