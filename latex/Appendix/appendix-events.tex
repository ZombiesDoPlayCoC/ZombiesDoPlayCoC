\chapter{突发事件}

\section{腐化值上升}

在一开始,这只是一些幻觉,但这些幻象如此真实,甚至会影响到特工在现实世界的功能性,例如在幻象中看到手被割伤,即便特工能意识到这只是幻觉,但仍然无法握稳手枪扳机——那种疼痛的感觉挥之不去。

腐化值的上升会导致特工的失真症越发严重。这里列出一些腐化值和在该腐化值下可能目的的幻觉的强度,以作为参考。

\subsection{意识流}

当特工的腐化值超过 30 时,就可以采用一些更加混乱的叙事,让叙事更加遵循“思想”,而非事实(但仍然需要保留某种自然而流畅的“逻辑”,而不是随心所欲地掐断某一段完整而诱人的叙事,生硬地插入一段毫不相关的B级片画面)。你可以选择在特工与人物A交流到一半的时刻,就立刻开始叙述某件发生在“剧院”中的事件,仿佛上一个时刻压根不存在人物A这个人——只是由于人物A提及的事实和特工自己的记忆交织了——并且在这个事件执行结束后,可以直接将特工放入到“剧院”中,尽管他自己完全不知道什么时候进来的(他总是会在失真症侵染的情况下忘掉一些东西)。

例子:特工在和闹钟酒吧里的巴特交流时,在谈及黑线行动的部分真相时,叙述突然跳转到\textbf{剧院事件:渐淡的伤疤}。

\section{特工的房间}
特工的房间里可能什么时候冒出一个纸条,上面用难以辨认的笔迹写着:“留下亚当森的家。家是那门中一扇,通往那个地方。”或者特工的墙上也出现了黄印。

\section{黄衣之王开始在网路上扩散}

\section{来自心理医生的电话}

\section{达夫的幻影}

\section{过去侵入梦境}

腐化阈值 30。

特工在梦中进入了 \textbf{剧院事件:渐淡的伤疤},并且在醒来时发现自己出现在“剧院”。

\section{特工成为了扩散者}

\section{特工联系了绿色三角洲}
特工可能会主动联系绿色三角洲,请求绿色三角洲的帮助,但这意味着需要提供适当的证据。

todo

\section{调查黑线行动}

调查黑线行动的细节只有一个方法:联络自己的主管,特工坎蒂。todo

\subsection{与特工坎蒂的会面}

特工的这个行动具有极大的被绿色三角洲盯上的风险,参见\textbf{突发事件:特工被绿色三角洲怀疑}。

\section{特工被绿色三角洲怀疑}


\section{被自己的老板责问}
特工的老板可能会突然打电话前来,表示特工已经有很多天没有来上班了,就算是提交了请假申请,也已经超过了这个期限。

\section{当患者被拘束}

这种情况发生于特工试图将一些受到黄衣之王腐蚀严重的受害者,借助警方或者精神病院时。这些受害者包括约瑟芬、皮特、吉姆等人。

在拘束后不超过两天——被拘束者会在夜间,创建一条新的连接“剧院”的通道(往往是以窗户或者门作为媒介),并从那里消失。被拘束者消失的消息也会借助官方传递到特工这里。

如果这一切都发生在特工的监视或者摄像头的范围中,特工会发现被拘束者仿佛是穿过了那些媒介一般,从这个房间中消失了。理智检定 1/1d4。如果特工腐化值抵达 30,则特工可以看见通道在媒介之中生成的过程,并切实地看到这些人是通过这些通道离开的。如果特工腐化值不低于 30 时,且位于失踪案的现场,那么他会发现这条通往“剧院”的道路。