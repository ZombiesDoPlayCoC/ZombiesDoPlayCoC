\chapter{当腐化值上升}

腐化值的上升会导致特工的失真症越发严重,也会相应提高特工在被腐化的空间中的“叙事权”。

\section{特殊技能:“心想事成”}

特工在腐化的空间中具有一定的心想事成的能力。这种能力来源于特工的思想对于卡尔克萨的影响,被思想所歪曲的卡尔克萨会以现实或者剧院的扭曲作为体现。部分简单的愿望并不需要检定便可直接实现。而一些更为主动或者困难的愿望则需要进行一些检定才能成立。例如当特工“希望”离开“剧院”的时刻。

“心想事成”的技能值等同于特工的腐化值(“剧院”和彼岸均提供30的环境腐化值,并且不提供时间腐化值)。无论是否检定成功,应用该技能都会导致特工损失 1D6 的毅力值,0/1d3 的理智损失,以及腐化值+3。

暗骰:在游戏的前期,当特工尚未意识到这个能力存在,或者没有主动要求时,掌局者可以利用暗骰代替特工进行检定。如果检定成功,特工的想法会以隐晦的方法实现,并伴随着一定的失力和晕眩。掌局者可以对这一现象进行一些描述,但是并不把晕眩等现象和其实现的结果直接联系起来。这种情况下,你可以只扣除特工的毅力值,而不扣除另外两项(特工可能将这种毅力扣除当成在异世界停留过久的惩罚,这种理解一定程度上也是正确的)。

明骰:随着游戏的进行,特工或许会意识到自己具有这种能力。他可以主动要求使用这种能力,通过在脑海中努力构想自己希望实现的情形来使用“心想事成”。在腐化值明示之前,都由掌局者进行检定,但特别的,检定中使用的骰子点数公示,并且掌局者不需为特工解释能力是否检定成功,而是直接描述其结果;在游戏的末期,特工的腐化值明示后,该技能由特工检定。这种情况下,特工获得使用该技能的完整惩罚。

对抗:如果两个角色希望争夺空间中的叙事权,那么他们需要进行“意志*5+腐化值”的对抗。这种情况,特工总是获得使用该技能的完整惩罚。

\section{幻象事件}

失真症引发的幻象并不真正存在于“现实”之中,但这些幻象如此真实,甚至会影响到特工在现实世界的功能性。例如在幻象中看到手被割伤,即便特工能意识到这只是幻觉,但仍然无法握稳手枪扳机——那种疼痛的感觉挥之不去。

构造这类事件的方式主要遵循意识流的约束:让叙事更加遵循思想的习惯,而非事实。例如你可以选择在特工与人物A交流到一半的时刻,就立刻开始叙述某件发生在另一时空的事件,仿佛上一个时刻压根不存在人物A这个人——只是由于人物A提及的事实和特工自己的记忆交织了。这种叙述切换的生硬程度和频率应随着腐化值上升而升高。但注意,你仍然需要保留某种自然而流畅的“逻辑”,而不是随心所欲地掐断某一段完整而诱人的叙事,生硬地插入一段毫不相关的B级片画面。

除了模组正文中提及的幻象,这里额外列出一些幻象事件及其对应的腐化值,以供您创作更多的幻象参考。

\subsection{错误的对象}

建议腐化值范围:6-20。

特工在日常生活中将一些正在处理的人或物弄混,或者幻视出一些压根不存在的人物或者把一些规律性的噪声认作敲门声。

\subsection{听筒和花洒}

建议腐化值范围:6-20。

特工误将座机电话的听筒和花洒搞混。例如特工正在和谁打电话时,你可以唐突地切入特工正在淋浴的场景,电话对面的声音从花洒中流出来。你既可以解释成他正在洗澡的时候被之前打电话的场景弄混了,也可以说是他正在洗澡的时候误将花洒当成了听筒,开始了这个荒唐的对话。或许将这两个场景调换。特工意识到这一事实后,损失 0/1D3 的理智值。

\subsection{空转}

建议腐化值范围:6-20。

特工在对话中可能会注意到对方的话只说了一半就突然停了下来看着自己——这是由特工短时间的走神导致的,他没能意识到自己的思想跳过了一个短暂的时间段。

\subsection{串线}

建议腐化值范围:10-20。

在对话时,特工可能会听到另一段对话的声音和当前正在进行的对话重叠,就像是正在听着电台和邻居打招呼那般,你也可以用这种方式将叙事切换至另一个对话的场景。

\subsection{错误印刷}

建议腐化值范围:10-20。

特工可能会在交流过程中,发现对方很难理解自己的意思,尽管自己已经尝试说得非常清楚。但在正常人眼中,特工正在用一种非常奇怪的语法进行措辞。

\subsection{巴特、巴特和巴特}

建议腐化值范围:10-20。

特工在流程中主要有三个地方可能会遇到前同事巴特:闹钟酒吧中,剧院事件“渐淡的伤疤”中,以及在彼岸事件“摩天大楼”中。这三个事件是很好的契机,让你将它们进行糅合。例如特工在和闹钟酒吧里的巴特交流时,在谈及黑线行动的部分真相时,叙述突然跳转到\textbf{剧院事件:渐淡的伤疤},并在这几者之间来回跳转。

\subsection{阻路巨石}

建议腐化值范围:20-40。

特工可能在出门时突然感到头部一阵疼痛,然后发现自己撞上了一块摩天的黑色巨石。这块石头突兀地立在城市的街道之中,而周围的人对其熟视无睹。损失 0/1 的理智值,及 0/1 的HP。

\subsection{入梦}

建议腐化值范围:20-40。

特工的梦境开始和“剧院”建立联系,往事和黄印(那种极具规律性的节奏)萦绕在脑海之中挥之不去,将其成为切换叙事,或者触发事件的接口。例如,特工在梦中进入了 \textbf{剧院事件:渐淡的伤疤},并且在醒来时发现自己出现在“剧院”。

\subsection{积水与彼岸}

建议腐化值范围:30-60。

天空持续不断地下雨,并在地面上蓄起积水。水平面逐渐上升,将城市都淹没过去,死去的水生生物从自己的脸颊旁划过。当特工腐化值高于30时,就会注意到雨的存在,并一直不消失。特工视线中的部分景象开始被彼岸所代替,一些建筑逐渐变化成彼岸中的样子,视线中的人物与自己的交互模式也开始与彼岸中的形式相近。特工此时可能无法辨认道路上的行人或者正在飞驰的汽车,这意味着极大的危险性。

\subsection{深入交流}

建议腐化值范围:30-60。

这是出现在模组正文中用以引入高腐化值概念的示例。特工和某个或某些幻想中的人物进行了迷醉的接触,比如用作例子的如果特工的腐化值很高。一些迷幻的、粘滞的场景在其中诞生。幻象的结束会使得特工损失1D6的毅力,1/1D6的理智损失,甚至是一些HP的下降。

\subsection{被造梦机弹出}

建议腐化值范围:30-60。

特工的视线突然开始扭曲,并猛地发现自己正在使用造梦机。特工需要至少见过一次造梦机才会进入这个幻象,特工可能会发现自己来到了彼岸中但丁的面前,或者在“剧院”之上。

\section{无意识的行径}

特工腐化值的上升可能会让特工做出一些无意识的举动,他混乱、不连贯的思维可能在其无法意识到之时就酿成大祸。他组织行动以外的生活开始被影响。尽管我们没有在正文中太多提及,但是这种影响是描述特工日常的不可忽视的组成部分,我们强烈建议掌局者在塑造故事时,引入少量的日常生活以打乱特工正常的调查节奏。我们在这里给出一些例子以供参考。

\subsection{特工的房间}

建议腐化值范围:10-20。

特工的房间里可能什么时候冒出一个纸条,上面用难以辨认的笔迹写着:“留下亚当森的家。家是那门中一扇,通往那个地方。”

\subsection{灰袍、面具与手稿}

建议腐化值范围:20-40。

特工在自己的衣柜中找到一件灰袍和一个苍白的面具。或者在自己的书桌上找到一些莫名其妙的手稿。甚至是特工的墙上也开始出现黄印。

\subsection{垃圾堆中惊醒}

建议腐化值范围:20-40。

特工突然从某些古怪的地方苏醒,并且衣冠不整,身上的物品也不知道扔到何处。

\subsection{陌生的床伴}

建议腐化值范围:20-60。

特工醒来时发现身边出现了完全陌生的床伴,希望会想起这个事情的细节也只会徒劳地发现自己的头疼得离谱,仿佛宿醉还未解除。注意:使用这种敏感情节需要充分考虑您的玩家的性格以及特工的性格。

\subsection{成为扩散者}

建议腐化值范围:20-60。

特工发现自己的亲密关系或者上司,也开始被黄衣之王所影响——他们的手指关节习惯性地敲打着那种特殊的节奏。

\subsection{臭气熏天}

建议腐化值范围:30-60。

特工在自己房间发现了一些臭味,其来源是在其垃圾堆中被粗暴撕碎的动物尸体。理智损失0/1D3。

\chapter{突发事件和其他可选项}

我们在这里列出一些生活中可能发生的影响模组进程的事件,以用作丰富游戏内容的参考。以及给出一些特殊的可选项(或者说挑战)。

\section{黄衣之王开始在网路上扩散}

尽管绿色三角洲很难注意到,特工会发现那种神秘的节奏开始出现在更多的 YouTube 的视频中,甚至是那个自己熟悉的符号。自己的生活中也开始出现受到其影响的人。

\section{被自己的老板责问}
特工的老板可能会突然打电话前来,表示特工已经有很多天没有来上班了,就算是提交了请假申请,也已经远远超过了这个期限。

\section{来自心理医生的电话}

特工的家人可能发现了特工的精神症状逐渐严重,于是偷偷联系了特工的心理医生。特工可能在傍晚时刻接到来自心理医生的来电,“您明天有时间吗?或许我们可以聊聊。”心理医生会和特工进行例行的沟通。心理医生的反馈可以一定程度反映特工此时的腐化值,并且为特工指出一些他已经开始崩坏的认知结构(如果此时已有的话)。

\section{重要关系出轨}

特工开始在生活中做一些费解之事,或者特工可能为了行动的顺利,严重地伤害了亲密关系,他会在街上发现自己的伴侣正在背着自己偷偷地做某些事情。

\section{特工联系了绿色三角洲}
特工可能会主动联系绿色三角洲的联络人坎蒂,请求绿色三角洲的帮助,但这意味着需要提供适当的证据。特工可以以 YouTube 视频以及被修改的中央广场屏幕代码为由,请求对于网络管理的权限。这需要一个成功的行政检定。特工也可以从坎蒂那里要到大部分黑线行动的资料,包括亚当森已经死亡的这一信息。

联系绿色三角洲,甚至表明自己“正在执行某项行动”,会直接引起绿色三角洲最高等级的怀疑,从而遭到组织的监视甚至是处决。参见\textbf{特工被绿色三角洲怀疑}。

\section{特工被绿色三角洲怀疑}

如果特工的怪异行径引起了绿色三角洲的注意(例如旷工太久),或者主动联络了组织,并暴露了一部分事实,那么特工会遭到组织的怀疑。在3D4小时内,一支从其他地方调来的小队就会盯上特工。特工在从建筑离开时,可以进行一次警觉检定以发现这支跟踪的小队。

这支小队几乎不会主动干涉特工的行动,但是会在发现特工做出一些“反常”的危险举动时将其镇压。如果在终幕前,这支小队已经开始监视特工,那么在终幕特工请求绿色三角洲提供帮助时,他会发现坎蒂几乎没有犹豫,立刻同意了这个请求并挂掉了电话。一支整装的队伍突然出现在了自己的眼前——但他们会拒绝和特工交流。

如果特工试图甩掉这支部队或者做出更加严重的敌对行为,则可能接到来自坎蒂的联络。坎蒂会发起一次新的简报会,并暗自评估特工的情况。如果有必要的话,当场击杀特工。如果特工从那里逃了出来,那么在此后的调查中都需要小心翼翼地避开组织已经掌握的地点。并且要不了多久,特工就可能在酒吧中遇见这支闯入的小队。

\section{当患者被拘束}

这种情况发生于特工试图将一些受到黄衣之王腐蚀严重的受害者,借助警方或者精神病院时。这些受害者包括约瑟芬、皮特、吉姆等人。

在拘束后不超过两天——被拘束者会在夜间,创建一条新的连接“剧院”的通道(往往是以窗户或者门作为媒介),并从那里消失。被拘束者消失的消息也会借助官方传递到特工这里。

如果这一切都发生在特工的监视或者摄像头的范围中,特工会发现被拘束者仿佛是穿过了那些媒介一般,从这个房间中消失了。理智检定 1/1d4。如果特工腐化值抵达 30,则特工可以看见通道在媒介之中生成的过程,并切实地看到这些人是通过这些通道离开的。如果特工腐化值不低于 30 时,且位于失踪案的现场,那么他会发现这条通往“剧院”的道路。