\section{操作本模组}

操作本模组将将是一件并不轻松的事情,你需要随时保持你的想象力,为本场游戏塑造出一个更加癫狂和神经质的“个人风格”——这是一个在游戏进展中仍然在不断变化着的模组,已经发生或尚未发生的事件都可以作为塑造故事的燃料。如何调动这些燃料组成一个将特工推向深渊的故事,将是掌局者在操作本模组时遇到的最大难题。

简单来说,本篇模组有四条线索:1)特工调查线,特工需要沿着现实的线索,逐渐调查到“悖论”所在,并发掘现实的崩坏与黄衣之王的关联;2)新剧本线,亚当森与“悖论”在模组的背后涌动着,推动着新剧本的诞生,并期望用这个剧本创造出通往卡尔克萨的道路;3)特工崩溃线,特工在调查途中将会很难避免成为失真症患者,因而特工的世界也会随着时间的推进而崩溃——特工节制的生活、温馨的家庭以及一切稳固的现实;4)现实崩溃线,除了这两支紧紧围绕着黄衣之王的人马,还有一些不小心卷入这场浩劫的普通人,这些普通人在这个故事中也扮演着举足轻重的作用,突然高发的毫无人性的犯罪、毫不遮掩地聚众淫乱。当然,我们也不能忽视那些卡尔克萨的入侵之物,逐步形成的哈利湖与两颗黑星等等。

这四条线索在故事发展的同时相互交织,并以任何可能的形式咬合在一起:崩溃的现实或许会打断特工的调查,并将其特工面对的恶疾更进一步;也或许特工看见的幻觉可以帮助其抓住新剧本线的蛛丝马迹,将其引导向最终的场景。

我们将在“调查”这一章节将描绘本模组故事的全貌,展现大部分可供调查的场景,这些场景服务于本模组的主线部分——发现并解除黄衣之王的威胁。这一章节将会以地点、事件以及人物为核心,对可供调查的主要场景进行叙述,NPC与特工的交互方式等等也都在这一部分。

“事件”一章会放置一些沿着时间顺序发生的可选事件。这些事件或许与主线相关,或许会以可怖的形式打乱特工们的生活。你需要时刻记住,在本篇模组中,现实的崩坏并不是一个背景板,而是驱动着特工向前的根本动机。它们并不直接影响剧情,但却是特工最能直观接触的深刻变化:不断向前演进的荒诞现实是本模组试图塑造的核心恐怖,在这些内容上花费笔墨绝不会一文不值。我们将在这一章节中描述一些用以塑造这些内容的可选事件,例如对于犯罪事件的报道,或者是一些更加详细的犯罪事实,或者是特工的亲人成为了黄衣之王的仆从。掌局者可以在合适的时机将这些崩坏的现实插入特工们调查的过程中,并在其中安插一些有效的线索。当然我们也并不是只会安排这些灰暗的事件,来自绿色三角洲的援助等内容也将放在这一章节。

\subsection{宿命的把戏}

现实生命的徒劳与宿命,是卡尔克萨影响下的必然结果,只是一些人意识到了这个绝望的答案,而一些人浑然不觉。卡尔克萨蕴藏的无数正在诞生或消亡的可能性,都只是一些瞬间和既定的事实。对于特工来说漫长的挣扎,在卡尔克萨里都只是无用的努力——这一切在尚未发生之前就已经决定了。在本模组中,掌局者有两个玩弄这个把戏的地方。

其中较小的一个把戏为亚当森决定将特工作为自己的继任者,完成这项任务。亚当森在“剧院”和最终的场景遇见了特工,并且得知是自己的信息将特工引导到此处。一种冥冥中的感觉促使亚当森做出了这个决定,并最终导致了相遇。

而另一个较大的把戏则蕴藏在绘本《瓦普内克剧院》之中:特工会逐渐发现自己就是绘本中所描述的主人公,自己所已经经历的痛苦和绝望,都如同预言一般地写在了这本由十岁天才绘制的故事之中;并且他会渐渐意识到自己即将经历的,也已经在这本书中注定。而这是由于在剧院之中,特工将遇到正在创作这本绘本的吉姆·格林伍德,而特工在剧院中的一切遭遇都会被吉姆注意到。
