\section{威胁}

\subsection{卡尔克萨与黄衣之王}

卡尔克萨是由思维组成的混沌区域,也或是爵士风味的铜管无穷无尽的嘶吼组成的集合:万千的世界如同浮沫一般在其中诞生与消亡;而现实世界只是其中的一个微不足道的可悲切片。然而卡尔克萨可以被人类投射出的思维所影响,一些可能性因此诞生或是消除,并最终反映于“现实”之中。而空虚与混乱,是任何试图操作卡尔克萨者的后遗症——这无限的疯狂会摧毁一切人性的正常动机,而卡尔克萨则会占领这片废墟,成为无法动摇的“真理”。

幸运的是,现实和卡尔克萨之间有两层薄薄的壁障:愚昧和对于自洽的满足——这两重壁障阻拦着人类的心智迈入深渊。但是在人类短暂的历史上,一些禁忌的“知识”曾透过了这两层屏障,将暴露于其下的人引导至疯狂。我们将这种“知识”称为黄衣之王。

黄衣之王是一种行走在人类认知世界中的恶疾,它无处不在、无孔不入:接受其光辉照耀的人们将会饥渴地寻求它,并使用它填补自身的空虚(这里曾经填充着名为理智和人性的棉花)。黄衣之王诱使着受害者将这份“知识”传递到更远之处,成为人类历史长河中隐秘的顽疾。

\subsection{黄印}
黄印是象征黄衣之王无穷可能性的图腾,它有着不同的姿态,以最简单的形式传递着来自卡尔克萨的启示。它本来无处不在,只是不被“注意”到。被腐化之人会开始在自己生活的各个角落发现黄印,下意识地刻下自己的黄印(他可能发现自己的左手正在右手手背上比比划划,而构成了一个黄印)。

在本模组中,黄印不仅指代一种特殊的图像结构,也指代一种特殊的振动规则。这种振动规则本身就是黄印,敲击这种节奏和绘制视觉上的黄印并无差异。

\textbf{注意}
在本模组中,我们有时会使用类似“黄衣之王诱惑……”,“黄衣之王驱使……”的句式,这只是为了方便理解而进行的拟人化陈述,并不意味着黄衣之王具有人格或者目的。但这种拟人化确实可能存在于部分黄衣之王崇拜者的认知中。

\subsection{“剧院”}
在卡尔克萨与现实世界之间,存在着一些通道连接着两者,这些通道只有“受邀者”才能发现。它们隐秘而黑暗,从暗处啃食着现实,如同蜘蛛网一般将人类文明逐渐包裹,捕食一个又一个误入其中的理性思想,将之转化为黄衣之王的教徒。

在我们这个故事里,“剧院”(Teatro)就是这样的一个通道。“剧院”是一列蒸汽动力火车,带着工业时代复古而华丽的装潢,永不停歇地穿梭在哈利湖湖底。

“剧院”之中的无穷包间连接着现实和卡尔克萨,它热闹非凡,各式各样的乘客被困于此。所有乘客都认为自己正在前往一个舞会,并且相信自己终会抵达那里。

在“剧院”中的人的生命是某种叠加态,生与死对他们来说没有意义,死去之人会以诡异的形式复活,但复活之人将不再能离开被黄衣之王腐蚀之地。

\subsection{St. Illness 与失真症}
St. Illness 是本模组黑幕吉姆·格林伍德创作的现代舞蹈,它可以利用视觉或者声音中的振动规律传递黄印的概念。观众会不由自主地参与舞蹈当中,将自己的感知和“剧院”融合起来,并能看见超越朴素现实世界的景象。这会带来极大的精神上的满足,但会反过来突显他们肉体感官的空虚,诱发对于性、暴力和疼痛的饥渴。在脱离舞蹈之后,这种空虚会更为强烈。St. Illness 可以被译成“静止病”或者“圣疫”。

“失真症”是接触 St. Illness 而患上的一种特殊病症,患者开始将“剧院”揭示的知识和现实世界混淆,其认知中的世界开始变得荒诞而割裂。一些神秘主义的意象逐渐开始在患者的感知中出现:神秘的符号在视线中形成,来自逝去爱人的声音从耳畔传来,哈利湖的湖水灌入自己鼻腔,湖岸的黑石割伤自己的手臂……

更严重的,这种认知上的影响最终会作用于现实世界,构建出从俗世前往“剧院”的门,这也意味着这些区域被黄衣之王腐蚀,并开始与“剧院”融合。

失真症的表现类似某种认知功能障碍或者臆想症。但事实远非如此——失真症患者透过虚假的屏障,看见了“剧院”所揭示的卡尔克萨的真相:现实世界只是飘渺盲目的哈利湖诞生的一缕微波,“所有的客观存在自始至终都只是盘旋在黄衣之王、戏剧与黄印周围的一圈泡影。除此之外的周遭一切,人类所有的产业与创造,仅仅是想象之余的愚剧。”

\textbf{什么是黑线?}本模组的描述中时常会使用“黑线”这一形容,它被用以描述“超越现实世界的景象与现实融合”这一现象:来自其他可能性世界的物体的轮廓开始与当前世界的物体重叠出现,就像是物体移动的残影那般。
