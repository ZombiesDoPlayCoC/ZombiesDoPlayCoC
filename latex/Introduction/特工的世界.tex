\section{特工的世界}

在游戏一开始,特工就成为了失真症的患者,尽管他并不知情。St. Illness 预演现场引发的心理创伤造成了特工的失忆。但是黄衣之王仍然作用着:制造出幻象,发布虚假的任务,引诱特工重新暴露于黄衣之王的注视之下。

从结果来看,黑线行动的后遗症、预演现场的心理创伤和失真症综合作用,令特工产生了以下的错误的记忆:

\begin{itemize}
    \item[\#] 亚当森并没有在黑线行动中死亡,且在那一次行动之后,自己和亚当森都没有参加过任何新的行动。巴特·伯格曼由于不明原因被调走了。
    \item[\#] 自己前几日正在按照正常工作作息完成自己的工作。自己与亚当森自从黑线行动以来没有再频繁联系过,最近几天更是没有看到亚当森的踪影。
    \item[\#] 特工将自己调查“悖论”的记忆,嫁接到了亚当森的身上:他会认为亚当森由于“不明原因”被腐蚀,并且现在自己需要去将他“杀死”。
\end{itemize}

特工会将一些幻象混进现实世界:往日的记忆、特工的想象、黄衣之王的诱惑,甚至是与“剧院”相连的其他人的记忆。在游戏一开始,这些幻象与上述错误的记忆基本自洽;但随着特工被腐化程度的加深,越来越多自相矛盾的记忆开始从幻象中浮现,打破这种自洽性,将特工引导至冰冷而疯狂的“真相”。也或许……特工会怀疑起自己找回的过去只是\textbf{黄衣之王的骗局}。

特工的病症也会极大破坏他的生活。来自现实世界的琐事开始塞满特工的生活——违停罚单、轧死的邻居小狗、出现在床铺上的陌生床伴,或者同僚的孤立、伴侣的质疑、上司的指责、来自绿色三角洲的监视……更不用说那无穷无尽的幻象和呓语。他可能会发现自己逐渐变成了被现实世界排斥的疯子,而“剧院”变得更加甜美而幸福。

\subsection{腐化值}
游戏中,我们将特工受黄衣之王腐蚀的程度记为腐化值(该值对特工不可见,并会根据时间、场合不同而变化)。腐化值决定了幻象的持久程度和难以被区分的程度,也决定了特工\textbf{洞穿真相}的程度(例如发现通往“剧院”的门、瞥见丧失存在感的闹钟酒吧等)。同时,腐化值也象征特工可以回忆起过去的程度。

特工的腐化值计算公式:腐化值=个体腐化值+环境腐化值+时间腐化值。个体腐化值表示特工自身被腐化的程度,目击、接触或者传播(无论是否特意)黄衣之王都会导致个体腐化值上升。被黄衣之王腐化的场地会提供环境腐化值。白天不提供时间腐化值,夜晚则提供10点时间腐化值。

游戏开始时,特工的个体腐化值为1。在较高腐化值状态下获取的材料可能会在腐化值下降之后被发现纰漏,也可能只是发现它们不翼而飞。我们会给部分信息或事件标注幻象阈值,以表示注意到它们所需的最低腐化值。

以下是导致个体腐化值上升的情况示例:

\begin{itemize}
    \item[\#] 时间流逝一天:+3。
    \item[\#] 目击黄印:+4。
    \item[\#] 第一次观看 St. Illness 完整版:+5。
    \item[\#] 第一次向其他人传递有关黄衣之王的信息:+4(之后每次+1)。
    \item[\#] 第一次进入“剧院”:+8(最高升至25)。
    \item[\#] 主动找回自己丢失的记忆:升至25。
    \item[\#] 成功地判断幻象:-3。
    \item[\#] 失败地判断幻象:+3。
\end{itemize}

以下是不同腐化值区间可见的幻象强度以及与人交流难度的示例。更多具体的幻象示例,可以参照\textbf{附录:腐化值上升}。

\begin{itemize}
    \item[\#] 1-5:幻象几乎不出现。特工可以与正常人顺畅沟通。
    \item[\#] 6-10:幻象偶尔出现,可以通过主动的 \textbf{意志*5} 检定识别幻象。特工可以与正常人顺畅沟通。
    \item[\#] 11-20:幻象持续出现,甚至见到幻想出来的人物、文本,可以通过主动的 \textbf{意志*5-20} 检定识别幻象。特工开始使用一些反常的遣词造句,但在反复的解释下可以与正常人沟通。
    \item[\#] 21-30:幻象持续出现,开始改变大规模环境的特征,例如认为自己身处水底,可以看见四周房屋上挂着海带等等,可以通过主动的 \textbf{意志*5-40} 检定识别幻象。特工的表达开始词不达意,不是所有的人都能明白特工在说些什么。
    \item[\#] 更高:特工具备轻微地将幻象变为现实的能力。特工开始糅合自己从没学过的未知语言进行自我表达,只有少部分拥有耐心的正常人可以借助手势理解特工的诉求。
\end{itemize}

通过吸食精神抑制类药物可以在药效期间内降低 5 点个体腐化值。

最后,由于被腐化的角色可以通过思想修改被黄衣之王腐蚀的空间,腐化值也象征着角色在这类空间中话语权的高低。如果两个角色希望争夺空间中的叙事权,那么他们需要进行“意志*5+腐化值*5”的对抗。

\subsection{叙述性诡计}

在游戏的前期,幻象和现实世界交织在一起,并充满了特工忘记的细节。你或许可以通过利用注意力盲区,或者双关等方式来构造一系列“看起来”自洽的信息。但是注意,这些诡计\textbf{不应当提供错误的信息}误导特工的调查,而只是偷偷地浮现于一些无关紧要的角落。

比如你的上司对你正在和空气说话感到不解,从而简单地问候:“呃……我想你看起来状态不太好,你应该早点休息”——玩家有一万条理由认为特工确实状态不太好,而不是特工正在和空气说话。

这些细节不需要严丝合缝到经得起反复推敲,我们希望让这种微妙的违和感作为小小的伏笔,这会成为整个故事的趣味的重要组成。构建这样的诡计并不容易,我们会在少数节点给出一点小例子作为抛砖引玉。我们更希望掌局者能构思出更多精妙的诡计,这也是主持本场游戏的挑战与乐趣之一。

最后,当你的特工问出一些难以合理躲过的问题时(对于敏锐的特工来说,这种幻象瞒不了太久),那还是大大方方地如实回答吧——特工会发现自己被卷入更深的困惑,并且为了这些乐趣歪曲真相是不值得的。