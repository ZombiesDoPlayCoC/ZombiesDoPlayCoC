\section{导言}

这是一篇受到绿色三角洲战役《不可思议的风景》启发而创作的模组,展开了绿色三角洲和黄衣之王对抗历程中的一个私密的小角落。本模组为单人游玩设计。

一个不够谨慎的特工在日常的工作中接触了黄衣之王的影响,并在这场难以抗拒的调查中,逐渐变成了黄衣之王的腐蚀者,而这个不幸的特工就是玩家扮演的特工自己。特工心理上的保护机制让他短暂地忘记了调查过程中一切,现在特工有了一次重新来过的机会,而特工不可能有第三次机会。

在整局游戏中,特工看见的世界就是记忆、幻象和现实世界的糅合。这种割裂、令人困惑的场景会一直伴随着玩家直到游戏的结束,并随着游戏的进展变得越来越歇斯底里和难以分辨。甚至在游戏的后期,特工会短暂地进入哈利湖湖底的一列飞驰的列车,完全进入幻象的世界。

游戏一开始,特工会接到幻象中的绿色三角洲的上司布置的任务。这个“上司”要求特工去杀死一个被非自然腐化的同僚,调查这个同僚腐化的原因。而在调查过程中,特工会意识到自己疯狂的事实并找回自己失去的记忆。无论是调查的推进,还是记忆的恢复,都会将特工引向自己疯狂的原因:他曾目睹了一场疯狂的表演。

与此同时,这个话剧的作者正在计划着一场面向2007年互联网的直播——这场直播的顺利播放会给社会带来难以想象的后果。特工需要在直播开始之前找到阻止直播的条件:时间、地点和方式,并顺利终结这一切。

最后,在主线之外,作为被腐蚀者,割裂的感知也会让维持特工的正常生活变得困难,甚至成为新的污染源,在无意间传递黄衣之王的旨意,扭曲自己曾经温馨的生活。这主线之外的故事也是模组叙事的重要一环,这一切都会让特工在探索真相时面临心理崩溃的考验。

特工会在这场探索中逐渐发现自己面临着苦痛的选择:冒着被黄衣之王彻底打垮的风险寻回记忆,抓住渺茫的希望破解危机;抑或逃离职责,在疯狂敲门之前安然度日?而揭露记忆的动机,是黄衣之王的诱惑,是特工不挠的责任心,还是对亚当森的追思?这都交给玩家自己回答。